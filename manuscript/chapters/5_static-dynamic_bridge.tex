\chapter{Static dynamic bridge}
\label{chap.static_dynamic_bridge}

%\section{Coexistence between the static world and the dynamic world}
%introduction

Image processing communities like to have bridges with interpretable language such as Python or Matlab, to interface
with their favorite tools, algorithms and/or facilities. As an example, with Python, the module
NumPy~\cite{oliphant.2006.numpy} is community standard which is heavily used. Henceforth, to broaden the usage of our
library, we should be able to provide a way to communicate between our library and NumPy. However here is a showstopper:
we only distribute source code, we don't hand over binaries. Indeed, genericity in C++ is achieved via usage of template
metaprogramming. One caveat of it is that the C++ compiler cannot generate a binary until it knows which type (of image,
of value) will be used. But we don't know this information: the user (on Python's end) is not going to recompile our
library each time he has another set of types to exercise. From here, there are still multiple ways to achieve our goal.

First option is to embark a JIT (Just-in-time) compiler whose job would be to generate the binaries and bindings just as
they are used. This solution brings speed (excluding the first run that includes the compilation time) and unrestrained
genericity. However we are now bound to specificities of a compiler vendor and loose platform portability.

Another option is to type-erase our types to enables the use of various concrete types through a single generic
interface. This would translate into a class hierarchy whose concrete classes are on the leaves (thus, whose value type
and dimension are known). This induces a non negligible slow down but allow us to keep the genericity and portability at
the cost of maintaining the class hierarchy.

Type generalization can also be considered: cast everything into a super-type that is suitable for the vast majority of
cases. For instance, we could say that we have a super-type \texttt{image4D<double>} into which we can easily cast
sub-types such as \texttt{image2D<int>} or \texttt{image3D<float>}. Of course we would loose the generic aspect and
induce non negligible speed cost. Although portability is kept.

And finally there is the dynamic dispatch. It consists in embarking dynamic information at runtime about types, and
dispatch (think of switch/case) to the correct facility which can handle those types. The obvious caveat is the cost of
maintenance induced by the genericity as we would have a number of possible dispatches that grow in a multiplicative way
with the number of handled types. Which is not very generic. On the other hand there is almost no speed loss and the
portability is guaranteed. Theoretical models exists that could bring solutions to lower the number of dispatcher to
write, such as multi-method~\cite{pirkelbauer.2010.multimethods}. Unfortunately they are currently not part of C++.

\FIXME{TODO: schémas static/dynamic worlds}

\section{Design hybrid solution, motivations and choices}

\FIXME{TODO: schémas solution hybride, arbre de dispatching etc.}

\section{Hybrid solution: $n*n$ dispatch thanks to variants}

\FIXME{TODO: liens, plus d'examples}


In Pylene we have chosen an hybrid solution between type-erasure and dynamic dispatch. The aim is to have a set of known
types for which we have no speed cost as well as continuing to handle other types to remain generic. To achieve this
goal, we have worked together with Célian Gossec~\cite{gossec.2019.pybind}, a student co-supervised by the authors of
this report, in order to type-erased the most important types (images) as well as the algorithms. We then embark runtime
information about carried types in the type-erased object. When the algorithm is called on the type-erased object, we
attempt to cast this object into a known set of types (dimensions: 2, 3, 4; value: int, double, rgba). Should the cast
succeed then we can call the fast, optimized routine. Otherwise we fallback onto an algorithm relying on dynamic
dispatch even for its most basic operation (e.g. addition).

Let us illustrate this path with an example: the stretch algorithm. First let's see the naive algorithm:
\begin{minted}{C++}
  template <typename T>
  image2d<float> stretch(const image2d<T>& src)
  {
    auto res = image2d<float>(src.width(), src.height());
    auto values_span = src.values();
    std::transform(values_span.begin(), values_span.end(), res.values().begin(),
      [](T val) -> float
      {
        return static_cast<float>(val) / std::numeric_limits<T>::max();
      }
    );
    return res;
  }
\end{minted}

One should observe that the function parameter is templated by a type \texttt{T}. This induces that "span" further in
the code is also templated. Both needs to be type erased. Furthermore, getting the max value of the type \texttt{T} is
also an issue that needs to be abstracted away.

We aim at obtaining this prototype:

\begin{minted}{C++}
  image2d<> stretch_py(const image2d<>& src);
\end{minted}

Here \texttt{image2d<>} which is also \texttt{image2d<void>} is the type-erased type of \texttt{image2d<T>}. This means
that we have hierarchy where \texttt{image2d<T>} inherits from \texttt{image2d<void>}. We then introduce an intermediate
step:

\begin{minted}[linenos]{C++}
  template <typename T>
  struct apply_stretch_t
  {
    auto operator()(const image2d<>& src)
    {
      return stretch(*src.cast_to<T>());
    }
  };

  image2d<float> stretch(const image2d<>& src)
  {
    return visit<apply_stretch_t>(src.type().tid(), src);
  };
\end{minted}

This piece of code is interesting in many ways. It introduces the way we dispatch according to the known types (line
12): there is the embedded runtime type information we are going to use for the dispatch. Then there is the call to
\texttt{visit} parametrized by the templated structure \texttt{apply\_stretch\_t}. This visitor will statically
instantiate the correct \texttt{apply\_stretch\_t} with the correct dynamic type (\text{src.type().tid()}) and call a
non type-erased version of the function stretch after having cast the values of the images. However, for this to work we
still have to tweak the first implementation a little.

\begin{minted}[linenos]{C++}
  template <class T>
  image2d<float> stretch(const image2d<T>& src)
  {
    auto        res  = image2d<float>(src.width(), src.height());
    auto        span = src.values();
    const auto& vs   = src.get_value_set();
    std::transform(span.begin(), span.end(), res.values().begin(),
                   [&vs](auto val) -> float
                   {
                     auto tmp = vs.max();
                     return vs.template cast<float>(val) / vs.template cast<float>(tmp);
                   });
    return res;
  }
\end{minted}

We introduce a new tool: the value-set (line 6). This value-set is a type-erased way to provide basic operations on
types such as casting, division, addition, getting the global max etc. This powerful tool, also embedded in the image,
allow us to write the algorithm in a generic way lines 10-11.

Thanks to this design, we have been able to type-erase our image types so that our algorithms can be called through
python via bindings generated by pybind~\cite{jakob.2017.pybind11}. As an example, we can then call our stretch
algorithm this way:

\begin{minted}{python}
  import pylena as pln, imageio, numpy as np
  img_in = imageio.imread("lena.png")
  np_arr = np.array(img_in, dtype='int8')
  img = pln.image2d(np_arr)
  print(timeit.timeit('img_out = pln.stretch(img)', number=1000,
        globals=globals()))
  >> 0.24129085899949132 # Seconds for 1000 cycles

  # Using a type that isn't int8 or int16
  np_arr64 = np.array(img_in, dtype='int64')
  img64 = pln.image2d(np_arr64)
  print(timeit.timeit('img_out64 = pln.invert(img64)', number=1000,
        globals=globals()))
  >> 26.124844277999728 # Seconds for 1000 cycles
\end{minted}

We can observe a hundred time factor between the fast and optimized path and the slow dynamically dispatched path.

\section{Performances & overhead}

\FIXME{TODO: courbes de performances mesurées depuis Python}

\section{JIT-based solutions: pros. and cons.}

\FIXME{TODO: état de l'art, schémas de fonctionnement du JIT}
