\chapter*{Conclusion}
\label{conclusion.chap.conclusion}

Through concrete benchmarks and examples, we have shown how to leverage genericity nowadays without slowing down the
performances. There are several types of genericity which have been presented, as well as several widely used
implementation of them in the industry. Our take offer a new approach to reach the goal of having one code for several
algorithm, one algorithm for several image types. Furthermore, we introduce meta-algorithms (canvas) that are based on
behavior patterns known once the image type is known. We also show how C++ template metaprogramming techniques allow not
to impact the performances despites the indirections induced. Finally, we show how an approach based on properties (on
image type as well as on external data such as structuring elements) can be beneficial to introduce customization points
to take advantage of opportunities to increase performances. When coupling both properties and algorithm canvas, it
becomes standard for a user to write efficient and generic algorithm by default.

It is also shown how we were able to abstract two of the three main families of algorithms. First are point-wise
algorithms that can all be expressed through the \emph{views}.\emph{Views} enables streamlining the writting process of
image processing pipelines so that it is shorter, efficient and expressive. Second are local algorithm whose
problematics (border management, structuring elements, pass number) can all be abstracted away behind a canvas hiding
the complexity and taking advantage of opportunities to increase performance for you.

The solutions presented in this paper do have some disadvantages, such as the readability when using local algorithm
canvas, code-bloat due to heavy instantiation of C++ templates especially in the views and finally the lack of
availability to the dynamic (prototyping) world by default. Indeed, C++ metaprogramming needs to be compiled at the time
of its usage preventing direct link with dynamic languages such as python. There is on-going work to introduce dynamic
dispatch at some key points to reduce code size without impairing the performance. For instance, a dynamic dispatch to
select the correct version of an algorithm has much less impact than a dynamic dispatch when accessing the value of a
pixel. Further work is required to improve the static-dynamic bridge and bring the capabilities of the techniques
presented in this paper to the dynamic world, and in particular, python which is vastly used in the image processing
world.

\vspace{1cm}

Through a simple example, we have shown a step-by-step methodology to make an algorithm \emph{generic} with zero
overhead~\footnote{The zero-cost abstraction of our approach is not argued here but will be discussed in an incoming
  paper with a comparison with the state of the art libraries}. To reach such a level of genericity and be able to write
versatile algorithms, we had to \emph{abstract} and \emph{define} the most simple and fundamental elements of the libray
(e.g. \emph{image}, \emph{pixel}, \emph{structuring element}). We have shown that some tools of the Modern C++, such as
\emph{concepts}, greatly facilitate the definition and the usage of such abstractions. These tools enable the library
designer to focus on the abstraction of the library components and on the user-visible development. The complex
\emph{template meta-programming} layer that used to be a large part of \emph{C++ generic programming} is no more
inevitable. In this context, it is worth pointing out the approach is not limited Image Processing libraries but works
for any library that wants to be modernized to augment its productivity.

As one may have noticed, the solution presented in this paper is mostly dedicated to C++ developer and C++ end-user.
Unlike dynamic environments (such as Python), C++ is not the most appropriate language when one has to prototype or
experiment an IP solution. As a future work, we will study the conciliation of the \emph{static genericity} from C++
(where types have to be known at compile time) with a \emph{dynamic} language (with a run-time polymorphism) to allows
the interactive usage of a C++ generic library.

The main issue, if we consider having an effective Python binding as our future main objective, is intrinsic to the way
generic C++ code works. To be able to bind Python and C++, we need a compiled binary that will be called from Python.
However, generic code is, by essence, \emph{header-only}. That means an abstraction layer that instantiate our generic
code with predefined types is needed to generate the binary code required by Python. However this induces two
disadvantages. First, there is no more \emph{zero-overhead} as the abstraction layer based on virtual dispatch will
impact performances. Second, it will be very hard to consider injecting new types from the Python side into the library.

A solution to the last disadvantage would be to consider solutions that compiles our C++ code on the fly when we know
the types injected from the Python side. There already exists a tool based on Clang/LLVM that makes this integration
relatively feasible. Namely there is Pythran~\cite{guelton.2015.pythran}, an ahead of time compiler that compiles
annotated Python code into optimized native code (with SIMD instructions). This would allow the injection of Python code
into our C++ at runtime. Henceforth this would provide a solution to inject new types defined on python side into the
already compiled C++ binary code and have the virtual dispatch somehow use it.

Another approach would be to consider a solution using another very promising tool:
cppyy~\cite{wimtlplavrijsen.2016.cppyy} which is a an automatic Python/C++ binding generator designed for large scale
programs. The C++ interpreter is based on Cling~\footnote{https://root.cern.ch/cling} (an interpreter based on
Clang/LLVM and maintained by CERN). This would allow us to compile our generic C++ code directly from our Python code
and have our bindings automatically generated. As a future work, we will implement both approaches to benchmark them in
order to measure which one is the most effective for the many use-cases we have.

\vspace{1cm}

During our $2^{nd}$ year of work on our Ph.D. we have broaden our horizon and seen many issues as well as studied
potential leads of solutions.

For traversing of an image, a satisfactory solution has been found that does not hinder performances, though some small
issues remain to be studied later.

Concerning the way to handle an image's borders when processing a local algorithm, we have studied the question in depth
and exercised the presented design against different kind of problem such as structuring elements which are static,
dynamic or adaptive. All of these are within scope and handled by the solution presented in this paper.

The static dynamic bridge proved to be a big challenge as there was very little experience feedback from existing
solution experimented by other libraries. It took a lot of incremental trial and error to design the hybrid solution
that would perform with satisfactory performances for common use cases without duplicating code. This question still
remains open as we would like to study the usability of bringing in a C++ interpreter such as
Cppyy~\cite{wimtlplavrijsen.2016.cppyy} (based on clang) to dynamically generate python bindings on the fly when those
are missing for specific types. This would allows us to inject user type from python into the C++ core library code. The
solution of code generation through tools like AsmJIT~\cite{kobalicek.2011.asmjit} also remains open for future work.

Finally the reflection about how we can bring parallelism and more generally heterogeneous computing into the library
brought us to study different approaches, different paradigms and think about how our data structure operate with the
on-going computation. We attempted to bring a solution with the algorithm canvas though we did not took advantage of it
for the moment. For instance, we do not dispatch CUDA kernel yet. This question remains open and can still be studied
further.

To conclude, the priority in the future will be put into having a first stable release of the library. Indeed, we want
to provide the library for the students so that they can give feedbacks on its usability. We will study both the
usability of the C++ aspect and the python aspect.


\clearpage


\begin{itemize}
  \item Synthèse générale
  \item Réponse à la problématique d'Introduction
  \item confrontation à d'autres travaux de recherche ayant donné naissance à une bibliothèque de TI
  \item Ouvertures, perspectives, limites
        \begin{itemize}
          \item continuité JIT
          \item ce qu'il reste à faire
        \end{itemize}
\end{itemize}