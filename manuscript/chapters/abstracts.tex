\section*{Abstract}
\label{sec:abstract}
C++ is a multi-paradigm language that enables the initiated programmer to set up efficient image processing algorithms.
This language strength comes from many aspects. C++ is high-level, so this enables developing powerful abstractions and
mixing different programming styles to ease the development. At the same time, C++ is low-level and can fully take
advantage of the hardware to deliver the best performance. It is also very portable and highly compatible which allows
algorithms to be called from high-level, fast-prototyping languages such as Python or Matlab. One of the most
fundamental aspects where C++ really shines is generic programming. Generic programming makes it possible to develop and
reuse bricks of software on objects (images) of different natures (types) without performance loss. Nevertheless,
conciliating genericity, efficiency, and simplicity at the same time is not trivial. Modern C++ (post-2011) has brought
new features that made it simpler and more powerful. In this thesis, we first explore one particular C++20 aspect: the
concepts, in order to build a concrete taxonomy of image related types and algorithms. Second, we explore another
addition to C++20, ranges (and views), and we apply this design to image processing algorithms and image types in order
to solve issues such as how hard it was to customize/tweak image processing algorithms. Finally, we explore
possibilities regarding how we can offer a bridge between static (compile-time) generic C++ code and dynamic (runtime)
Python code. We offer our own hybrid solution and benchmark its performance as well as discussing what can be done in
the future with JIT technologies. Those three axes aim to address the issue of generic programming all the while
remaining efficient and easy to use.

\bigskip

\bigskip

\noindent C++ est un langage de programmation multiparadigme qui permet au développeur initié de mettre au point des
algorithmes de traitement d'images. La force de langage se base sur plusieurs aspects. C++ est haut-niveau, cela
signifie qu'il est possible de développer des abstractions puissantes mélangeant plusieurs styles de programmation pour
faciliter le développement. En même temps, C++ est bas-niveau et peut pleinement tirer partie du matériel pour fournir
un maximum de performances. Il est aussi portable et très compatible ce qui lui permet de se brancher à d'autres
langages de haut niveau pour le prototypage rapide tel que Python ou Matlab. Un des aspects les plus fondamentaux où le
C++ brille, c'est la programmation générique. La programmation générique rend possible le développement et la
réutilisation de briques logiciel comme des objets (images) de différentes natures (types) sans avoir de perte au niveau
performance. Néanmoins, concilier la généricité, la performance et la simplicité d'utilisation tout en même temps n'est
pas trivial. Le C++ moderne (post-2011) amène de nouvelles fonctionnalités qui le rendent plus simple et plus puissant.
Dans cette thèse, nous explorons en premier un aspect particulier du C++20 : les concepts, dans le but de construire une
taxonomie des types relatifs au traitement d'images. Deuxièmement, nous explorons une autre fonctionnalité ajoutée au
C++20 : les ranges (et les vues). Nous appliquons ce design aux algorithmes de traitement d'images et aux types d'image,
dans le but résoudre les problèmes liés, notamment, à la difficulté qu'il existe pour customiser les algorithmes de
traitement d'image. Enfin, nous explorons les possibilités concernant la façon dont il est possible de construire un
pont entre du code C++ générique statique (compile-time) et du code Python dynamique (runtime). Nous fournissons une
solution hybride et nous mesurons ses performances. Nous discutons aussi les pistes qui peuvent être explorées dans le
futur, notamment celles qui concernent les technologies JIT. Ces trois axes visent à résoudre la problématique
concernant la programmation générique tout en veillant à ce que le code reste efficace et accessible.
