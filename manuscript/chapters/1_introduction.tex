\chapter{Introduction}
\label{chap:introduction}

\section*{Outline}

\lettrine[lines=2]{N}{owadays} \emph{Computer Vision} and \emph{Image Processing (IP)} are omnipresent in the day to day
life of the people. It is present each time we pass by a CCTV camera, each time we go to the hospital do an MRI, each
time we drive our car and pass in front of a speed camera and each time we use our computer, smartphone or tablet. We
just cannot avoid it anymore. The systems using this technology are sometimes simple and, sometimes, more complex. Also
the usage made of this technology has several different purposes: space observation, medical, quality of life
improvement, surveillance, control, autonomous system, etc. Henceforth, Image processing has a wide range of research
and despite having a mass of previous of work already contributed to, there are still a lot to explore.

Let us take the example of a modern smartphone application which provides facial recognition in order to recognize
people whom are featuring inside a photo. To provide accurate result, this application will have to do a lot of
different processing. Indeed, there are a lot of elements to handle. We can list (non exhaustively) the weather, the
light exposition, the resolution, the orientation, the number of person, the localization of the person, the distinction
between humans and objects/animals, etc. All of these is in order to finally recognizing the person(s) inside the photo.
What the application does not tell you is the complexity of the image processing pipeline behind the scene that can not
even be executed in its entirety on one's device (smartphone, tablet, \ldots). Indeed, image processing is costly in
computing ressources and would not meet the time requirement desired by the user if the entire pipeline was executed on
the device. Furthermore, for the final part which is "recognize the person on the photo", one needs to feed the
pre-processed photo to a neural network trained beforehand through deep learning techniques in order to give an accurate
response. There exists technologies able to embed neural network into mobile phone such as
MobileNets~\parencite{howard.2017.mobilenets} but it is still limited. It can detect a human being inside a photo but
not give the answer about who this human being is for instance. That is why, accurate neural network system usually are
abstracted away in cloud technologies making them available only via Internet. When uploading his image, the user does
not imagine the amount of technologies and computing power that will be used to find who is on the photo.

We now understand that in order to build applications that interact with photos or videos nowadays, we need to be able
to do accurate, fast and scalable image processing on a multitude of devices (smartphone, tablet, \ldots). In order to
achieve this goal, image processing practitioners needs to have two kinds of tools at their disposal. One will be the
prototyping environnement, a toolbox which allow the practitioner to develop, test and improve its application logic.
The other is the production environnement which deploy the viable version of the application that was developed by the
practitioner. Both environment may not have the same needs. On one hand, the prototyping environment usually requires to
have a fast feedback loop for testing, an availability of state-of-the-art algorithms and existing software. This way
the practitioner can easily build upon them and be fast enough in order not to keep waiting for results when testing
many prototypes. On the other hand, the production environment must be stable, resilient, fast and scalable.

When looking at standards in the industry nowadays, we notice that Python is the main choice for prototyping. Also,
Python may not be enough so that a viable prototype can be pushed in production with minimal changes afterwards. We find
it non-ideal that the practitioner cannot take advantages of many optimisation opportunities, both in term of algorithm
efficiency and better hardware usage, when proceeding this way. It would be much more efficient to have basic low level
building blocks that can be adapted to fit as mush use cases as possible. This way, the practitioner can easily build
upon them when designing its application. We distinguishes two kind of use cases. The first one is about the
multiplicity of types or algorithms the practitioner is facing. The second one is about the diversity of hardware the
practitioner may want to run his program. The goal is to have building blocks that can be intelligent enough to take
advantage of many optimization opportunities, with regard to both input data types/algorithms and target hardware. Then
the practitioner would have a huge performance improvement, by default, without specifically tweaking its application.
As such, the concept of genericity was introduced. It aims at providing a common ground about how an image should behave
when passed to basic algorithms needed for complex applications. This way, in theory, one only needs to write the
algorithm once for it to work with any given kind of image.


\section*{Different data types and algorithms}

In Image Processing, there exists a multitude of image types whose characteristics can be vastly different from one
another. This large specter is also resulting from the large domain of application of image processing. For instance,
when considering photography we have 2D image whose values can vary from 8 bits grayscale to multiple band 32-bits color
scheme storing informations about the non-visible specter of human eye. If we consider another domain of application,
such as medical imaging, we now can consider sequence of images such as sequence of 3D image for an MRI for instance.
More broadly there are two orthogonal constituent of an image: its topology (or structure) and its values. However,
there are two more aspectes to consider here. Firstly, image processing provide plenty of algorithms that can or cannot
operate over specific data types. There are also different kind of algorithms. Some will extract informations,(e.g.
histogram) other will transform the image point-wise (e.g. thresholding), and some other will even combine several image
to render a different kind of informations (e.g. background substraction). There are many simple algorithms and also
many complex algorithms out there. Secondly, there are orbiting data around image types and algorithms that are also
very diverse and necessary for their smooth operation. Indeed, a dilation algorithm will also need an additional
information: the dilation disc. A thresholding algorithm may be given a threshold. A convolution filter requires a
convolution matrix to operate. That is why, when considering both image types and algorithms, we need a 3D-chart
(illustrated in~\cref{fig:int.possibility_space}) to enumerate all possibilities, where one axis is the image topology,
one axis is the color scheme and one axis enumerate the additional data that can be associated to an image.


\begin{figure}[tbh]
  \centering
  \includegraphics{figs/possibility_space}
  \caption{Illustration of the specter of the multitude of possibilities in the image processing world.}
  \label{fig:int.possibility_space}
\end{figure}


\section*{Different user profiles and their use cases}

\paragraph{The end user} is a non programmer user who wants to occasionally use image processing software through
UI-rich interface, such as Adobe Photoshop~\parencite{adobe.2019.photoshop} or The GIMP~\parencite{gimp.2019}. Its
skills are non-relevant as the end user is using the software to get work done even though he does not fully understand
the underlying principles. For instance, the end user will want to correct the brightness of an image, of remove some
impurities from a face or a building. The end user does not want to build an application but wants to save time. The
needs of the end user mainly revolves around a clean and intuitive software UI as well as a well as support for
mainstream image types and operation a photograph may need to do.

\paragraph{The practitioner} is what we are called when we first approach the image processing area. A practitioner is
the end user of image processing libraries. Its skills mainly revolve around applied mathematics for image processing,
prototyping and algorithms. A practitioner aims at leveraging the features the libraries can offer to build his
application. For instance, a practitioner can be a researcher in medical imaging, an engineer build a facial recognition
application, a data scientist labeling its image set, etc. The needs of practitioner are mainly revolving around a fast
feedback loop. The developing environment must be easily accessible and installable. This way a practitioner can judge
quickly wether one library will answer his needs. The documentation of the library must be exhaustive and didactic with
examples. When prototyping, the library must provide fast feedback loops, as in a python notebook for instance. Finally
it must be easily integrated in a standard ecosystem such as being able to work with NumPy's array natively without
imposing its own types. To sum up, practitioner's programmatic skills do not need to be high as his main goal is to
focus on algorithms and mathematics formulas.

\paragraph{The contributer} is an advanced user of a library who is very comfortable with its inner working, philosophy,
aims, strengths and potential shortcomings. As such, he is able to add new specific features to library, fix some
shortcomings or bugs. Usually a contributer is able to add a feature needed for a practitioner to finish his
application. Furthermore he can then contribute back his features to the main project via pull requests if it is
relevant. This way, a maintainer will assess the pull request and review it. The two main points of a contributer are
his deep knowledge of a library and his ability to write code in the same language as the source code of it. Also, a
contributer must have knowledge of coding best practices such as writing unit tests which are mandatory when adding a
feature to an existing library. To facilitate contribution, a library must provide clear guidelines about the way to
contribute, be easy to bootstrap and compile without having heavy requirements on dependencies. The best case would be
that the library is handled by standard packages managers such as system apt or python conan.

\paragraph{The maintainer} is usually the creator, founder of the library or someone that took over the project when the
founder stepped back. Also, when a library grows, it is not rare that regular contributers end up being maintainer as
well to help the project. The maintainer is in charge of keeping alive the project by fulfilling several aspects:
upgrade and release new features according to the user (practitioner) needs and the library philosophy. Also, a library
may not evolve as fast as the user may want it because of lack of time from maintainers. A lot of open source projects
are maintained by volunteers and lack of time is usually the main aspect slowing development progress. The maintainer is
also in charge of reviewing all the contributers pull requests. He must check if they are relevant and completed enough,
(for instance, presence of tests and documentation) to be integrated in the project. Indeed, merging a pull requests
equals to accepting to take care of this code in the future too. It means that further upgrade, bug fix, refactoring of
the project will consider this new code too. If the maintainer is not able to take care of this code then it should
probably not be integrated in the project in the first place. Any project and library has its maintainers. A maintainer
is someone very familiar with the inner working and architectural of the project. He is also someone that has some
history in the project to understand why some decisions has been made, what choices has been made at some points and
what the philosophy of the project is. It is important to be able to refuse a contribution that would go contrary to the
philosophy of the project, even a very interesting one. Finally the profile of a maintainer is one of a developer that
is used to the standard workflow in open source based on: forks, branches, merge/pull requests and continuous
integration.

\section*{Different tools}

Before stating the topic of the thesis, it is important to enumerate the different kind of tools the current market has
to offer to know where we will be positioning ourself.

\paragraph{Graphic editors} are what neophyte thinks about when they imagine what image processing is. Those are tools
that allow a non expert user to apply a wide array of operation on an image from an intuitive GUI in a way the user does
not have to understand the underlying logic behind each and every operation he is applying. Such tools are usually large
complex software such as The GIMP~\parencite{gimp.2019} or Photoshop~\parencite{adobe.2019.photoshop}. Their aim is to
be useable by end users while supporting a large set of popular image format.

\paragraph{Command line utilities} are binaries that perform one operation or more invocable from a console interface or
from a shell script through a command line interface (CLI). This CLI usually offers several options to pass data and/or
information to the programs in order to have an processing happening. The informations can be, for instance, the input
image path, the ouput image name and the name of a mathematical morphology algorithm to apply. Usually command line
utilities come as projects such as ImageMagick~\parencite{imagemagick.2021},
GraphicsMagick~\parencite{graphicsmagick.2021} or MegaWave~\parencite{froment.2012.megawave,froment.2004.megawave2}.

\paragraph{Visual programming environment} are software that allow the user to graphically and intuitively link one or
several image processing operations while interactively displaying the result. The processing can easily be modified and
the results are updated accordingly. Those software are usually aimed at engineer or researchers doing prototyping work
not exclusive to image processing. Mathcad~\parencite{ptc.2019.mathcad} is a good example of such a software.

\paragraph{Integrated environment} are feature-rich platforms for scientists oriented toward prototyping. Those
platforms provides a fully functional programming language and a graphical interface allowing the user to run commands
and scripts as well as viewing results and data (image, matrices, etc.). The most well-known integrated environnement
are Matlab~\parencite{mathworks.2020.matlab}, Scilab~\parencite{scilab.2020}, Octave~\parencite{gnu.2021.octave},
Mathematica~\parencite{wolfram.2020.mathematica} and Jupyter~\parencite{kluyver.2016.jupyter} notebooks.

\paragraph{Package for dynamic language} has known a surge in development these last few years and a multitude of
libraries has been brought to dynamic languages this way. For instance, let us consider the python programming language.
There are two main package provider: PyPi~\parencite{pypi.2021} and Conda~\parencite{anaconda.2020}. Both allow to
install packages to enable the user to program his prototypes in Python very quickly. In image processing, there are
packages such as Scipi~\parencite{jones.2006.scipy}, NumPy~\parencite{oliphant.2006.numpy},
scikit-image~\parencite{vanderWalts.2014.scikit-image}, Pillow~\parencite{clark.2021.pillow} as well as binding for
OpenCV~\parencite{bradski.2000.opencv}.

\paragraph{Programming libraries} is the most common tool available out there. They are a collection of routines,
functions and structures providing features through a documentation and binaries. They require the user to be proficient
with a certain programming language and also to be able to integrate a library into his project. For image processing we
have: IPP~\parencite{taylor.2004.intel}, ITK~\parencite{johnson.2013.ITKSoftwareGuideThirdEdition},
Boost.GIL~\parencite{bourdev.2006.bgil}, Vigra~\parencite{kothe.2011.generic}, GrAL~\parencite{berti.2006.gral},
DGTal~\parencite{coeurjolly.2016.dgtal}, OpenCV~\parencite{bradski.2000.opencv}, CImg~\parencite{tschumperle.2012.cimg},
Video++~\parencite{garrigues.2014.video++}, Generic Graphic Library~\parencite{kolas.2000.gegl}
Milena~\parencite{levillain.2009.ismm,levillain.2010.icip} and
Olena~\parencite{olena.2000.www,levillain.2011.phd,geraud.2012.hdr,levillain.2014.ciarp}.

\paragraph{Domain Specific Languages (DSL)} are tools developed when a library developer deem he is unable to express
the concepts and abstraction layers he wants to express through publishing a library. In this case, the barrier is often
the programming language itself and so the developer does think that another layer of abstraction above the programming
language would be a good thing. It leads to the genesis of a new programming language in some cases like
Halide~\parencite{ragankelley.2013.halide} and SYCL~\parencite{brown.2019.heterogeneous,wong.2019.heterogeneous} but can
also be a case of having the current programming language be "upgraded" to include another subset of features that are
not natively included. This is often the case in C++ where we have in-language DSL like
Eigen~\cite{guennebaud.2010.eigen}, Blaze~\cite{iglberger.2012.blaze,iglberger.2012_1.blaze},
Blitz++~\cite{veldhuizen.2000.blitz} or Armadillo~\cite{sanderson.2016.armadillo}. They leverage a possibility of the
C++ programming language (\emph{expression templates}~\cite{veldhuizen.1995.expression}) to achieve it.


\section*{Topic of thesis}

%Définitions du périmètre de la bibliothèque et de ses objectifs:
%\begin{itemize}
%  \item Performance
%  \item Facile d'utilisation (UX client)
%  \item Facile de développement (Core developer xp)
%  \item Versatilité des types d'images
%  \item Utilisable depuis Python, Orientée MM
%\end{itemize}

In the end, \FIXME{(find the quote)} it is often known that there is a rule of three about genericity, efficiency and
ease of use. The rule states that one can only have two of those items by sacrificing the third one. If one wants to be
generic and efficient, then the naive solution will be very complex to use with lot of parameters. If one wants a
solution to be generic and easy to use, then it will be not very efficient by default. If one wants a solution to be
easy to use and efficient then it will not be very generic. In this thesis, we chose to work on an image processing
library though continuing the work on Pylene~\cite{carlinet.2018.pylena}. But only working at library level would
restrict the usability of our work and thus its impact. That is why we aim to reach prototyping users through providing
a package that can be used in dynamic language such as Python without sacrificing efficiency. In particular, we aim to
be usable in a jupyter notebook. It is a very important goal for us to reach a usability able to permeate into the
educational side which is a strength of Python. In this library, we demonstrate how to achieve genericity and efficiency
while remaining easy to use all at the same time. The scope of this library would be to specialize in mathematical
morphology as well as providing very versatile image types. We leverage the modern C++ language and its many new
features related to genericity and performance to break this rule in the image processing area. Finally, we attempt,
through a static/dynamic bridge, to bring low level tools and concepts from the static world to the high level and
dynamic prototyping world for a better diffusion and ease of use.

With this philosophy in mind, this manuscript aims at presenting our thesis work related to the C++ language applied to
the Image Processing domain. It is organized as followed:

\paragraph{Genericity~\ref{chap:genericity}} presents a state-of-the-art overview about the notion of genericity. We
explain its origin, how it has evolved (especially within the C++ language), what issues it is solving, what issues it
is creating. We explain why image processing and genericity work well together. Finally we tour around existing
facilities that allows genericity (intrinsically restricted to compiled language) to exists in the dynamic world (with
interpreted languages such as Python).

\paragraph{Images and Algorithms taxonomy~\ref{chap:image.algorithms.taxonomy}} presents our first contribution
which is a comprehensive work in the image processing area around the taxonomy of different images families as well of
different algorithms families. This part explains, among others, the notion of concept and how it applies to the image
processing domain. We explain how to extract a concept from existing code, how to leverage it to make code more
efficient and readable. We finally offer our take about a collection of concepts related to image processing area.

\paragraph{Images Views~\ref{chap.image_views}} presents our second contribution which is a generalization of the
concept of View (from the C++ language, the work on ranges~\parencite{niebler.2018.ranges}) to images. This allows the
creation of lightweight, cheap-to-copy images. It also enable a much simpler way to design image processing pipeline by
chaining operations directly in the code in an intuitive way. Ranges are the cement of news design to ease the use of
image into algorithms which can further extend their generic behavior. Finally we discuss the concept of lazy evaluation
and the impacts of views on performances.

\paragraph{Static dynamic bridge~\ref{chap.static_dynamic_bridge}} presents our third contribution which is a way to
grant access to the generic facilities of a compiled language (such as C++) to a dynamic language (such as Python) to
ease the gap between the prototyping phase and the production phase. Indeed, it is really not obvious to be able to
conciliate generic code from C++ whose genericity is resolved at compilation-time (we call this the "static world"), and
dynamic code from Python which rely on pre-compiled package binaries to achieve an efficient communication between the
dynamic code and the library (we call this the "dynamic world"). We also cannot ask of the user to provide a compiler
each time he wants to use our library from Python. In this part, we discuss what are the existing solutions that can be
considered as well as their pros and cons. We then discuss how we designed an hybrid solution to make a bridge between
the static world and the dynamic world: a static-dynamic bridge.
