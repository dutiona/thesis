\chapter{Static-dynamic bridge}
\label{appendix:static-dynamic-bridge}


\section{Ndimage module}
\label{appendix:static-dynamic-bridge.ndimage}

Here is the code of the ndimage module, including the helpers for the type conversion.

\paragraph{ndimage.hpp}
\label{appendix:static-dynamic-bridge.ndimage.hpp}

\begin{minted}{C++}
#include <mln/core/image/ndbuffer_image.hpp>

#include <pybind11/numpy.h>
#include <pybind11/pybind11.h>


namespace mln::py
{
  mln::ndbuffer_image from_numpy(pybind11::array arr);

  pybind11::object to_numpy(const mln::ndbuffer_image& img);

  void init_pylena_numpy(pybind11::module& m);
} // namespace mln::py

namespace pybind11::detail
{
  template <>
  struct type_caster<mln::ndbuffer_image>
  {
    PYBIND11_TYPE_CASTER(mln::ndbuffer_image, _("numpy.ndarray"));

    bool load(handle h, bool)
    {
      pybind11::array arr = reinterpret_borrow<pybind11::array>(h);
      value               = mln::py::from_numpy(arr);
      return true;
    }

    static handle cast(const mln::ndbuffer_image& img, return_value_policy, handle)
    {
      return mln::py::to_numpy(img).inc_ref();
    }
  };
} // namespace pybind11::detail
\end{minted}

\paragraph{ndimage.cpp}
\label{appendix:static-dynamic-bridge.ndimage.cpp}

\begin{minted}{C++}
#include "ndimage.hpp"

#include <mln/core/image/ndimage.hpp>

#include <fmt/core.h>
#include <pybind11/cast.h>

#include <cassert>
#include <stdexcept>
#include <string>


namespace
{

  namespace details
  {
    template <mln::sample_type_id T>
    static pybind11::dtype dtype_of()
    {
      return pybind11::dtype::of<typename mln::sample_type_id_traits<T>::type>();
    }
  } // namespace details

  pybind11::dtype get_sample_type(mln::sample_type_id type)
  {
    switch (type)
    {
    case mln::sample_type_id::INT8:
      return details::dtype_of<mln::sample_type_id::INT8>();
    case mln::sample_type_id::INT16:
      return details::dtype_of<mln::sample_type_id::INT16>();
    case mln::sample_type_id::INT32:
      return details::dtype_of<mln::sample_type_id::INT32>();
    case mln::sample_type_id::INT64:
      return details::dtype_of<mln::sample_type_id::INT64>();
    case mln::sample_type_id::UINT8:
      return details::dtype_of<mln::sample_type_id::UINT8>();
    case mln::sample_type_id::UINT16:
      return details::dtype_of<mln::sample_type_id::UINT16>();
    case mln::sample_type_id::UINT32:
      return details::dtype_of<mln::sample_type_id::UINT32>();
    case mln::sample_type_id::UINT64:
      return details::dtype_of<mln::sample_type_id::UINT64>();
    case mln::sample_type_id::FLOAT:
      return details::dtype_of<mln::sample_type_id::FLOAT>();
    case mln::sample_type_id::DOUBLE:
      return details::dtype_of<mln::sample_type_id::DOUBLE>();
    case mln::sample_type_id::BOOL:
      return details::dtype_of<mln::sample_type_id::BOOL>();
    case mln::sample_type_id::RGB8:
      return details::dtype_of<mln::sample_type_id::UINT8>();
    default:
      throw std::runtime_error("Invalid sample_type_id");
    }
    return pybind11::none();
  }

  mln::sample_type_id get_sample_type(const std::string& type_format)
  {
    pybind11::dtype type;
    try
    {
      type = pybind11::dtype(type_format);
    }
    catch (const std::exception&)
    {
      return mln::sample_type_id::OTHER;
    }
    if (type.is(details::dtype_of<mln::sample_type_id::INT8>()))
      return mln::sample_type_id::INT8;
    else if (type.is(details::dtype_of<mln::sample_type_id::INT16>()))
      return mln::sample_type_id::INT16;
    else if (type.is(details::dtype_of<mln::sample_type_id::INT32>()))
      return mln::sample_type_id::INT32;
    else if (type.is(details::dtype_of<mln::sample_type_id::INT64>()))
      return mln::sample_type_id::INT64;
    else if (type.is(details::dtype_of<mln::sample_type_id::UINT8>()))
      return mln::sample_type_id::UINT8;
    else if (type.is(details::dtype_of<mln::sample_type_id::UINT16>()))
      return mln::sample_type_id::UINT16;
    else if (type.is(details::dtype_of<mln::sample_type_id::UINT32>()))
      return mln::sample_type_id::UINT32;
    else if (type.is(details::dtype_of<mln::sample_type_id::UINT64>()))
      return mln::sample_type_id::UINT64;
    else if (type.is(details::dtype_of<mln::sample_type_id::FLOAT>()))
      return mln::sample_type_id::FLOAT;
    else if (type.is(details::dtype_of<mln::sample_type_id::DOUBLE>()))
      return mln::sample_type_id::DOUBLE;
    else if (type.is(details::dtype_of<mln::sample_type_id::BOOL>()))
      return mln::sample_type_id::BOOL;
    return mln::sample_type_id::OTHER;
  }
} // namespace


namespace mln::py
{
  mln::ndbuffer_image from_numpy(pybind11::array arr)
  {
    if (!pybind11::detail::check_flags(arr.ptr(), pybind11::detail::npy_api::NPY_ARRAY_C_CONTIGUOUS_))
      throw std::invalid_argument("Array should be C contiguous");
    auto                base = arr.base();
    const auto          info = arr.request();
    mln::sample_type_id type = get_sample_type(info.format);
    if (type == mln::sample_type_id::OTHER)
      throw std::invalid_argument(fmt::format(
          "Invalid dtype argument (Got dtype format {} expected types: [u]int[8, 16, 32, 64], float, double or bool)",
          info.format));
    const bool is_rgb8 = info.ndim == 3 && info.shape[2] == 3 && type == mln::sample_type_id::UINT8;
    const auto pdim    = info.ndim - (is_rgb8 ? 1 : 0);
    if (pdim > mln::PYLENE_NDBUFFER_DEFAULT_DIM)
      throw std::invalid_argument(
          fmt::format("Invalid number of dimension from numpy array (Got {} but should be less than {})", pdim,
                      mln::PYLENE_NDBUFFER_DEFAULT_DIM));
    int            size[mln::PYLENE_NDBUFFER_DEFAULT_DIM]    = {0};
    std::ptrdiff_t strides[mln::PYLENE_NDBUFFER_DEFAULT_DIM] = {0};
    for (auto d = 0; d < pdim; d++)
    {
      size[d]    = info.shape[pdim - d - 1];
      strides[d] = info.strides[pdim - d - 1];
    }
    const auto sample_type = is_rgb8 ? mln::sample_type_id::RGB8 : type;
    auto       res =
        mln::ndbuffer_image::from_buffer(reinterpret_cast<std::byte*>(info.ptr), sample_type, pdim, size, strides);
    if (base && pybind11::isinstance<mln::internal::ndbuffer_image_data>(base))
      res.__data() = pybind11::cast<std::shared_ptr<mln::internal::ndbuffer_image_data>>(base);
    return res;
  }


  pybind11::object to_numpy(const mln::ndbuffer_image& img)
  {
    const auto&      api   = pybind11::detail::npy_api::get();
    pybind11::object data  = pybind11::none();
    int              flags = pybind11::detail::npy_api::NPY_ARRAY_WRITEABLE_;
    if (img.__data())
    {
      data = pybind11::cast(img.__data());
      assert(data.ref_count() > 0);
    }

    /* For the moment, restrict RGB8 image to 2D image */
    const bool               is_rgb8 = img.pdim() == 2 && img.sample_type() == mln::sample_type_id::RGB8;
    const auto               ndim    = img.pdim() + (is_rgb8 ? 1 : 0);
    std::vector<std::size_t> strides(ndim, 1);
    std::vector<std::size_t> shapes(ndim, 3);
    auto                     descr = get_sample_type(img.sample_type());

    for (auto d = 0; d < img.pdim(); d++)
    {
      strides[d] = img.byte_stride(img.pdim() - d - 1);
      shapes[d]  = img.size(img.pdim() - d - 1);
    }

    auto res = pybind11::reinterpret_steal<pybind11::object>(api.PyArray_NewFromDescr_(
        api.PyArray_Type_, descr.release().ptr(), ndim, reinterpret_cast<Py_intptr_t*>(shapes.data()),
        reinterpret_cast<Py_intptr_t*>(strides.data()), reinterpret_cast<void*>(img.buffer()), flags, nullptr));

    if (!res)
      throw std::runtime_error("Unable to create the numpy array in ndimage -> array");
    if (data)
      // **Steal** a reference to data (https://numpy.org/devdocs/reference/c-api/array.html#c.PyArray_SetBaseObject)
      api.PyArray_SetBaseObject_(res.ptr(), data.release().ptr());
    return res;
  }

  void init_pylena_numpy(pybind11::module& m)
  {
    if (!pybind11::detail::get_global_type_info(typeid(mln::internal::ndbuffer_image_data)))
    {
      pybind11::class_<mln::internal::ndbuffer_image_data, std::shared_ptr<mln::internal::ndbuffer_image_data>>(
          m, "ndbuffer_image_data");
    }
  }

} // namespace mln::py
\end{minted}


\clearpage

\section{Structuring element module}
\label{appendix:static-dynamic-bridge.se}

Here is the code of the structuring element module. This module defines what structuring elements available from Python.

\paragraph{se.hpp}
\label{appendix:static-dynamic-bridge.se.hpp}

\begin{minted}{C++}
#include <pybind11/pybind11.h>


namespace mln::py
{

  void init_module_se(pybind11::module& m);

}
\end{minted}

\paragraph{se.cpp}
\label{appendix:static-dynamic-bridge.se.cpp}

\begin{minted}{C++}
#include "se.hpp"

#include <mln/core/image/ndimage.hpp>
#include <mln/core/se/disc.hpp>
#include <mln/core/se/rect2d.hpp>

#include <pybind11/pybind11.h>


namespace mln::py
{

  void init_module_se(pybind11::module& m)
  {
    pybind11::class_<mln::se::disc_non_decomp>(m, "disc").def(
        pybind11::init([](float radius) { return mln::se::disc_non_decomp{radius}; }));

    pybind11::class_<mln::se::disc>(m, "disc_decomposable").def(pybind11::init([](float radius) {
      return mln::se::disc{radius};
    }));

    pybind11::class_<mln::se::rect2d_non_decomp>(m, "rect2d").def(pybind11::init([](int width, int height) {
      return mln::se::rect2d_non_decomp{width, height};
    }));

    pybind11::class_<mln::se::rect2d>(m, "rect2d_decomposable").def(pybind11::init([](int width, int height) {
      return mln::se::rect2d{width, height};
    }));
  }

} // namespace mln::py
\end{minted}


\clearpage

\section{Mathematical morphology module}
\label{appendix:static-dynamic-bridge.mm}

This module defines the routines that are exposed and available from Python. It is divided in three sub-parts. First is
the \(n \times n\) dispatcher code. Second is the value set mechanic. Third is the code exposing the mathematical morphology
algorithms.

\subsection{The \(\protect n \times n\) dispatcher}
\label{appendix:static-dynamic-bridge.mm.nndispatcher}

\paragraph{visit.hpp}
\label{appendix:static-dynamic-bridge.mm.nndispatcher.visit.hpp}

\begin{minted}{C++}
#include <mln/core/image/ndimage.hpp>

#include <stdexcept>

namespace mln::py
{
  template <template <typename> class F, typename... Args>
  auto visit(mln::sample_type_id tid, Args&&... args)
  {
    switch (tid)
    {
    case (mln::sample_type_id::INT8):
      return F<std::int8_t>{}(std::forward<Args>(args)...);
    case (mln::sample_type_id::INT16):
      return F<std::int16_t>{}(std::forward<Args>(args)...);
    case (mln::sample_type_id::INT32):
      return F<std::int32_t>{}(std::forward<Args>(args)...);
    case (mln::sample_type_id::INT64):
      return F<std::int64_t>{}(std::forward<Args>(args)...);
    case (mln::sample_type_id::UINT8):
      return F<std::uint8_t>{}(std::forward<Args>(args)...);
    case (mln::sample_type_id::UINT16):
      return F<std::uint16_t>{}(std::forward<Args>(args)...);
    case (mln::sample_type_id::UINT32):
      return F<std::uint32_t>{}(std::forward<Args>(args)...);
    case (mln::sample_type_id::UINT64):
      return F<std::uint64_t>{}(std::forward<Args>(args)...);
    case (mln::sample_type_id::FLOAT):
      return F<float>{}(std::forward<Args>(args)...);
    case (mln::sample_type_id::DOUBLE):
      return F<double>{}(std::forward<Args>(args)...);
    /*case (mln::sample_type_id::RGB8):
      return F<mln::rgb8>{}(std::forward<Args>(args)...);*/
    case (mln::sample_type_id::OTHER):
      [[fallthrough]];
    default:
      throw std::runtime_error("Unhandled data type");
    }
  }

  template <template <typename> class F, typename D, typename... Args>
  auto visit_d(mln::sample_type_id tid, D&& d, Args&&... args)
  {
    switch (tid)
    {
    case (mln::sample_type_id::INT8):
      return F<std::int8_t>{}(std::forward<Args>(args)...);
    case (mln::sample_type_id::INT16):
      return F<std::int16_t>{}(std::forward<Args>(args)...);
    case (mln::sample_type_id::INT32):
      return F<std::int32_t>{}(std::forward<Args>(args)...);
    case (mln::sample_type_id::INT64):
      return F<std::int64_t>{}(std::forward<Args>(args)...);
    case (mln::sample_type_id::UINT8):
      return F<std::uint8_t>{}(std::forward<Args>(args)...);
    case (mln::sample_type_id::UINT16):
      return F<std::uint16_t>{}(std::forward<Args>(args)...);
    case (mln::sample_type_id::UINT32):
      return F<std::uint32_t>{}(std::forward<Args>(args)...);
    case (mln::sample_type_id::UINT64):
      return F<std::uint64_t>{}(std::forward<Args>(args)...);
    case (mln::sample_type_id::FLOAT):
      return F<float>{}(std::forward<Args>(args)...);
    case (mln::sample_type_id::DOUBLE):
      return F<double>{}(std::forward<Args>(args)...);
    /*case (mln::sample_type_id::RGB8):
      return F<mln::rgb8>{}(std::forward<Args>(args)...);*/
    case (mln::sample_type_id::OTHER):
      [[fallthrough]];
    default:
      return std::forward<D>(d);
    }
  }
} // namespace mln::py
\end{minted}


\subsection{Value-set mechanics}
\label{appendix:static-dynamic-bridge.mm.vs}

\paragraph{value\_set.hpp}
\label{appendix:static-dynamic-bridge.mm.vs.value_set.hpp}

\begin{minted}{C++}
#include <algorithm>
#include <any>
#include <cassert>
#include <cmath>
#include <limits>
#include <type_traits>
#include <utility>


namespace mln
{

  template <class T = void>
  struct value_set
  {
    template <class U>
    U cast(T v) const
    {
      return static_cast<U>(v);
    }

    T max() const noexcept { return std::numeric_limits<T>::max(); }
    T min() const noexcept { return std::numeric_limits<T>::min(); }
    T sup() const noexcept { return std::numeric_limits<T>::max(); }
    T inf() const noexcept { return std::numeric_limits<T>::min(); }

    T plus(T v) const noexcept { return +v; }
    T minus(T v) const noexcept { return -v; }

    auto max(T l, T r) const noexcept { return std::max(l, r); }
    auto min(T l, T r) const noexcept { return std::min(l, r); }
    auto add(T l, T r) const noexcept { return l + r; }
    auto sub(T l, T r) const noexcept { return l - r; }
    auto mult(T l, T r) const noexcept { return l * r; }
    auto div(T l, T r) const noexcept { return l / r; }
    auto mod(T l, T r) const noexcept { return l % r; }
    auto pow(T l, T r) const noexcept { return std::pow(l, r); }
  };

  template <>
  struct value_set<void>
  {

    template <class U, class T>
    U cast(T&& t) const
    {
      return static_cast<U>(std::forward<T>);
    }

    template <class T>
    T max() const noexcept
    {
      return std::numeric_limits<T>::max();
    }
    template <class T>
    T min() const noexcept
    {
      return std::numeric_limits<T>::min();
    }
    template <class T>
    T sup() const noexcept
    {
      return std::numeric_limits<T>::max();
    }
    template <class T>
    T inf() const noexcept
    {
      return std::numeric_limits<T>::min();
    }

    template <class T>
    T plus(T&& v) const noexcept
    {
      return +std::forward<T>(v);
    }
    template <class T>
    T minus(T&& v) const noexcept
    {
      return -std::forward<T>(v);
    }

    template <class T, class U>
    auto max(T&& l, T&& r) const noexcept
    {
      return std::max(std::forward<T>(l), std::forward<T>(r));
    }
    template <class T, class U>
    auto min(T&& l, T&& r) const noexcept
    {
      return std::min(std::forward<T>(l), std::forward<T>(r));
    }

    template <class T, class U>
    auto add(T&& l, U&& r) const noexcept
    {
      return std::forward<T>(l) + std::forward<T>(r);
    }
    template <class T, class U>
    auto sub(T&& l, U&& r) const noexcept
    {
      return std::forward<T>(l) - std::forward<T>(r);
    }
    template <class T, class U>
    auto mult(T&& l, U&& r) const noexcept
    {
      return std::forward<T>(l) * std::forward<T>(r);
    }
    template <class T, class U>
    auto div(T&& l, U&& r) const noexcept
    {
      return std::forward<T>(l) / std::forward<T>(r);
    }
    template <class T, class U>
    auto mod(T&& l, U&& r) const noexcept
    {
      return std::forward<T>(l) % std::forward<T>(r);
    }
    template <class T, class U>
    auto pow(T&& l, U&& r) const noexcept
    {
      return std::pow(std::forward<T>(l), std::forward<T>(r));
    }
  };

  struct abstract_value_set
  {
    virtual ~abstract_value_set() {}

    /*
    template <class U>
    virtual std::any cast(std::any v) const = 0;
    */

    virtual std::any max() const = 0;
    virtual std::any min() const = 0;
    virtual std::any sup() const = 0;
    virtual std::any inf() const = 0;

    virtual std::any plus(const std::any& v) const  = 0;
    virtual std::any minus(const std::any& v) const = 0;

    virtual std::any max(const std::any& l, const std::any& r) const  = 0;
    virtual std::any min(const std::any& l, const std::any& r) const  = 0;
    virtual std::any add(const std::any& l, const std::any& r) const  = 0;
    virtual std::any sub(const std::any& l, const std::any& r) const  = 0;
    virtual std::any mult(const std::any& l, const std::any& r) const = 0;
    virtual std::any div(const std::any& l, const std::any& r) const  = 0;
    virtual std::any mod(const std::any& l, const std::any& r) const  = 0;
    virtual std::any pow(const std::any& l, const std::any& r) const = 0;
  };

  template <typename T>
  struct concrete_value_set : abstract_value_set
  {
    ~concrete_value_set() override {}

    template <class U>
    std::any cast(std::any v) const /* override */
    {
      return {static_cast<U>(std::any_cast<T>(v))};
    }

    std::any max() const override { return {std::numeric_limits<T>::max()}; };
    std::any min() const override { return {std::numeric_limits<T>::min()}; };
    std::any sup() const override { return {std::numeric_limits<T>::max()}; };
    std::any inf() const override { return {std::numeric_limits<T>::min()}; };

    std::any plus(const std::any& v) const override { return {+std::any_cast<T>(v)}; };
    std::any minus(const std::any& v) const override { return {-std::any_cast<T>(v)}; };

    std::any max(const std::any& l, const std::any& r) const override
    {
      return {std::max(std::any_cast<T>(l), std::any_cast<T>(r))};
    }
    std::any min(const std::any& l, const std::any& r) const override
    {
      return {std::min(std::any_cast<T>(l), std::any_cast<T>(r))};
    }
    std::any add(const std::any& l, const std::any& r) const override
    {
      return {std::any_cast<T>(l) + std::any_cast<T>(r)};
    }
    std::any sub(const std::any& l, const std::any& r) const override
    {
      return {std::any_cast<T>(l) - std::any_cast<T>(r)};
    }
    std::any mult(const std::any& l, const std::any& r) const override
    {
      return {std::any_cast<T>(l) * std::any_cast<T>(r)};
    }
    std::any div(const std::any& l, const std::any& r) const override
    {
      return {std::any_cast<T>(l) / std::any_cast<T>(r)};
    }
    std::any mod(const std::any& l, const std::any& r) const override
    {
      return {std::any_cast<T>(l) % std::any_cast<T>(r)};
    }
    std::any pow(const std::any& l, const std::any& r) const override
    {
      return {std::pow(std::any_cast<T>(l), std::any_cast<T>(r))};
    }
  };

  struct type_erased_value
  {
    type_erased_value(const std::any& v, const abstract_value_set& vs);

    const std::any&       val() const;
    const std::type_info& tid() const;

    template <typename From, typename To>
    type_erased_value cast(abstract_value_set* new_vs = nullptr) const
    {
      assert(typeid(From) == tid());
      auto new_val = std::any{static_cast<To>(std::any_cast<From>(v_))};
      return {new_val, new_vs != nullptr ? *new_vs : *vs_};
    }

    type_erased_value max() const;
    type_erased_value min() const;
    type_erased_value sup() const;
    type_erased_value inf() const;

    type_erased_value plus() const;
    type_erased_value minus() const;

    type_erased_value max(const type_erased_value& rhs) const;
    type_erased_value min(const type_erased_value& rhs) const;
    type_erased_value add(const type_erased_value& rhs) const;
    type_erased_value sub(const type_erased_value& rhs) const;
    type_erased_value mult(const type_erased_value& rhs) const;
    type_erased_value div(const type_erased_value& rhs) const;
    type_erased_value mod(const type_erased_value& rhs) const;
    type_erased_value pow(const type_erased_value& rhs) const;

  private:
    std::any                  v_;
    const abstract_value_set* vs_;
  };

  template <>
  struct value_set<type_erased_value> : abstract_value_set
  {
    value_set(const type_erased_value& v);
    ~value_set() override;

    template <typename From, typename To>
    std::any cast(const std::any& v, abstract_value_set* new_vs = nullptr) const
    {
      auto absv = std::any_cast<type_erased_value>(v);
      return {absv.template cast<From, To>(new_vs)};
    }

    std::any max() const override;
    std::any min() const override;
    std::any sup() const override;
    std::any inf() const override;

    std::any plus(const std::any& v) const override;

    std::any minus(const std::any& v) const override;

    std::any max(const std::any& l, const std::any& r) const override;
    std::any min(const std::any& l, const std::any& r) const override;
    std::any add(const std::any& l, const std::any& r) const override;
    std::any sub(const std::any& l, const std::any& r) const override;
    std::any mult(const std::any& l, const std::any& r) const override;
    std::any div(const std::any& l, const std::any& r) const override;
    std::any mod(const std::any& l, const std::any& r) const override;
    std::any pow(const std::any& l, const std::any& r) const override;

  private:
    const type_erased_value* v_;
  };
} // namespace mln
\end{minted}

\paragraph{value\_set.cpp}
\label{appendix:static-dynamic-bridge.mm.vs.value_set.cpp}

\begin{minted}{C++}
#include <mln/core/value/value_set.hpp>

#include <algorithm>
#include <any>
#include <cinttypes>
#include <cmath>
#include <limits>
#include <optional>
#include <type_traits>
#include <typeinfo>
#include <variant>


namespace details
{


  template <class... Args>
  auto any_to_variant_cast(std::any a) -> std::optional<std::variant<Args...>>
  {
    if (!a.has_value())
      return std::nullopt;

    std::optional<std::variant<Args...>> v = std::nullopt;

    bool found = ((a.type() == typeid(Args) && (v = std::any_cast<Args>(std::move(a)), true)) || ...);

    if (!found)
      return std::nullopt;

    return {std::move(*v)};
  }

  template <template <typename> class Op, typename... Args>
  auto call_template_op_from_type_info(const std::type_info& tid) -> std::optional<std::variant<Op<Args>...>>
  {
    std::optional<std::variant<Op<Args>...>> v = std::nullopt;

    bool found = ((tid == typeid(Args) && (v = Op<Args>{}, true)) || ...);

    if (!found)
      return std::nullopt;

    return {std::move(*v)};
  }

  template <class... Ts>
  struct type_list
  {
  };

  template <class... Ts>
  auto any_to_variant_cast_type_list(std::any a, type_list<Ts...>) -> std::optional<std::variant<Ts...>>
  {
    return any_to_variant_cast<Ts...>(a);
  }

  template <template <typename> class Op, typename... Ts>
  auto call_template_op_from_type_info_type_list(const std::type_info& tid, type_list<Ts...>)
      -> std::optional<std::variant<Op<Ts>...>>
  {
    return call_template_op_from_type_info<Op, Ts...>(tid);
  }

  using supported_types = type_list<std::uint8_t, std::uint16_t, std::uint32_t, std::uint64_t, std::int8_t,
                                    std::int16_t, std::int32_t, std::int64_t, float, double>;

  template <class T, class U, class C = std::common_type_t<T, U>>
  C max(const T& a, const U& b)
  {
    return std::max(static_cast<C>(a), static_cast<C>(b));
  }

  template <class T, class U, class C = std::common_type_t<T, U>>
  C min(const T& a, const U& b)
  {
    return std::min(static_cast<C>(a), static_cast<C>(b));
  }

  template <class T, class U>
  long double pow(const T& a, const U& b)
  {
    return std::pow(static_cast<long double>(a), static_cast<long double>(b));
  }

} // namespace details


namespace mln
{

  type_erased_value::type_erased_value(const std::any& v, const abstract_value_set& vs)
    : v_(v)
    , vs_(&vs)
  {
  }

  const std::any&       type_erased_value::val() const { return v_; }
  const std::type_info& type_erased_value::tid() const { return v_.type(); }

  template <typename T>
  struct limit_max
  {
    auto operator()() { return std::numeric_limits<T>::max(); }
  };

  type_erased_value type_erased_value::max() const
  {
    auto op = details::call_template_op_from_type_info_type_list<limit_max>(tid(), details::supported_types{});

    auto ret = std::visit([](auto op_) -> std::any { return {op_()}; }, *op);
    return {ret, *vs_};
  }

  template <typename T>
  struct limit_min
  {
    auto operator()() { return std::numeric_limits<T>::min(); }
  };

  type_erased_value type_erased_value::min() const
  {
    auto op = details::call_template_op_from_type_info_type_list<limit_min>(tid(), details::supported_types{});

    auto ret = std::visit([](auto op_) -> std::any { return {op_()}; }, *op);
    return {ret, *vs_};
  }

  type_erased_value type_erased_value::sup() const
  {
    auto op = details::call_template_op_from_type_info_type_list<limit_max>(tid(), details::supported_types{});

    auto ret = std::visit([](auto op_) -> std::any { return {op_()}; }, *op);
    return {ret, *vs_};
  }

  type_erased_value type_erased_value::inf() const
  {
    auto op = details::call_template_op_from_type_info_type_list<limit_min>(tid(), details::supported_types{});

    auto ret = std::visit([](auto op_) -> std::any { return {op_()}; }, *op);
    return {ret, *vs_};
  }

  type_erased_value type_erased_value::plus() const { return {vs_->plus(v_), *vs_}; }
  type_erased_value type_erased_value::minus() const { return {vs_->plus(v_), *vs_}; }

  type_erased_value type_erased_value::max(const type_erased_value& rhs) const
  {
    return {vs_->max(v_, rhs.val()), *vs_};
  }
  type_erased_value type_erased_value::min(const type_erased_value& rhs) const
  {
    return {vs_->min(v_, rhs.val()), *vs_};
  }
  type_erased_value type_erased_value::add(const type_erased_value& rhs) const
  {
    return {vs_->add(v_, rhs.val()), *vs_};
  }
  type_erased_value type_erased_value::sub(const type_erased_value& rhs) const
  {
    return {vs_->sub(v_, rhs.val()), *vs_};
  }
  type_erased_value type_erased_value::mult(const type_erased_value& rhs) const
  {
    return {vs_->mult(v_, rhs.val()), *vs_};
  }
  type_erased_value type_erased_value::div(const type_erased_value& rhs) const
  {
    return {vs_->div(v_, rhs.val()), *vs_};
  }
  type_erased_value type_erased_value::mod(const type_erased_value& rhs) const
  {
    return {vs_->mod(v_, rhs.val()), *vs_};
  }
  type_erased_value type_erased_value::pow(const type_erased_value& rhs) const
  {
    return {vs_->pow(v_, rhs.val()), *vs_};
  }


  value_set<type_erased_value>::value_set(const type_erased_value& v)
    : v_(&v)
  {
  }

  value_set<type_erased_value>::~value_set() {}

  std::any value_set<type_erased_value>::max() const { return {v_->max()}; }
  std::any value_set<type_erased_value>::min() const { return {v_->min()}; }
  std::any value_set<type_erased_value>::sup() const { return {v_->sup()}; }
  std::any value_set<type_erased_value>::inf() const { return {v_->inf()}; }

  std::any value_set<type_erased_value>::plus(const std::any& v) const
  {

    auto va = details::any_to_variant_cast_type_list(v, details::supported_types{});
    if (va)
      return {std::visit([](auto _v) -> std::any { return {+_v}; }, *va)};
    else
    {
      auto absv = std::any_cast<type_erased_value>(v);
      return {absv.plus()};
    }
  }

  std::any value_set<type_erased_value>::minus(const std::any& v) const
  {
    auto va = details::any_to_variant_cast_type_list(v, details::supported_types{});
    if (va)
      return {std::visit([](auto _v) -> std::any { return {-_v}; }, *va)};
    else
    {
      auto absv = std::any_cast<type_erased_value>(v);
      return {absv.minus()};
    }
  }

  std::any value_set<type_erased_value>::max(const std::any& l, const std::any& r) const
  {
    auto lhs = details::any_to_variant_cast_type_list(l, details::supported_types{});
    auto rhs = details::any_to_variant_cast_type_list(r, details::supported_types{});

    if (lhs && rhs)
      return {std::visit([](auto lhs_, auto rhs_) -> std::any { return {details::max(lhs_, rhs_)}; }, *lhs, *rhs)};
    else
    {
      auto absl = std::any_cast<type_erased_value>(l);
      auto absr = std::any_cast<type_erased_value>(r);
      return {absl.max(absr)};
    }
  }

  std::any value_set<type_erased_value>::min(const std::any& l, const std::any& r) const
  {
    auto lhs = details::any_to_variant_cast_type_list(l, details::supported_types{});
    auto rhs = details::any_to_variant_cast_type_list(r, details::supported_types{});

    if (lhs && rhs)
      return {std::visit([](auto lhs_, auto rhs_) -> std::any { return {details::min(lhs_, rhs_)}; }, *lhs, *rhs)};
    else
    {
      auto absl = std::any_cast<type_erased_value>(l);
      auto absr = std::any_cast<type_erased_value>(r);
      return {absl.min(absr)};
    }
  }

  std::any value_set<type_erased_value>::add(const std::any& l, const std::any& r) const
  {
    auto lhs = details::any_to_variant_cast_type_list(l, details::supported_types{});
    auto rhs = details::any_to_variant_cast_type_list(r, details::supported_types{});

    if (lhs && rhs)
      return {std::visit([](auto lhs_, auto rhs_) -> std::any { return {lhs_ + rhs_}; }, *lhs, *rhs)};
    else
    {
      auto absl = std::any_cast<type_erased_value>(l);
      auto absr = std::any_cast<type_erased_value>(r);
      return {absl.add(absr)};
    }
  }

  std::any value_set<type_erased_value>::sub(const std::any& l, const std::any& r) const
  {
    auto lhs = details::any_to_variant_cast_type_list(l, details::supported_types{});
    auto rhs = details::any_to_variant_cast_type_list(r, details::supported_types{});

    if (lhs && rhs)
      return {std::visit([](auto lhs_, auto rhs_) -> std::any { return {lhs_ - rhs_}; }, *lhs, *rhs)};
    else
    {
      auto absl = std::any_cast<type_erased_value>(l);
      auto absr = std::any_cast<type_erased_value>(r);
      return {absl.sub(absr)};
    }
  }

  std::any value_set<type_erased_value>::mult(const std::any& l, const std::any& r) const
  {
    auto lhs = details::any_to_variant_cast_type_list(l, details::supported_types{});
    auto rhs = details::any_to_variant_cast_type_list(r, details::supported_types{});

    if (lhs && rhs)
      return {std::visit([](auto lhs_, auto rhs_) -> std::any { return {lhs_ * rhs_}; }, *lhs, *rhs)};
    else
    {
      auto absl = std::any_cast<type_erased_value>(l);
      auto absr = std::any_cast<type_erased_value>(r);
      return {absl.mult(absr)};
    }
  }

  std::any value_set<type_erased_value>::div(const std::any& l, const std::any& r) const
  {
    auto lhs = details::any_to_variant_cast_type_list(l, details::supported_types{});
    auto rhs = details::any_to_variant_cast_type_list(r, details::supported_types{});

    if (lhs && rhs)
      return {std::visit([](auto lhs_, auto rhs_) -> std::any { return {lhs_ / rhs_}; }, *lhs, *rhs)};
    else
    {
      auto absl = std::any_cast<type_erased_value>(l);
      auto absr = std::any_cast<type_erased_value>(r);
      return {absl.div(absr)};
    }
  }

  std::any value_set<type_erased_value>::mod(const std::any& l, const std::any& r) const
  {
    auto lhs = details::any_to_variant_cast_type_list(l, details::supported_types{});
    auto rhs = details::any_to_variant_cast_type_list(r, details::supported_types{});

    if (lhs && rhs)
      return {std::visit([](auto lhs_, auto rhs_) -> std::any { return {lhs_ % rhs_}; }, *lhs, *rhs)};
    else
    {
      auto absl = std::any_cast<type_erased_value>(l);
      auto absr = std::any_cast<type_erased_value>(r);
      return {absl.mod(absr)};
    }
  }

  std::any value_set<type_erased_value>::pow(const std::any& l, const std::any& r) const
  {
    auto lhs = details::any_to_variant_cast_type_list(l, details::supported_types{});
    auto rhs = details::any_to_variant_cast_type_list(r, details::supported_types{});

    if (lhs && rhs)
      return {std::visit([](auto lhs_, auto rhs_) -> std::any { return {details::pow(lhs_, rhs_)}; }, *lhs, *rhs)};
    else
    {
      auto absl = std::any_cast<type_erased_value>(l);
      auto absr = std::any_cast<type_erased_value>(r);
      return {absl.pow(absr)};
    }
  }

} // namespace mln
\end{minted}

\paragraph{py\_value\_set.hpp}
\label{appendix:static-dynamic-bridge.mm.vs.py_value_set.hpp}

\begin{minted}{C++}
#include <mln/core/value/value_set.hpp>
#include <pybind11/pybind11.h>

#include <any>
#include <cinttypes>
#include <memory>
#include <string>
#include <string_view>
#include <typeinfo>
#include <variant>


namespace mln
{
  template <>
  struct value_set<pybind11::object> : abstract_value_set
  {
    value_set(pybind11::object python_vs_instance);

    ~value_set() override;

    template <typename U>
    std::any cast(const std::any& v) const;

    std::any max() const override;
    std::any min() const override;
    std::any sup() const override;
    std::any inf() const override;

    std::any plus(const std::any& v) const override;
    std::any minus(const std::any& v) const override;

    std::any max(const std::any& l, const std::any& r) const override;
    std::any min(const std::any& l, const std::any& r) const override;
    std::any add(const std::any& l, const std::any& r) const override;
    std::any sub(const std::any& l, const std::any& r) const override;
    std::any mult(const std::any& l, const std::any& r) const override;
    std::any div(const std::any& l, const std::any& r) const override;
    std::any mod(const std::any& l, const std::any& r) const override;
    std::any pow(const std::any& l, const std::any& r) const override;

  private:
    pybind11::object vs_instance_;
  };

} // namespace mln


namespace details
{

  template <class U>
  auto get_python_type()
  {
    // C++ type -> Python type
    static std::unordered_map<std::type_index, std::string> type_names{
        {std::type_index(typeid(bool{})), "bool"},         //
        {std::type_index(typeid(int8_t{})), "int"},        //
        {std::type_index(typeid(int16_t{})), "int"},       //
        {std::type_index(typeid(int32_t{})), "int"},       //
        {std::type_index(typeid(int64_t{})), "int"},       //
        {std::type_index(typeid(uint8_t{})), "int"},       //
        {std::type_index(typeid(uint16_t{})), "int"},      //
        {std::type_index(typeid(uint32_t{})), "int"},      //
        {std::type_index(typeid(uint64_t{})), "int"},      //
        {std::type_index(typeid(float{})), "float"},       //
        {std::type_index(typeid(double{})), "float"},      //
        {std::type_index(typeid((char*){})), "str"},       //
        {std::type_index(typeid((const char*){})), "str"}, //
        {std::type_index(typeid(std::string{})), "str"}};
    return type_names[std::type_index(typeid(U{}))];
  }

} // namespace details


namespace mln
{
  template <typename U>
  std::any value_set<pybind11::object>::cast(const std::any& v) const
  {
    pybind11::object v_ = std::any_cast<pybind11::object>(v);
    // if constexpr (std::is_arithmetic_v<U>) // bool, signed/unsigned int, floating point
    if constexpr (std::is_same_v<U, bool>) // bool
    {
      pybind11::bool_ pyv = vs_instance_.attr("cast")(v_, ::details::get_python_type<U>());
      return {static_cast<U>(pyv)};
    }
    else if constexpr (std::is_integral_v<U>) // signed/unsigned int
    {
      pybind11::int_ pyv = vs_instance_.attr("cast")(v_, ::details::get_python_type<U>());
      return {static_cast<U>(pyv)};
    }
    else if constexpr (std::is_floating_point_v<U>) // floating point
    {
      pybind11::float_ pyv = vs_instance_.attr("cast")(v_, ::details::get_python_type<U>());
      return {static_cast<U>(pyv)};
    }
    else
    {
      pybind11::object ret = vs_instance_.attr("cast")(v, ::details::get_python_type<U>());
      return {*(ret.cast<U*>())};
    }
  }

} // namespace mln
\end{minted}

\paragraph{py\_value\_set.cpp}
\label{appendix:static-dynamic-bridge.mm.vs.py_value_set.cpp}

\begin{minted}{C++}
#include "py_value_set.hpp"

#include <any>

namespace mln
{

  value_set<pybind11::object>::value_set(pybind11::object python_vs_instance)
    : vs_instance_(python_vs_instance)
  {
  }

  value_set<pybind11::object>::~value_set()
  {
  }

  std::any value_set<pybind11::object>::max() const { return {vs_instance_.attr("max")()}; }

  std::any value_set<pybind11::object>::min() const { return {vs_instance_.attr("min")()}; }

  std::any value_set<pybind11::object>::sup() const { return {vs_instance_.attr("sup")()}; }

  std::any value_set<pybind11::object>::inf() const { return {vs_instance_.attr("inf")()}; }


  std::any value_set<pybind11::object>::plus(const std::any& v) const
  {
    auto pyv = std::any_cast<pybind11::object>(v);
    return {vs_instance_.attr("plus")(pyv)};
  }

  std::any value_set<pybind11::object>::minus(const std::any& v) const
  {
    auto pyv = std::any_cast<pybind11::object>(v);
    return {vs_instance_.attr("minus")(pyv)};
  }


  std::any value_set<pybind11::object>::max(const std::any& l, const std::any& r) const
  {
    auto pyl = std::any_cast<pybind11::object>(l);
    auto pyr = std::any_cast<pybind11::object>(r);
    return {vs_instance_.attr("max")(pyl, pyr)};
  }

  std::any value_set<pybind11::object>::min(const std::any& l, const std::any& r) const
  {
    auto pyl = std::any_cast<pybind11::object>(l);
    auto pyr = std::any_cast<pybind11::object>(r);
    return {vs_instance_.attr("min")(pyl, pyr)};
  }

  std::any value_set<pybind11::object>::add(const std::any& l, const std::any& r) const
  {
    auto pyl = std::any_cast<pybind11::object>(l);
    auto pyr = std::any_cast<pybind11::object>(r);
    return {vs_instance_.attr("add")(pyl, pyr)};
  }

  std::any value_set<pybind11::object>::sub(const std::any& l, const std::any& r) const
  {
    auto pyl = std::any_cast<pybind11::object>(l);
    auto pyr = std::any_cast<pybind11::object>(r);
    return {vs_instance_.attr("sub")(pyl, pyr)};
  }

  std::any value_set<pybind11::object>::mult(const std::any& l, const std::any& r) const
  {
    auto pyl = std::any_cast<pybind11::object>(l);
    auto pyr = std::any_cast<pybind11::object>(r);
    return {vs_instance_.attr("mult")(pyl, pyr)};
  }

  std::any value_set<pybind11::object>::div(const std::any& l, const std::any& r) const
  {
    auto pyl = std::any_cast<pybind11::object>(l);
    auto pyr = std::any_cast<pybind11::object>(r);
    return {vs_instance_.attr("div")(pyl, pyr)};
  }

  std::any value_set<pybind11::object>::mod(const std::any& l, const std::any& r) const
  {
    auto pyl = std::any_cast<pybind11::object>(l);
    auto pyr = std::any_cast<pybind11::object>(r);
    return {vs_instance_.attr("mod")(pyl, pyr)};
  }

  std::any value_set<pybind11::object>::pow(const std::any& l, const std::any& r) const
  {
    auto pyl = std::any_cast<pybind11::object>(l);
    auto pyr = std::any_cast<pybind11::object>(r);
    return {vs_instance_.attr("pow")(pyl, pyr)};
  }

} // namespace mln
\end{minted}


\subsection{Mathematical morphology routines}
\label{appendix:static-dynamic-bridge.mm.algos}

\paragraph{morpho.hpp}
\label{appendix:static-dynamic-bridge.mm.mropho.hpp}

\begin{minted}{C++}
#include <pybind11/pybind11.h>


namespace mln::py
{

  void init_module_morpho(pybind11::module& m);

}
\end{minted}

\paragraph{morpho.cpp}
\label{appendix:static-dynamic-bridge.mm.mropho.cpp}

\begin{minted}{C++}
#include "morpho.hpp"

#include "ndimage.hpp"
#include "py_value_set.hpp"
#include "visit.hpp"

#include <mln/core/algorithm/transform.hpp>
#include <mln/core/image/ndimage.hpp>
#include <mln/core/se/disc.hpp>
#include <mln/core/se/rect2d.hpp>
#include <mln/morpho/experimental/dilation.hpp>

#include <fmt/core.h>

#include <pybind11/stl.h>

#include <algorithm>
#include <string_view>
#include <variant>


namespace my
{
  template <class T>
  mln::image2d<float> slow_stretch(const mln::image2d<T>& src, const mln::value_set<pybind11::object>& py_vs)
  {
    mln::image2d<float> res = mln::transform(src, [&py_vs](auto val) -> float {
      auto anymax  = py_vs.max();                        // returns std::any of pybind11:object
      auto anyval  = std::any{pybind11::cast(val)};      // converts to std::any of pybind11::object
      auto anyret  = py_vs.div(anyval, anymax);          // returns std::any of pybind11::object
      auto anyfret = py_vs.template cast<float>(anyret); // returns std::any of float
      return std::any_cast<float>(anyfret);              // returns float
    });
    return res;
  }

  template <class T>
  mln::image2d<float> virtual_dispatch_stretch(const mln::image2d<T>& src)
  {
    auto                vs   = mln::concrete_value_set<T>{};     // value-set for T
    auto                vs_f = mln::concrete_value_set<float>{}; // value-set for float
    mln::image2d<float> res  = mln::transform(src, [&vs, &vs_f](auto val) -> float {
      auto anymax  = vs.max();                                 // returns std::any
      auto fanyval = vs.template cast<float>(val);             // cast to float in std::any
      auto fanymax = vs.template cast<float>(anymax);          // cast to float in std::any
      return std::any_cast<float>(vs_f.div(fanyval, fanymax)); // div returns float
    });
    return res;
  }

  template <class T>
  mln::image2d<float> stretch_virtual_dispatch_type_erased_value(const mln::image2d<T>& src)
  {
    auto                vs   = mln::concrete_value_set<T>{};     // value-set for T
    auto                vs_f = mln::concrete_value_set<float>{}; // value-set for float
    mln::image2d<float> res  = mln::transform(src, [&vs, &vs_f](auto val) -> float {
      // simulate having an image<type_erased_value>
      auto anyval = std::any{val}; // std::any of T
                                    // type_erased_value of std::any of T aware of value-set of T
      auto abs_anyval = mln::type_erased_value{anyval, vs};
      // instantiate a value-set for type_erased_value
      auto abs_vs = mln::value_set<mln::type_erased_value>{abs_anyval};
      auto anyabs_anymax =
          abs_vs.max(); // returns std::any of type_erased_value
                        // cast underlying std::any of type_erased_value of std::any of T into
                        // std::any of type_erased_value of std::any of float aware of value-set for float
      auto anyabs_fanyval = abs_vs.template cast<T, float>(std::any{abs_anyval}, &vs_f);
      auto anyabs_fanymax = abs_vs.template cast<T, float>(anyabs_anymax, &vs_f);
      // dispatch on known type, find a type_erased_value, then call
      // anyabs_fanyval.div(anyabs_fanymax) to perform division which will call
      // the underlying value-set for float for this operation
      auto anyabs_fanyret = abs_vs.div(anyabs_fanyval, anyabs_fanymax);
      // convert result back into float for returning to the image
      auto anyfret = std::any_cast<mln::type_erased_value>(anyabs_fanyret).val();
      return std::any_cast<float>(anyfret);
    });
    return res;
  }

  template <class T>
  mln::image2d<float> fast_stretch(const mln::image2d<T>& src)
  {
    auto                vs   = src.get_value_set();     // value-set for T
    auto                vs_f = mln::value_set<float>{}; // fast value-set for float
    mln::image2d<float> res  = mln::transform(src, [&vs, &vs_f](auto val) -> float {
      auto max  = vs.max();                     // returns T
      auto fval = vs.template cast<float>(val); // returns float
      auto fmax = vs.template cast<float>(max); // returns float
      return vs_f.div(fval, fmax);              // div directly returns float
    });
    return res;
  }
} // namespace my

namespace
{

  template <typename T>
  struct stretch_operator_t
  {
    template <typename Img>
    mln::ndbuffer_image operator()(Img&& img, const std::optional<pybind11::object>& py_vs) const
    {
      if (auto* image_ptr = std::forward<Img>(img).template cast_to<T, 2>(); image_ptr)
      {
        if (py_vs.has_value())
        {
          auto vs = mln::value_set<pybind11::object>{py_vs.value()};
          return my::slow_stretch(*image_ptr, vs);
        }
        else
        {
          return my::fast_stretch(*image_ptr);
        }
      }
      else
      {
        std::runtime_error("Unable to convert the image to the required type.");
        return {};
      }
    }
  };

  mln::ndbuffer_image stretch(mln::ndbuffer_image input, std::optional<pybind11::object> py_vs)
  {
    if (input.pdim() == 2)
    {
      return mln::py::visit<stretch_operator_t>(input.sample_type(), input, py_vs);
    }
    else
    {
      std::runtime_error("Unsupported dimension.");
      return {};
    }
  }

  template <typename T>
  struct stretch_virtual_dispatch_operator_t
  {
    template <typename Img>
    mln::ndbuffer_image operator()(Img&& img) const
    {
      if (auto* image_ptr = std::forward<Img>(img).template cast_to<T, 2>(); image_ptr)
      {
        return my::virtual_dispatch_stretch(*image_ptr);
      }
      else
      {
        std::runtime_error("Unable to convert the image to the required type.");
        return {};
      }
    }
  };

  mln::ndbuffer_image stretch_virtual_dispatch(mln::ndbuffer_image input)
  {
    if (input.pdim() == 2)
    {
      return mln::py::visit<stretch_virtual_dispatch_operator_t>(input.sample_type(), input);
    }
    else
    {
      std::runtime_error("Unsupported dimension.");
      return {};
    }
  }

  template <typename T>
  struct stretch_virtual_dispatch_type_erased_value_operator_t
  {
    template <typename Img>
    mln::ndbuffer_image operator()(Img&& img) const
    {
      if (auto* image_ptr = std::forward<Img>(img).template cast_to<T, 2>(); image_ptr)
      {
        return my::stretch_virtual_dispatch_type_erased_value(*image_ptr);
      }
      else
      {
        std::runtime_error("Unable to convert the image to the required type.");
        return {};
      }
    }
  };

  mln::ndbuffer_image stretch_virtual_dispatch_type_erased_value(mln::ndbuffer_image input)
  {
    if (input.pdim() == 2)
    {
      return mln::py::visit<stretch_virtual_dispatch_type_erased_value_operator_t>(input.sample_type(), input);
    }
    else
    {
      std::runtime_error("Unsupported dimension.");
      return {};
    }
  }

  using se_t = std::variant<mln::se::disc, mln::se::disc_non_decomp, mln::se::rect2d, mln::se::rect2d_non_decomp>;

  template <typename T>
  struct dilate2d_operator_t
  {
    template <typename Img, typename SE>
    mln::ndbuffer_image operator()(Img&& img, SE&& se) const
    {
      if (auto* image_ptr = std::forward<Img>(img).template cast_to<T, 2>(); image_ptr)
      {
        return mln::morpho::experimental::dilation(*image_ptr, std::forward<SE>(se));
      }
      else
      {
        std::runtime_error("Unable to convert the image to the required type.");
        return {};
      }
    }
  };

  mln::ndbuffer_image dilate(mln::ndbuffer_image input, const se_t& se)
  {
    if (input.pdim() == 2)
    {
      return std::visit(
          [&input](const auto& se_) { return mln::py::visit<dilate2d_operator_t>(input.sample_type(), input, se_); },
          se);
    }
    else
    {
      std::runtime_error("Unsupported dimension.");
      return {};
    }
  }

} // namespace


namespace mln::py
{

  using namespace pybind11::literals;

  void init_module_morpho(pybind11::module& m)
  {
    m.def("dilate", dilate,
          "Perform a morphological dilation.\n"
          "\n"
          "structuring element must be valid.",
          "Input"_a, "se"_a)

      .def("stretch", stretch,
            "Perform a morphological stretch.\n"
            "Input"_a,
            "vs"_a = pybind11::none())

      .def("stretch_virtual_dispatch", stretch_virtual_dispatch,
            "Perform a morphological stretch.\n"
            "Input"_a)

      .def("stretch_virtual_dispatch_type_erased_value", stretch_virtual_dispatch_type_erased_value,
            "Perform a morphological stretch.\n"
            "Input"_a);
  }

} // namespace mln::py
\end{minted}


\clearpage

\section{Exposed Pylena main module}
\label{appendix:static-dynamic-bridge.pylena}

\paragraph{pylena.cpp}
\label{appendix:static-dynamic-bridge.pylena.cpp}

\begin{minted}{C++}
#include "morpho.hpp"
#include "ndimage.hpp"
#include "se.hpp"

#include <pybind11/pybind11.h>


PYBIND11_MODULE(pylena, m)
{
  mln::py::init_pylena_numpy(m);

  auto mmorpho = m.def_submodule("morpho", "Mathematical morphology module.");
  mln::py::init_module_morpho(mmorpho);

  auto mse = m.def_submodule("se", "Structuring elements module.");
  mln::py::init_module_se(mse);
}
\end{minted}


\clearpage

\section{Python value set module}
\label{appendix:static-dynamic-bridge.python.vs}

The module has the following directory structure:
\begin{minted}{bash}
  ValueSetExample/
  | __init__.py
  | BaseValueSet.py
  | ExampleValueSet.py
\end{minted}

\paragraph{BaseValueSet.py}
\label{appendix:static-dynamic-bridge.python.vs.BaseValueSet.py}

\begin{minted}{python}
from abc import ABC, abstractmethod
from typing import Any


class AbstractValueSet(ABC):

    def __init__(self): pass

    @abstractmethod
    def cast(self, value: Any, type_):
        if type_ in ["int", "float", "bool", "str"]:
            import importlib
            module = importlib.import_module('builtins')
            cls = getattr(module, type_)
            return cls(value)
        raise ValueError()

    @abstractmethod
    def max(self):
        import math
        return math.inf

    @abstractmethod
    def min(self):
        import math
        return -math.inf

    @abstractmethod
    def sup(self):
        import math
        return math.inf

    @abstractmethod
    def inf(self):
        import math
        return -math.inf

    @abstractmethod
    def plus(self, value: Any) -> Any:
        return + value

    @abstractmethod
    def minus(self, value: Any) -> Any:
        return - value

    @abstractmethod
    def add(self, lhs: Any, rhs: Any) -> Any:
        return lhs + rhs

    @abstractmethod
    def sub(self, lhs: Any, rhs: Any) -> Any:
        return lhs - rhs

    @abstractmethod
    def mult(self, lhs: Any, rhs: Any) -> Any:
        return lhs * rhs

    @abstractmethod
    def div(self, lhs: Any, rhs: Any) -> Any:
        return lhs / rhs

    @abstractmethod
    def mod(self, lhs: Any, rhs: Any) -> Any:
        return lhs % rhs

    @abstractmethod
    def pow(self, lhs: Any, rhs: Any) -> Any:
        return lhs ** rhs
\end{minted}

\paragraph{ExampleStruct.py}
\label{appendix:static-dynamic-bridge.python.vs.ExampleStruct.py}

\begin{minted}{python}
from typing import Any


class MyStruct:
    v_: Any

    def __init__(self, v: Any):
        self.v_ = v

    def getV(self) -> Any:
        return self.v_

    def setV(self, v: Any):
        self.v_ = v
\end{minted}

\paragraph{ExampleValueSet.py}
\label{appendix:static-dynamic-bridge.python.vs.ExampleValueSet.py}

\begin{minted}{python}
from ValueSetExample.ExampleStruct import MyStruct
from ValueSetExample.BaseValueSet import AbstractValueSet

from typing import Any


class MyValueSet(AbstractValueSet):

    def __init__(self): pass

    def get_MyStruct__(self, v: Any):
        return v.getV() if isinstance(v, MyStruct) else v

    def cast(self, value: Any, type_):
        v = self.get_MyStruct__(value)
        return super().cast(v, type_)

    def max(self):
        return 255

    def min(self):
        return 0

    def sup(self):
        return 255

    def inf(self):
        return 0

    def plus(self, value: Any) -> Any:
        v = self.get_MyStruct__(value)
        return MyStruct(super().plus(v))

    def minus(self, value: Any) -> Any:
        v = self.get_MyStruct__(value)
        return MyStruct(super().minus(v))

    def add(self, lhs: Any, rhs: Any) -> Any:
        l = self.get_MyStruct__(lhs)
        r = self.get_MyStruct__(rhs)
        return MyStruct(super().add(l, r))

    def sub(self, lhs: Any, rhs: Any) -> Any:
        l = self.get_MyStruct__(lhs)
        r = self.get_MyStruct__(rhs)
        return MyStruct(super().sub(l, r))

    def mult(self, lhs: Any, rhs: Any) -> Any:
        l = self.get_MyStruct__(lhs)
        r = self.get_MyStruct__(rhs)
        return MyStruct(super().mult(l, r))

    def div(self, lhs: Any, rhs: Any) -> Any:
        l = self.get_MyStruct__(lhs)
        r = self.get_MyStruct__(rhs)
        return MyStruct(super().div(l, r))

    def mod(self, lhs: Any, rhs: Any) -> Any:
        l = self.get_MyStruct__(lhs)
        r = self.get_MyStruct__(rhs)
        return MyStruct(super().mod(l, r))

    def pow(self, lhs: Any, rhs: Any) -> Any:
        l = self.get_MyStruct__(lhs)
        r = self.get_MyStruct__(rhs)
        return MyStruct(super().pow(l, r))
\end{minted}

\paragraph{\_\_init\_\_.py}
\label{appendix:static-dynamic-bridge.python.vs.__init__.py}

\begin{minted}{python}
from ValueSetExample.BaseValueSet import AbstractValueSet
from ValueSetExample.ExampleStruct import MyStruct
from ValueSetExample.ExampleValueSet import MyValueSet
\end{minted}


\clearpage

\section{Benchmark source code}
\label{appendix:static-dynamic-bridge.benchmark}

\paragraph{benchmark.py}
\label{appendix:static-dynamic-bridge.benchmark.py}

\begin{minted}{python}
import numpy as np
import skimage
import timeit
import pylena as pln
from ValueSetExample import MyValueSet, MyStruct

grayscale = skimage.data.camera()

vs = MyValueSet()


def test_fast():
    global grayscale
    pln_ret = pln.morpho.stretch(grayscale)
    return pln_ret


def test_slow():
    global grayscale, vs
    pln_ret = pln.morpho.stretch(grayscale, vs)
    return pln_ret


def test_virtual_dispatch():
    global grayscale
    pln_ret = pln.morpho.stretch_virtual_dispatch(grayscale)
    return pln_ret


def test_virtual_dispatch_type_erased_value():
    global grayscale
    pln_ret = pln.morpho.stretch_virtual_dispatch_type_erased_value(grayscale)
    return pln_ret


if __name__ == "__main__":
    print("Benchmarking static-dynamic bridge...")
    cpp_fast_ret = timeit.timeit("test_fast()", globals=locals(), number=10)
    cpp_virtual_dispatch_ret = timeit.timeit(
        "test_virtual_dispatch()", globals=locals(), number=10)
    cpp_virtual_dispatch_type_erased_value_ret = timeit.timeit(
        "test_virtual_dispatch_type_erased_value()", globals=locals(), number=10)
    cpp_python_vs_ret = timeit.timeit(
        "test_slow()", globals=locals(), number=10)
    print("Benchmarking done...")

    print(
        "Native value-set with native C++ value-type: {0:.4f}sec".format(cpp_fast_ret))
    print(
        "Value-set with virtual dispatch with native C++ value-type: {0:.4f}sec".format(cpp_virtual_dispatch_ret))
    print(
        "Value-set with virtual dispatch with C++ type-erased values: {0:.4f}sec".format(cpp_virtual_dispatch_type_erased_value_ret))
    print(
        "Injected Python value-set with native C++ value-type: {0:.4f}sec".format(cpp_python_vs_ret))
\end{minted}
