\chapter{Concepts \& archetypes}
\label{appendix.concepts.and.archetypes}

\section{Concepts}

\subsection{Index}

\begin{minted}{c++}
// Index
template <typename Idx>
concept Index = std::signed_integral<Idx>;
\end{minted}


\subsection{Value}

\begin{minted}{c++}
// Value
template <typename Val>
concept Value = std::semiregular<Val>;

// ComparableValue
template <typename RegVal>
concept ComparableValue =
  std::regular<RegVal>;

// OrderedValue
template <typename STORegVal>
concept OrderedValue =
  std::regular<STORegVal> &&
  std::totally_ordered<STORegVal>;
\end{minted}


\subsection{Point}

\begin{minted}{c++}
// Point
template <typename P>
concept Point =
  std::regular<P> &&
  std::totally_ordered<P>;
\end{minted}


\subsection{Pixel}

\begin{minted}{c++}
// Pixel
template <class Pix> concept Pixel =
std::is_base_of_v<mln::details::Pixel<Pix>, Pix> &&
  std::copy_constructible<Pix> &&
  std::move_constructible<Pix> &&
  requires {
    typename pixel_value_t<Pix>;
    typename pixel_reference_t<Pix>;
    typename pixel_point_t<Pix>;
  } &&
  std::semiregular<pixel_value_t<Pix>> &&
  Point<pixel_point_t<Pix>> &&
  !std::is_const_v<pixel_value_t<Pix>> &&
  !std::is_reference_v<pixel_value_t<Pix>> &&
  requires(const Pix cpix, Pix pix, pixel_point_t<Pix> p) {
    { cpix.point() } -> std::convertible_to<pixel_point_t<Pix>>;
    { cpix.val() }   -> std::convertible_to<pixel_reference_t<Pix>>;
    { pix.shift(p) };
  };

// WritablePixel
template <typename WPix>
concept WritablePixel =
  Pixel<WPix> &&
  requires(const WPix cpix, pixel_value_t<WPix> v) {
    // Not deep-const, view-semantic.
    { cpix.val() = v };
    // Proxy rvalues must not be deep-const on their assignement semantic (unlike tuple...)
    { const_cast<typename WPix::reference const &&>(cpix.val()) = v };
  };

// OutputPixel
template <typename Pix>
concept OutputPixel = detail::WritablePixel<Pix>;
\end{minted}


\subsection{Ranges}

\begin{minted}{c++}
template <class C>
concept MDCursor =
  std::ranges::detail::forward_cursor<C> &&
  std::ranges::detail::forward_cursor<std::ranges::detail::begin_cursor_t<C>> &&
  requires (C c)
  {
    { c.read() } -> std::ranges::forward_range;
    c.end_cursor();
  };

template <class C>
concept NDCursor = std::semiregular<C> &&
  requires (C c)
  {
    { C::rank } -> std::same_as<int>;
    c.read();
    c.move_to_next(0);
    c.move_to_end(0);
  };

template <class C>
concept MDBidirectionalCursor = MDCursor<C> &&
  requires (C c)
  {
    c.move_to_prev();
    c.move_to_prev_line();
  };

template <class R>
concept MDRange =
  requires(R r)
  {
    { r.rows() } -> std::ranges::forward_range;
    { r.begin_cursor() } -> MDCursor;
    { r.end_cursor() } -> std::same_as<std::ranges::default_sentinel_t>;
  };

template <class R>
concept MDBidirectionalRange = MDRange<R> &&
  requires (R r)
  {
    { r.rrows() } -> std::ranges::forward_range;
    { r.rbegin_cursor() } -> MDCursor;
    { r.rend_cursor() } -> std::same_as<std::ranges::default_sentinel_t>;
  };

template <class R>
concept mdrange = MDRange<R> || std::ranges::range<R>;

template <class R, class V>
concept output_mdrange = mdrange<R> && std::ranges::output_range<mdrange_row_t<R>, V>;

template <class R>
concept reversible_mdrange = MDBidirectionalRange<R> || std::ranges::bidirectional_range<R>;
\end{minted}


\subsection{Domain}

\begin{minted}{c++}
// Domain
template <typename Dom>
concept Domain =
  mln::ranges::mdrange<Dom> &&
  Point<mln::ranges::mdrange_value_t<Dom>> &&
  requires(const Dom cdom, mln::ranges::mdrange_value_t<Dom> p) {
    { cdom.has(p) }   -> std::same_as<bool>;
    { cdom.empty() }  -> std::same_as<bool>;
    { cdom.dim() }    -> std::same_as<int>;
  };


// SizedDomain
template <typename Dom>
concept SizedDomain =
  Domain<Dom> &&
  requires(const Dom cdom) {
    { cdom.size() } -> std::unsigned_integral;
  };

// ShapedDomain
template <typename Dom>
concept ShapedDomain =
  SizedDomain<Dom> &&
  requires(const Dom cdom) {
    { cdom.extents() }  -> std::ranges::forward_range;
  };
\end{minted}


\subsection{Extension}

\begin{minted}{c++}
template <typename Ext, typename Pnt>
concept Extension =
  std::is_base_of_v<mln::Extension<Ext>, Ext> &&
  requires {
    typename Ext::support_fill;
    typename Ext::support_mirror;
    typename Ext::support_periodize;
    typename Ext::support_clamp;
    typename Ext::support_extend_with;
  } &&
  Value<typename Ext::value_type> &&
  requires (const Ext cext,
      mln::archetypes::StructuringElement<
        Pnt,
        mln::archetypes::Pixel> se, const Pnt pnt) {
    { cext.fit(se) }      -> bool;
    { cext.extent() }     -> int;
  };

template <typename Ext>
concept FillableExtension =
  Extension<Ext> &&
  std::convertible_to<typename Ext::support_fill, std::true_type> &&
  requires {
    typename Ext::value_type;
  } &&
  requires (Ext ext, const Ext cext, const typename Ext::value_type& v) {
    { ext.fill(v) };
    { cext.is_fill_supported() }  -> bool;
  };

template <typename Ext>
concept MirrorableExtension =
  Extension<Ext> &&
  std::convertible_to<typename Ext::support_mirror, std::true_type> &&
  requires (Ext ext, const Ext cext, std::size_t padding) {
    { ext.mirror() };
    { ext.mirror(padding) };
    { cext.is_mirror_supported() }  -> bool;
  };

template <typename Ext>
concept PeriodizableExtension =
  Extension<Ext> &&
  std::convertible_to<typename Ext::support_periodize, std::true_type> &&
  requires (Ext ext, const Ext cext) {
    { ext.periodize() };
    { cext.is_periodize_supported() }  -> bool;
  };

template <typename Ext>
concept ClampableExtension =
  Extension<Ext> &&
  std::convertible_to<typename Ext::support_clamp, std::true_type> &&
  requires (Ext ext, const Ext cext) {
    { ext.clamp() };
    { cext.is_clamp_supported() }  -> bool;
  };

template <typename Ext, typename U>
concept ExtendWithExtension =
  Extension<Ext> &&
  std::convertible_to<typename Ext::support_extend_with, std::true_type> &&
  InputImage<U> &&
  requires {
    typename Ext::value_type;
    typename Ext::point_type;
  } &&
  std::convertible_to<typename U::value_type, typename Ext::value_type> &&
  std::convertible_to<typename Ext::point_type, typename U::point_type> &&
  requires (Ext ext, const Ext cext, U u, typename Ext::point_type offset) {
    { ext.extend_with(u, offset) };
    { cext.is_extend_with_supported() }  -> bool;
  };
\end{minted}


\subsection{Image}

\begin{minted}{c++}
template <typename I>
concept Image =
  // Minimum constraint on image object
  // Do not requires DefaultConstructible
  std::is_base_of_v<mln::details::Image<I>, I> &&
  std::copy_constructible<I> &&
  std::move_constructible<I> &&
  std::derived_from<image_category_t<I>, forward_image_tag> &&
  requires {
    typename image_pixel_t<I>;
    typename image_point_t<I>;
    typename image_value_t<I>;
    typename image_domain_t<I>;
    typename image_reference_t<I>;
    typename image_concrete_t<I>;
    typename image_ch_value_t<I, mln::archetypes::Value>;
    // traits
    typename image_indexable_t<I>;
    typename image_accessible_t<I>;
    typename image_extension_category_t<I>;
    typename image_category_t<I>;
    typename image_view_t<I>;
  } &&
  Pixel<image_pixel_t<I>> &&
  Point<image_point_t<I>> &&
  Value<image_value_t<I>> &&
  Domain<image_domain_t<I>> &&
  std::convertible_to<pixel_point_t<image_pixel_t<I>>, image_point_t<I>> &&
  std::convertible_to<pixel_reference_t<image_pixel_t<I>>, image_reference_t<I>> &&
  // Here we don't want a convertible constraint as value_type is the decayed type and should really be the same
  std::same_as<pixel_value_t<image_pixel_t<I>>, image_value_t<I>> &&
  std::common_reference_with<image_reference_t<I>&&, image_value_t<I>&> &&
  std::common_reference_with<image_reference_t<I>&&, image_value_t<I>&&> &&
  std::common_reference_with<image_value_t<I>&&, const image_value_t<I>&> &&
  requires(I ima, const I cima, image_domain_t<I> d, image_point_t<I> p) {
    { cima.template ch_value<mln::archetypes::Value>() }
        -> std::convertible_to<image_ch_value_t<I, mln::archetypes::Value>>;
    { cima.concretize() } -> std::convertible_to<image_concrete_t<I>>;
    { cima.domain() }     -> std::convertible_to<image_domain_t<I>>;
    { ima.pixels() }  -> mln::ranges::mdrange;
    { ima.values() }  -> mln::ranges::mdrange;
    requires std::convertible_to<mln::ranges::mdrange_value_t<decltype(ima.pixels())>, image_pixel_t<I>>;
    requires std::convertible_to<mln::ranges::mdrange_value_t<decltype(ima.values())>, image_value_t<I>>;
  };

namespace detail
{
  // WritableImage
  template <typename I>
  concept WritableImage =
    Image<I> &&
    OutputPixel<image_pixel_t<I>> &&
    requires(I ima) {
    { ima.values() }  -> mln::ranges::output_mdrange<image_value_t<I>>;
      // Check Writability of each pixel of the range
      requires OutputPixel<
                  std::common_type_t<
                    mln::ranges::mdrange_value_t<decltype(ima.pixels())>,
                    image_pixel_t<I>>>;
    };
} // namespace detail


// InputImage
template <typename I>
concept InputImage = Image<I>;


// ForwardImage
template <typename I>
concept ForwardImage = InputImage<I>;


// IndexableImage
template <typename I>
concept IndexableImage =
  Image<I> &&
  requires {
    typename image_index_t<I>;
  } &&
  image_indexable_v<I> &&
  requires (I ima, image_index_t<I> k) {
    { ima[k] }  -> std::same_as<image_reference_t<I>>; // For concrete image it returns a const_reference
  };


namespace detail
{
  // WritableIndexableImage
  template <typename I>
  concept WritableIndexableImage =
    WritableImage<I> &&
    IndexableImage<I> &&
    requires(I ima, image_index_t<I> k, image_value_t<I> v) {
      { ima[k] = v } -> std::same_as<image_reference_t<I>>;
    };
} // namespace detail

// AccessibleImage
template <typename I>
concept AccessibleImage =
  Image<I> &&
  image_accessible_v<I> &&
  requires (I ima, image_point_t<I> p) {
    { ima(p) }              -> std::same_as<image_reference_t<I>>; // For concrete image it returns a const_reference
    { ima.at(p) }           -> std::same_as<image_reference_t<I>>; // idem
    { ima.pixel(p) }    -> std::same_as<image_pixel_t<I>>; // For concrete image pixel may propagate constness
    { ima.pixel_at(p) } -> std::same_as<image_pixel_t<I>>; // idem
  };


namespace detail
{
  // WritableAccessibleImage
  template <typename I>
  concept WritableAccessibleImage =
    detail::WritableImage<I> &&
    AccessibleImage<I> &&
    requires(I ima, image_point_t<I> p, image_value_t<I> v) {
      { ima(p) = v };
      { ima.at(p) = v };
    };
} // namespace detail

// IndexableAndAccessibleImage
template <typename I>
concept IndexableAndAccessibleImage =
  IndexableImage<I> &&
  AccessibleImage<I> &&
  requires (const I cima, image_index_t<I> k, image_point_t<I> p) {
    { cima.point_at_index(k) }  -> std::same_as<image_point_t<I>>;
    { cima.index_of_point(p) }  -> std::same_as<image_index_t<I>>;
    { cima.delta_index(p) }     -> std::same_as<image_index_t<I>>;
  };


namespace detail
{
  // WritableIndexableAndAccessibleImage
  template <typename I>
  concept WritableIndexableAndAccessibleImage =
    IndexableAndAccessibleImage<I> &&
    detail::WritableImage<I> &&
    detail::WritableIndexableImage<I>;
} // namespace detail

// BidirectionalImage (not in STL term)
template <typename I>
concept BidirectionalImage =
  Image<I> &&
  std::derived_from<image_category_t<I>, bidirectional_image_tag> &&
  requires (I ima) {
  { ima.pixels() }  -> mln::ranges::reversible_mdrange;
  { ima.values() }  -> mln::ranges::reversible_mdrange;
};


namespace detail
{
  // WritableBidirectionalImage
  template <typename I>
  concept WritableBidirectionalImage =
    WritableImage<I> &&
    BidirectionalImage<I>;
} // namespace detail

// RawImage (not contiguous, stride = padding)
template <typename I>
concept RawImage =
  IndexableAndAccessibleImage<I> &&
  BidirectionalImage<I> &&
  std::derived_from<image_category_t<I>, raw_image_tag> &&
  requires (I ima, const I cima, int dim) {
    { ima.data() }        -> std::convertible_to<const image_value_t<I>*>; // data() may be proxied by a view
    { cima.stride(dim) } -> std::same_as<std::ptrdiff_t>;
  };


namespace detail
{
  // WritableRawImage
  template <typename I>
  concept WritableRawImage =
    WritableImage<I> &&
    WritableIndexableAndAccessibleImage<I> &&
    WritableBidirectionalImage<I> &&
    RawImage<I> &&
    requires(I ima, image_value_t<I> v) {
      { ima.data() }        -> std::convertible_to<image_value_t<I>*>;
      { *(ima.data()) = v };
    };
} // namespace detail


// OutputImage
// Usage: RawImage<I> && OutputImage<I>
template <typename I>
concept OutputImage =
  (not ForwardImage<I> || (detail::WritableImage<I>)) &&
  (not IndexableImage<I> || (detail::WritableIndexableImage<I>)) &&
  (not AccessibleImage<I> || (detail::WritableAccessibleImage<I>)) &&
  (not IndexableAndAccessibleImage<I> ||
    (detail::WritableIndexableAndAccessibleImage<I>)) &&
  (not BidirectionalImage<I> || (detail::WritableBidirectionalImage<I>)) &&
  (not RawImage<I> || (detail::WritableRawImage<I>));


template <typename I>
concept WithExtensionImage =
  Image<I> &&
  requires {
    typename image_extension_t<I>;
  } &&
  Extension<image_extension_t<I>> &&
  not ::std::same_as<mln::extension::none_extension_tag, image_extension_category_t<I>> &&
  requires (I ima, image_point_t<I> p) {
    { ima.extension() } -> ::std::convertible_to<image_extension_t<I>>;
  };


// ConcreteImage
template <typename I>
concept ConcreteImage =
  Image<I> &&
  std::semiregular<I> &&  // A concrete image is default constructible
  not image_view_v<I>;


// ViewImage
template <typename I>
concept ViewImage =
  Image<I> &&
  image_view_v<I>;
\end{minted}


\subsection{Structuring Element}

\begin{minted}{c++}
namespace details
{
  template <typename SE>
  concept DynamicStructuringElement =
    requires (SE se) {
    { se.radial_extent() }  -> std::same_as<int>;
    };


  constexpr bool implies(bool a, bool b) { return !a || b; }
}


template <typename SE, typename P>
concept StructuringElement =
  std::convertible_to<SE, mln::details::StructuringElement<SE>> &&
  std::ranges::regular_invocable<SE, P> &&
  std::ranges::regular_invocable<SE, mln::archetypes::PixelT<P>> &&
  requires {
    typename SE::category;
    typename SE::incremental;
    typename SE::decomposable;
    typename SE::separable;
  } &&
  std::convertible_to<typename SE::category, mln::adaptative_neighborhood_tag> &&
  details::implies(std::convertible_to<typename SE::category, mln::dynamic_neighborhood_tag>,
                    details::DynamicStructuringElement<SE>) &&
  requires (SE se, const SE cse, P p, mln::archetypes::PixelT<P> px) {
    { se(p) }         -> std::ranges::forward_range;
    { se(px) }        -> std::ranges::forward_range;
    { cse.offsets() } -> std::ranges::forward_range;

    requires std::convertible_to<std::ranges::range_value_t<decltype(se(p))>, P>;
    requires std::Pixel<std::ranges::range_value_t<decltype(se(px))>>;
    requires std::convertible_to<std::ranges::range_value_t<decltype(cse.offsets())>, P>;
  };

namespace details
{
  template <typename R, typename P>
  concept RangeOfStructuringElement =
    StructuringElement<std::ranges::range_value_t<R>, P>;
}



template <typename SE, typename P>
concept DecomposableStructuringElement =
  StructuringElement<SE, P> &&
  std::convertible_to<typename SE::decomposable, std::true_type> &&
  requires(const SE se) {
    { se.is_decomposable() }  -> std::same_as<bool>;
    { se.decompose() }        -> std::ranges::forward_range;
    requires details::RangeOfStructuringElement<decltype(se.decompose()), P>;
  };


template <typename SE, typename P>
concept SeparableStructuringElement =
  StructuringElement<SE, P> &&
  std::convertible_to<typename SE::separable, std::true_type> &&
  requires(const SE se) {
    { se.is_separable() } -> std::same_as<bool>;
    { se.separate() }     -> std::ranges::forward_range;
    requires details::RangeOfStructuringElement<decltype(se.separate()), P>;
  };


template <typename SE, typename P>
concept IncrementalStructuringElement =
  StructuringElement<SE, P> &&
  std::convertible_to<typename SE::incremental, std::true_type> &&
  requires(const SE se) {
    { se.inc() }  -> StructuringElement<P>;
    { se.dec() }  -> StructuringElement<P>;
  };
\end{minted}

\subsection{Neighborhood}

\begin{minted}{c++}
template <typename SE, typename P>
concept Neighborhood =
  StructuringElement<SE, P> &&
  requires (SE se, P p, mln::archetypes::PixelT<P> px) {
    { se.before(p) }  -> std::ranges::forward_range;
    { se.after(p) }   -> std::ranges::forward_range;
    { se.before(px) } -> std::ranges::forward_range;
    { se.after(px) }  -> std::ranges::forward_range;

    requires std::convertible_to<std::ranges::range_value_t<decltype(se.before(p))>, P>;
    requires std::convertible_to<std::ranges::range_value_t<decltype(se.after(p))>, P>;
    requires std::Pixel<std::ranges::range_value_t<decltype(se.before(px))>>;
    requires std::Pixel<std::ranges::range_value_t<decltype(se.after(px))>>;
  };
\end{minted}


\section{Archetypes}

\subsection{Index}

\begin{minted}{c++}
using Index = int;

static_assert(mln::concepts::Index<Index>, "Index archetype does not model the Index concept!");
\end{minted}


\subsection{Value}

\begin{minted}{c++}
struct Value
{
};

struct ComparableValue
{
};
bool operator==(const ComparableValue&, const ComparableValue&);
bool operator!=(const ComparableValue&, const ComparableValue&);


struct OrderedValue
{
};
bool operator==(const OrderedValue&, const OrderedValue&);
bool operator!=(const OrderedValue&, const OrderedValue&);
bool operator<(const OrderedValue&, const OrderedValue&);
bool operator>(const OrderedValue&, const OrderedValue&);
bool operator<=(const OrderedValue&, const OrderedValue&);
bool operator>=(const OrderedValue&, const OrderedValue&);

static_assert(mln::concepts::Value<Value>, "Value archetype does not model the Value concept!");
static_assert(mln::concepts::ComparableValue<ComparableValue>, "ComparableValue archetype does not model the ComparableValue concept!");
static_assert(mln::concepts::OrderedValue<OrderedValue>, "OrderedValue archetype does not model the OrderedValue concept!");
\end{minted}


\subsection{Point}

\begin{minted}{c++}
struct Point final
{
};

bool operator==(const Point&, const Point&);
bool operator!=(const Point&, const Point&);
bool operator<(const Point&, const Point&);
bool operator>(const Point&, const Point&);
bool operator<=(const Point&, const Point&);
bool operator>=(const Point&, const Point&);

static_assert(mln::concepts::Point<Point>, "Point archetype does not model the Point concept!");
\end{minted}


\subsection{Pixel}

\begin{minted}{c++}
namespace details
{
  template <class P, class V>
  struct PixelT
  {
    using value_type = V;
    using point_type = P;
    using reference  = const value_type&;

    PixelT()              = delete;
    PixelT(const PixelT&) = default;
    PixelT(PixelT&&)      = default;
    PixelT& operator=(const PixelT&) = delete;
    PixelT& operator=(PixelT&&) = delete;

    point_type point() const;
    reference  val() const;
    void       shift(const P& dp);
  };

  struct OutputPixel : PixelT<Point, Value>
  {
    using reference = Value&;
    reference val() const;
  };


  template <class Pix>
  struct AsPixel : Pix, mln::details::Pixel<AsPixel<Pix>>
  {
  };
} // namespace details

template <class P, class V = Value>
using PixelT      = details::AsPixel<details::PixelT<P, V>>;
using Pixel       = PixelT<Point, Value>;
using OutputPixel = details::AsPixel<details::OutputPixel>;

static_assert(mln::concepts::Pixel<Pixel>, "Pixel archetype does not model the Pixel concept!");
static_assert(mln::concepts::OutputPixel<OutputPixel>, "OutputPixel archetype does not model the OutputPixel concept!");
\end{minted}


\subsection{Ranges}

\begin{minted}{c++}
  // TODO
\end{minted}


\subsection{Domain}

\begin{minted}{c++}
struct Domain
{
  using value_type = Point;
  using reference  = Point&;

  value_type* begin();
  value_type* end();

  bool has(value_type) const;
  bool empty() const;
  int  dim() const;
};

static_assert(mln::concepts::Domain<Domain>, "Domain archetype does not model the Domain concept!");

struct SizedDomain : Domain
{
  unsigned size() const;
};

static_assert(mln::concepts::SizedDomain<SizedDomain>,
                          "SizedDomain archetype does not model the SizedDomain concept!");

struct ShapedDomain final : SizedDomain
{
  static constexpr std::size_t  ndim = 1;
  value_type                    shape() const;
  std::array<std::size_t, ndim> extents() const;
};

static_assert(mln::concepts::ShapedDomain<ShapedDomain>,
                          "ShapedDomain archetype does not model the ShapedDomain concept!");
\end{minted}


\subsection{Extension}

\begin{minted}{c++}
  // TODO
\end{minted}


\subsection{Image}

\begin{minted}{c++}
namespace details
{
  template <class I>
  struct AsImage : I, mln::details::Image<AsImage<I>>
  {
    using I::I;

    using concrete_type = AsImage<typename I::concrete_type>;
    concrete_type concretize() const;


    template <typename V>
    using ch_value_type = AsImage<typename I::template ch_value_type<V>>;

    template <typename V>
    ch_value_type<V> ch_value() const;
  };


  struct ConcreteImage
  {
    using pixel_type = archetypes::Pixel;
    using value_type     = pixel_value_t<mln::archetypes::Pixel>;
    using reference      = pixel_reference_t<mln::archetypes::Pixel>;
    using point_type     =  std::ranges::range_value_t<Domain>;
    using domain_type    = Domain;
    using category_type  = forward_image_tag;
    using concrete_type  = ConcreteImage;

    template <class V>
    using ch_value_type = ConcreteImage;

    // additional traits
    using extension_category = mln::extension::none_extension_tag;
    using indexable          = std::false_type;
    using accessible         = std::false_type;
    using view               = std::false_type;

    ConcreteImage()                     = default;
    ConcreteImage(const ConcreteImage&) = default;
    ConcreteImage(ConcreteImage&&)      = default;
    ConcreteImage& operator=(const ConcreteImage&) = default;
    ConcreteImage& operator=(ConcreteImage&&) = default;

    domain_type domain() const;


    struct pixel_range
    {
      const pixel_type* begin();
      const pixel_type* end();
    };
    pixel_range pixels();


    struct value_range
    {
      const value_type* begin();
      const value_type* end();
    };

    value_range values();
  };


  struct ViewImage : ConcreteImage
  {
    using view = std::true_type;

    ViewImage()                 = delete;
    ViewImage(const ViewImage&) = default;
    ViewImage(ViewImage&&)      = default;
    ViewImage& operator=(const ViewImage&) = delete;
    ViewImage& operator=(ViewImage&&) = delete;
  };

  using Image = ViewImage;


  struct OutputImage : Image
  {
    using pixel_type = archetypes::OutputPixel;
    using reference      = pixel_reference_t<mln::archetypes::OutputPixel>;

    struct pixel_range
    {
      const pixel_type* begin();
      const pixel_type* end();
    };

    pixel_range pixels();

    struct value_range
    {
      value_type* begin();
      value_type* end();
    };

    value_range values();
  };


  struct OutputIndexableImage : OutputImage
  {
    using index_type = int;
    using indexable  = std::true_type;

    using concrete_type = OutputIndexableImage;

    template <class V>
    using ch_value_type = OutputIndexableImage;

    reference operator[](index_type);
  };


  struct IndexableImage : Image
  {
    using index_type = int;
    using indexable  = std::true_type;

    using concrete_type = OutputIndexableImage;

    template <class V>
    using ch_value_type = OutputIndexableImage;


    reference operator[](index_type);
  };

  struct OutputAccessibleImage : OutputImage
  {
    using accessible    = std::true_type;
    using concrete_type = OutputAccessibleImage;

    template <class V>
    using ch_value_type = OutputAccessibleImage;


    reference      operator()(point_type);
    reference      at(point_type);
    pixel_type pixel(point_type);
    pixel_type pixel_at(point_type);
  };


  struct AccessibleImage : Image
  {
    using accessible    = std::true_type;
    using concrete_type = OutputAccessibleImage;

    template <class V>
    using ch_value_type = OutputAccessibleImage;

    reference      operator()(point_type);
    reference      at(point_type);
    pixel_type pixel(point_type);
    pixel_type pixel_at(point_type);
  };


  struct OutputIndexableAndAccessibleImage : OutputAccessibleImage
  {
    using index_type = int;
    using indexable  = std::true_type;

    using concrete_type = OutputIndexableAndAccessibleImage;

    template <class V>
    using ch_value_type = OutputIndexableAndAccessibleImage;


    reference  operator[](index_type);
    point_type point_at_index(index_type) const;
    index_type index_of_point(point_type) const;
    index_type delta_index(point_type) const;
  };


  struct IndexableAndAccessibleImage : AccessibleImage
  {
    using index_type = int;
    using indexable  = std::true_type;

    using concrete_type = OutputIndexableAndAccessibleImage;

    template <class V>
    using ch_value_type = OutputIndexableAndAccessibleImage;

    reference  operator[](index_type);
    point_type point_at_index(index_type) const;
    index_type index_of_point(point_type) const;
    index_type delta_index(point_type) const;
  };


  struct BidirectionalImage : Image
  {
    using category_type = bidirectional_image_tag;

    struct pixel_range
    {
      const pixel_type* begin();
      const pixel_type* end();
      pixel_range           reversed();
    };

    pixel_range pixels();

    struct value_range
    {
      const value_type* begin();
      const value_type* end();
      value_range       reversed();
    };

    value_range values();
  };


  struct OutputBidirectionalImage : BidirectionalImage
  {
    using pixel_type = archetypes::OutputPixel;
    using reference      = pixel_reference_t<mln::archetypes::OutputPixel>;

    struct value_range
    {
      value_type* begin();
      value_type* end();
      value_range reversed();
    };
    value_range values();


    struct pixel_range
    {
      const pixel_type* begin();
      const pixel_type* end();
      pixel_range           reversed();
    };

    pixel_range pixels();
  };


  struct RawImage : IndexableAndAccessibleImage
  {
    using category_type = raw_image_tag;
    using pixel_range   = BidirectionalImage::pixel_range;
    using value_range   = BidirectionalImage::value_range;

    pixel_range pixels();
    value_range values();


    const value_type* data() const;
    std::ptrdiff_t    strides(int) const;
  };


  struct OutputRawImage : OutputIndexableAndAccessibleImage
  {
    using category_type = raw_image_tag;
    using pixel_range   = OutputBidirectionalImage::pixel_range;
    using value_range   = OutputBidirectionalImage::value_range;

    pixel_range pixels();
    value_range values();

    value_type*    data() const;
    std::ptrdiff_t strides(int) const;
  };


  struct WithExtensionImage : Image
  {
    struct Extension
    {
      // FIXME
    };

    using extension_type = Extension;

    using extension_category = mln::extension::custom_extension_tag;

    extension_type extension() const;
  };
} // namespace details


using Image         = details::AsImage<details::Image>;
using ConcreteImage = details::AsImage<details::ConcreteImage>;
using ViewImage     = details::AsImage<details::ViewImage>;

using ForwardImage       = Image;
using BidirectionalImage = details::AsImage<details::BidirectionalImage>;
using RawImage           = details::AsImage<details::RawImage>;

using InputImage                  = Image;
using IndexableImage              = details::AsImage<details::IndexableImage>;
using AccessibleImage             = details::AsImage<details::AccessibleImage>;
using IndexableAndAccessibleImage = details::AsImage<details::IndexableAndAccessibleImage>;


using OutputImage              = details::AsImage<details::OutputImage>;
using OutputForwardImage       = OutputImage;
using OutputBidirectionalImage = details::AsImage<details::OutputBidirectionalImage>;
using OutputRawImage           = details::AsImage<details::OutputRawImage>;

using OutputIndexableImage              = details::AsImage<details::OutputIndexableImage>;
using OutputAccessibleImage             = details::AsImage<details::OutputAccessibleImage>;
using OutputIndexableAndAccessibleImage = details::AsImage<details::OutputIndexableAndAccessibleImage>;

using WithExtensionImage = details::AsImage<details::WithExtensionImage>;
\end{minted}


\subsection{Structuring Element}

\begin{minted}{c++}
namespace details
{
  template <class P, class Pix>
  requires mln::concepts::Point<P>&& mln::concepts::Pixel<Pix>
  struct StructuringElement
  {
    using category     = adaptative_neighborhood_tag;
    using incremental  = std::false_type;
    using decomposable = std::false_type;
    using separable    = std::false_type;

     std::ranges::subrange<P*> operator()(P p);

     std::ranges::subrange<Pix*> operator()(Pix px);
     std::ranges::subrange<P*>   offsets() const;
  };


  template <class SE>
  struct AsSE : SE, mln::details::Neighborhood
  helper<AsSE<SE>>
  {
  };
} // namespace details

template <class P = Point, class Pix = PixelT<P>>
using StructuringElement = details::AsSE<details::StructuringElement<P, Pix>>;


namespace details
{
  template <class P, class Pix>
  struct DecomposableStructuringElement : StructuringElement<P, Pix>
  {
    using decomposable = std::true_type;

    bool                                                                is_decomposable() const;
    std::ranges::subrange<mln::archetypes::StructuringElement<P, Pix>*> decompose() const;
  };

  template <class P, class Pix>
  struct SeparableStructuringElement : StructuringElement<P, Pix>
  {
    using separable = std::true_type;

    bool                                                                is_separable() const;
    std::ranges::subrange<mln::archetypes::StructuringElement<P, Pix>*> separate() const;
  };

  template <class P, class Pix>
  struct IncrementalStructuringElement : StructuringElement<P, Pix>
  {
    using incremental = std::true_type;

    archetypes::StructuringElement<P, Pix> inc() const;
    archetypes::StructuringElement<P, Pix> dec() const;
  };
} // namespace details


template <class P = Point, class Pix = PixelT<P>>
using DecomposableStructuringElement = details::AsSE<details::DecomposableStructuringElement<P, Pix>>;

template <class P = Point, class Pix = PixelT<P>>
using SeparableStructuringElement = details::AsSE<details::SeparableStructuringElement<P, Pix>>;

template <class P = Point, class Pix = PixelT<P>>
using IncrementalStructuringElement = details::AsSE<details::IncrementalStructuringElement<P, Pix>>;
\end{minted}


\subsection{Neighborhood}

\begin{minted}{c++}
namespace details
{
  template <class P, class Pix>
  requires mln::concepts::Point<P>&& mln::concepts::Pixel<Pix>
  struct Neighborhood : StructuringElement<P, Pix>
  {
     std::ranges::iterator_range<P*>   before(P p);
     std::ranges::iterator_range<P*>   after(P p);
     std::ranges::iterator_range<Pix*> before(Pix px);
     std::ranges::iterator_range<Pix*> after(Pix px);
  };


  template <class N>
  struct AsNeighborhood : N, mln::details::Neighborhood<AsNeighborhood<N>>
  {
  };

} // namespace details

template <class P = Point, class Pix = PixelT<P>>
using Neighborhood = details::AsSE<details::Neighborhood<P, Pix>>;
\end{minted}
