\chapter{Concepts \& archetypes}
\label{appendix:concepts.and.archetypes}

\section{The fundamentals}

\subsection{Value}

\paragraph{Concept table}

\begin{table}[H]
  \begin{scriptsize}
    \begin{tabular}{llll}
      \cline{1-4}
      \thead{Concept} & \thead{Modeling type} & \thead{Inherit behavior from} & \thead{Instance of type}      \\
      \cline{1-4}
      Value           & \texttt{Val}          & $\emptyset$                   & \texttt{val}                  \\
      ComparableValue & \texttt{CmpVal}       & Value                         & \texttt{cmp\_val1, cmp\_val2} \\
      OrderedValue    & \texttt{OrdVal}       & ComparableValue               & \texttt{ord\_val1, ord\_val2} \\
      \cline{1-4}
    \end{tabular}
    \smallskip

    \begin{tabular}{llll}
      \cline{1-4}
      \thead{Concept}                                       & \thead{Expression}                                                   & \thead{Return
      Type}                                                 & \thead{Description}                                                                     \\
      \cline{1-4}
      \multicolumn{1}{c|}{Value}                            & \texttt{std::semiregular<Val>}                                       &
      \texttt{std::true\_type}                              & \makecell[l]{\texttt{Val} is a semiregular type. It can be:                             \\
      copied, moved, swapped,                                                                                                                         \\ and default constructed.}                                                                                                    \\
      \cline{1-4}
      \multicolumn{1}{c|}{\multirow{2}{*}{ComparableValue}} & \texttt{std::regular<CmpVal>}                                        &
      \texttt{std::true\_type}                              & \makecell[l]{\texttt{CmpVal} is a regular type. It is a semiregular                     \\
      type that is equality comparable.}                                                                                                              \\
      \multicolumn{1}{c|}{}                                 & \texttt{cmp\_val1 == cmp\_val2}                                      & \texttt{boolean}
                                                            & Supports equality comparison                                                            \\
      \cline{1-4}
      \multicolumn{1}{c|}{\multirow{2}{*}{OrderedValue}}    & \texttt{std::totally\_ordered<OrdVal>}                               &
      \texttt{std::true\_type}                              & \makecell[l]{\texttt{CmpVal} is a totally ordered as well as                            \\
      a regular type. Additionally the expressions                                                                                                    \\ must be equality preserving.}                                                                                           \\
      \multicolumn{1}{c|}{}                                 & \texttt{ord\_val1 < ord\_val2}                                       & \texttt{boolean}
                                                            & \multicolumn{1}{l}{\multirow{2}{*}{Supports inequality comparisons}}                    \\
      \multicolumn{1}{c|}{}                                 & \texttt{ord\_val1 <= ord\_val2, \dots}                               & \texttt{boolean}
                                                            & \multicolumn{1}{l}{}                                                                    \\
      \cline{1-4}
    \end{tabular}
    \smallskip

    \caption{Concepts Value: expressions}
  \end{scriptsize}
  \label{table:concept.value.expressions}
\end{table}

\paragraph{Concept code}
\begin{minted}{c++}
// Value
template <typename Val>
concept Value = std::semiregular<Val>;

// ComparableValue
template <typename RegVal>
concept ComparableValue =
  std::regular<RegVal>;

// OrderedValue
template <typename STORegVal>
concept OrderedValue =
  std::regular<STORegVal> &&
  std::totally_ordered<STORegVal>;
\end{minted}

\paragraph{Archetype code}

\begin{minted}{c++}
struct Value
{
};

struct ComparableValue
{
};
bool operator==(const ComparableValue&, const ComparableValue&);
bool operator!=(const ComparableValue&, const ComparableValue&);


struct OrderedValue
{
};
bool operator==(const OrderedValue&, const OrderedValue&);
bool operator!=(const OrderedValue&, const OrderedValue&);
bool operator<(const OrderedValue&, const OrderedValue&);
bool operator>(const OrderedValue&, const OrderedValue&);
bool operator<=(const OrderedValue&, const OrderedValue&);
bool operator>=(const OrderedValue&, const OrderedValue&);

static_assert(mln::concepts::Value<Value>, "Value archetype does not model the Value concept!");
static_assert(mln::concepts::ComparableValue<ComparableValue>, "ComparableValue archetype does not model the ComparableValue concept!");
static_assert(mln::concepts::OrderedValue<OrderedValue>, "OrderedValue archetype does not model the OrderedValue concept!");
\end{minted}


\clearpage

\subsection{Point}

\paragraph{Concept table}

\begin{table}[H]
  \begin{scriptsize}
    \begin{tabular}{llll}
      \cline{1-4}
      \thead{Concept} & \thead{Modeling type} & \thead{Inherit behavior from} & \thead{Instance of type} \\
      \cline{1-4}
      Point           & \texttt{Pnt}          & $\emptyset$                   & \texttt{pnt1, pnt2}      \\
      \cline{1-4}
    \end{tabular}
    \smallskip

    \begin{tabular}{llll}
      \cline{1-4}
      \thead{Concept}                             & \thead{Expression}                     & \thead{Return Type}      &
      \thead{Description}                                                                                               \\
      \cline{1-4}
      \multicolumn{1}{c|}{\multirow{4}{*}{Point}} & \texttt{std::regular<Pnt>}             & \texttt{std::true\_type} &
      \makecell[l]{\texttt{Pnt} is a regular type. It can be:                                                           \\ copied, moved, swapped, and default
      constructed.                                                                                                      \\ It also is equality comparable.} \\
      \multicolumn{1}{c|}{}                       & \texttt{std::totally\_ordered<OrdVal>} & \texttt{std::true\_type} &
      \makecell[l]{\texttt{Pnt} is a totally ordered as well as a regular type.                                         \\ Additionally the expressions must be \\
      equality preserving.}                                                                                             \\
      \multicolumn{1}{c|}{}                       & \texttt{pnt1 < pnt2}                   & \texttt{boolean}         &
      \multicolumn{1}{l}{\multirow{2}{*}{supports inequality comparisons}}                                              \\
      \multicolumn{1}{c|}{}                       & \texttt{pnt1 <= pnt2, \dots}           & \texttt{boolean}         &
      \multicolumn{1}{l}{}                                                                                              \\
      \cline{1-4}
    \end{tabular}
    \smallskip

    \caption{Concepts Point: expressions}
  \end{scriptsize}
  \label{table:concept.point.expressions}
\end{table}

\paragraph{Concept code}

\begin{minted}{c++}
// Point
template <typename P>
concept Point =
  std::regular<P> &&
  std::totally_ordered<P>;
\end{minted}

\paragraph{Archetype code}

\begin{minted}{c++}
struct Point final
{
};

bool operator==(const Point&, const Point&);
bool operator!=(const Point&, const Point&);
bool operator<(const Point&, const Point&);
bool operator>(const Point&, const Point&);
bool operator<=(const Point&, const Point&);
bool operator>=(const Point&, const Point&);

static_assert(mln::concepts::Point<Point>, "Point archetype does not model the Point concept!");
\end{minted}


\clearpage

\subsection{Pixel}

\paragraph{Concept table}

\begin{table}[H]
  \begin{scriptsize}
    \begin{tabular}{llll}
      \cline{1-4}
      \thead{Concept} & \thead{Modeling type} & \thead{Inherit behavior from} & \thead{Instance of type} \\
      \cline{1-4}
      Pixel           & \texttt{Pix}          & $\emptyset$                   & \texttt{pix}             \\
      OutputPixel     & \texttt{OPix}         & Pixel                         & \texttt{opix}            \\
      \cline{1-4}
    \end{tabular}
    \smallskip

    \begin{tabular}{llll}
      \cline{1-4}
      \thead{Concept}                             & \thead{Definition}       & \thead{Description}            &
      \thead{Requirement}                                                                                       \\
      \cline{1-4}
      \multicolumn{1}{c|}{\multirow{3}{*}{Pixel}} & \texttt{value\_type}     & \makecell[l]{Type of the value
      contained in the pixel.                                                                                   \\ Cannot be constant or reference.}       & Models
      the concept \texttt{Value}.                                                                               \\
      \multicolumn{1}{c|}{}                       & \texttt{reference\_type} & \makecell[l]{Type used to
      mutate the pixel's value                                                                                  \\ if non-const. Can be a proxy.}       & \makecell[l]{Models the concept \\
      \texttt{std::indirectly\_writable} if non-const.}                                                         \\
      \multicolumn{1}{c|}{}                       & \texttt{point\_type}     & Type of the pixel's point.     &
      Models the concept \texttt{Point}                                                                         \\
      \cline{1-4}
    \end{tabular}
    \smallskip

    \caption{Concepts Pixel: definitions}
  \end{scriptsize}
  \label{table:concept.pixel.definitions}
\end{table}

\begin{table}[H]
  \begin{scriptsize}
    \begin{tabular}{ll}
      \cline{1-2}
      \thead{Type}              & \thead{Instance of type} \\
      \cline{1-2}
      \texttt{Pix::value\_type} & \texttt{val}             \\
      \texttt{Pix::point\_type} & \texttt{pnt}             \\
      \cline{1-2}
    \end{tabular}
    \smallskip

    \begin{tabular}{llll}
      \cline{1-4}
      \thead{Concept}                             & \thead{Expression}        & \thead{Return Type}           &
      \thead{Description}                                                                                              \\
      \cline{1-4}
      \multicolumn{1}{c|}{\multirow{3}{*}{Pixel}} & \texttt{pix.val()}        & \texttt{Pix::reference\_type} &
      \makecell[l]{Access the pixel's value for read and/or write purpose.}                                            \\
      \multicolumn{1}{c|}{}                       & \texttt{pix.point()}      & \texttt{Pix::point\_type}     &
      \makecell[l]{Read the pixel's point.}                                                                            \\
      \multicolumn{1}{c|}{}                       & \texttt{pix.shift(pnt)}   & \texttt{void}                 & Shift
      pixel's point coordinate base on \texttt{pnt}'s coordinates.                                                     \\
      \cline{1-4}
      \multicolumn{1}{c|}{OutputPixel}            & \texttt{opix.val() = val} & \texttt{void}                 & Mutate
      pixel's value.                                                                                                   \\
      \cline{1-4}
    \end{tabular}
    \smallskip

    \caption{Concepts Pixel: expressions}
  \end{scriptsize}
  \label{table:concept.pixel.expressions}
\end{table}

\paragraph{Concept code}

\begin{minted}{c++}
// Pixel
template <class Pix>
concept Pixel =
  std::is_base_of_v<mln::details::Pixel<Pix>, Pix> &&
  std::copy_constructible<Pix> &&
  std::move_constructible<Pix> &&
  requires {
    typename pixel_value_t<Pix>;
    typename pixel_reference_t<Pix>;
    typename pixel_point_t<Pix>;
  } &&
  std::semiregular<pixel_value_t<Pix>> &&
  Point<pixel_point_t<Pix>> &&
  !std::is_const_v<pixel_value_t<Pix>> &&
  !std::is_reference_v<pixel_value_t<Pix>> &&
  requires(const Pix cpix, Pix pix, pixel_point_t<Pix> p) {
    { cpix.point() } -> std::convertible_to<pixel_point_t<Pix>>;
    { cpix.val() }   -> std::convertible_to<pixel_reference_t<Pix>>;
    { pix.shift(p) };
  };

// WritablePixel
template <typename WPix>
concept WritablePixel =
  Pixel<WPix> &&
  requires(const WPix cpix, pixel_value_t<WPix> v) {
    // Not deep-const, view-semantic.
    { cpix.val() = v };
    // Proxy rvalues must not be deep-const on their assignement semantic (unlike tuple...)
    { const_cast<typename WPix::reference const &&>(cpix.val()) = v };
  };

// OutputPixel
template <typename Pix>
concept OutputPixel = detail::WritablePixel<Pix>;
\end{minted}

\paragraph{Archetype code}

\begin{minted}{c++}
namespace details
{
  template <class P, class V>
  struct PixelT
  {
    using value_type = V;
    using point_type = P;
    using reference  = const value_type&;

    PixelT()              = delete;
    PixelT(const PixelT&) = default;
    PixelT(PixelT&&)      = default;
    PixelT& operator=(const PixelT&) = delete;
    PixelT& operator=(PixelT&&) = delete;

    point_type point() const;
    reference  val() const;
    void       shift(const P& dp);
  };

  struct OutputPixel : PixelT<Point, Value>
  {
    using reference = Value&;
    reference val() const;
  };


  template <class Pix>
  struct AsPixel : Pix, mln::details::Pixel<AsPixel<Pix>>
  {
  };
} // namespace details

template <class P, class V = Value>
using PixelT      = details::AsPixel<details::PixelT<P, V>>;
using Pixel       = PixelT<Point, Value>;
using OutputPixel = details::AsPixel<details::OutputPixel>;

static_assert(mln::concepts::Pixel<Pixel>, "Pixel archetype does not model the Pixel concept!");
static_assert(mln::concepts::OutputPixel<OutputPixel>, "OutputPixel archetype does not model the OutputPixel concept!");
\end{minted}


\clearpage

\subsection{Ranges}

\paragraph{Concept table}

\begin{table}[H]
  \begin{scriptsize}
    \begin{tabular}{llll}
      \cline{1-4}
      \thead{Concept}   & \thead{Modeling type} & \thead{Inherit behavior from} & \thead{Instance of type} \\
      \cline{1-4}
      MDRange           & \texttt{MDRng}        & $\emptyset$                   & \texttt{mdrng}           \\
      OutputMDRange     & \texttt{OMDRng}       & MDRange                       & \texttt{omdrng}          \\
      ReversibleMDRange & \texttt{RMDRng}       & MDRange                       & \texttt{rmdrng}          \\
      \cline{1-4}
    \end{tabular}
    \smallskip

    \begin{tabular}{llll}
      \cline{1-4}
      \thead{Concept}                               & \thead{Definition}       & \thead{Description}                      &
      \thead{Requirement}                                                                                                   \\
      \cline{1-4}
      \multicolumn{1}{c|}{\multirow{2}{*}{MDRange}} & \texttt{value\_type}     & \makecell[l]{Type of the value contained
      in the range.                                                                                                         \\ Cannot be constant or reference.} &  Models the concept \texttt{Value}. \\
      \multicolumn{1}{c|}{}                         & \texttt{reference\_type} & \makecell[l]{Type used to mutate the
      pixel's value                                                                                                         \\if non-const.                                                                                            Can be a proxy.}    & \makecell[l]{Models the concept \\
      \texttt{std::indirectly\_writable} if non-const.}                                                                     \\
      \cline{1-4}
    \end{tabular}

    \smallskip

    \caption{Concepts Ranges: definitions}
  \end{scriptsize}
  \label{table:concept.ranges.definitions}
\end{table}

\begin{table}[H]
  \begin{scriptsize}
    \begin{tabular}{ll}
      \cline{1-2}
      \thead{Type}                                 & \thead{Instance of type} \\
      \cline{1-2}
      \texttt{std::ranges::range\_value\_t<MDRng>} & \texttt{val}             \\
      \cline{1-2}
    \end{tabular}
    \smallskip

    \begin{tabular}{llll}
      \cline{1-4}
      \thead{Concept}                                         & \thead{Expression}                              &
      \thead{Return Type}                                     & \thead{Description}                                                                            \\
      \cline{1-4}
      \multicolumn{1}{c|}{\multirow{2}{*}{MDRange}}           & \texttt{mdrng.begin()}                          &
      \texttt{unspecified}                                    & \makecell[l]{Return a forward iterator allowing                                                \\ a traversing of the range.} \\
      \multicolumn{1}{c|}{}                                   & \texttt{mdrng.end()}                            &
      \texttt{unspecified}                                    & \makecell[l]{Return a sentinel allowing to                                                     \\know when the end is reached.} \\
      \cline{1-4}
      \multicolumn{1}{c|}{OutputMDRange}                      & \makecell[l]{\texttt{auto it = omdrng.begin()}                                                 \\
      \texttt{*it++ = val}}                                   & \texttt{void}                                   & \makecell[l]{Mutate a value inside the range \\
      then increment the iterator's position}                                                                                                                  \\
      \cline{1-4}
      \multicolumn{1}{c|}{\multirow{2}{*}{ReversibleMDRange}} & \texttt{rmdrng.rbegin()}                        &
      \texttt{unspecified}                                    & \makecell[l]{Return a forward iterator allowing                                                \\ a traversing of the range \\ starting from
      the end.}                                                                                                                                                \\
      \multicolumn{1}{c|}{}                                   & \texttt{rmdrng.rend()}                          &
      \texttt{unspecified}                                    & \makecell[l]{Return a sentinel allowing to                                                     \\ know when the end is reached.} \\
      \cline{1-4}
    \end{tabular}
    \smallskip

    \caption{Concepts Ranges: expressions}
  \end{scriptsize}
  \label{table:concept.ranges.expressions}
\end{table}

\paragraph{Concept code}

\begin{minted}{c++}
template <class C>
concept MDCursor =
  std::ranges::detail::forward_cursor<C> &&
  std::ranges::detail::forward_cursor<std::ranges::detail::begin_cursor_t<C>> &&
  requires (C c)
  {
    { c.read() } -> std::ranges::forward_range;
    c.end_cursor();
  };

template <class C>
concept NDCursor = std::semiregular<C> &&
  requires (C c)
  {
    { C::rank } -> std::same_as<int>;
    c.read();
    c.move_to_next(0);
    c.move_to_end(0);
  };

template <class C>
concept MDBidirectionalCursor = MDCursor<C> &&
  requires (C c)
  {
    c.move_to_prev();
    c.move_to_prev_line();
  };

template <class R>
concept MDRange =
  requires(R r)
  {
    { r.rows() } -> std::ranges::forward_range;
    { r.begin_cursor() } -> MDCursor;
    { r.end_cursor() } -> std::same_as<std::ranges::default_sentinel_t>;
  };

template <class R>
concept MDBidirectionalRange = MDRange<R> &&
  requires (R r)
  {
    { r.rrows() } -> std::ranges::forward_range;
    { r.rbegin_cursor() } -> MDCursor;
    { r.rend_cursor() } -> std::same_as<std::ranges::default_sentinel_t>;
  };

template <class R>
concept mdrange = MDRange<R> || std::ranges::range<R>;

template <class R, class V>
concept output_mdrange = mdrange<R> && std::ranges::output_range<mdrange_row_t<R>, V>;

template <class R>
concept reversible_mdrange = MDBidirectionalRange<R> || std::ranges::bidirectional_range<R>;
\end{minted}

\paragraph{Archetype code}

\begin{minted}{c++}
  // TODO
\end{minted}


\clearpage

\subsection{Domain}

\paragraph{Concept table}

\begin{table}[H]
  \begin{scriptsize}
    \begin{tabular}{llll}
      \cline{1-4}
      \thead{Concept} & \thead{Modeling type} & \thead{Inherit behavior from} & \thead{Instance of type} \\
      \cline{1-4}
      Domain          & \texttt{MDRng}        & MDRange                       & \texttt{dom}             \\
      SizedDomain     & \texttt{OMDRng}       & Domain                        & \texttt{sdom}            \\
      ShapedDomain    & \texttt{RMDRng}       & SizedDomain                   & \texttt{shdom}           \\
      \cline{1-4}
    \end{tabular}
    \smallskip

    \caption{Concepts Domain: definitions}
    \label{table:concept.domain.definitions}
  \end{scriptsize}
\end{table}


\begin{table}[H]
  \begin{scriptsize}
    \begin{tabular}{ll}
      \cline{1-2}
      \thead{Type}              & \thead{Instance of type} \\
      \cline{1-2}
      \texttt{Dom::value\_type} & \texttt{pnt}             \\
      \cline{1-2}
    \end{tabular}
    \smallskip

    \begin{tabular}{llll}
      \cline{1-4}
      \thead{Concept}                              & \thead{Expression}               & \thead{Return Type}          &
      \thead{Description}                                                                                              \\
      \cline{1-4}
      \multicolumn{1}{c|}{\multirow{4}{*}{Domain}} & \texttt{Point<Dom::value\_type>} & \texttt{std::true\_type}     &
      \makecell[l]{Domain's value models the \texttt{Point} concept}                                                   \\
      \multicolumn{1}{c|}{}                        & \texttt{dom.has(pnt)}            & \texttt{bool}                &
      \makecell[l]{Check if a points is included in the domain.}                                                       \\
      \multicolumn{1}{c|}{}                        & \texttt{dom.empty()}             & \texttt{void}                &
      \makecell[l]{Read the pixel's point.}                                                                            \\
      \multicolumn{1}{c|}{}                        & \texttt{dom.dim()}               & \texttt{void}                &
      Returns the domain's dimension.                                                                                  \\
      \cline{1-4}
      \multicolumn{1}{c|}{SizedDomain}             & \texttt{sdom.size()}             & \texttt{unsigned int}        &
      Returns the number of points inside the domain.                                                                  \\
      \cline{1-4}
      \multicolumn{1}{c|}{ShapedDomain}            & \texttt{shdom.extends()}         & \texttt{std::forward\_range} &
      \makecell[l]{Return a range that yields the number                                                               \\ of elements for each dimension.}
      \\
      \cline{1-4}
    \end{tabular}
    \smallskip

    \caption{Concepts Domain: expressions}
  \end{scriptsize}
  \label{table:concept.domain.expressions}
\end{table}

\paragraph{Concept code}

\begin{minted}{c++}
// Domain
template <typename Dom>
concept Domain =
  mln::ranges::mdrange<Dom> &&
  Point<mln::ranges::mdrange_value_t<Dom>> &&
  requires(const Dom cdom, mln::ranges::mdrange_value_t<Dom> p) {
    { cdom.has(p) }   -> std::same_as<bool>;
    { cdom.empty() }  -> std::same_as<bool>;
    { cdom.dim() }    -> std::same_as<int>;
  };


// SizedDomain
template <typename Dom>
concept SizedDomain =
  Domain<Dom> &&
  requires(const Dom cdom) {
    { cdom.size() } -> std::unsigned_integral;
  };

// ShapedDomain
template <typename Dom>
concept ShapedDomain =
  SizedDomain<Dom> &&
  requires(const Dom cdom) {
    { cdom.extents() }  -> std::ranges::forward_range;
  };
\end{minted}

\paragraph{Archetype code}

\begin{minted}{c++}
struct Domain
{
  using value_type = Point;
  using reference  = Point&;

  value_type* begin();
  value_type* end();

  bool has(value_type) const;
  bool empty() const;
  int  dim() const;
};

static_assert(mln::concepts::Domain<Domain>, "Domain archetype does not model the Domain concept!");

struct SizedDomain : Domain
{
  unsigned size() const;
};

static_assert(mln::concepts::SizedDomain<SizedDomain>,
                          "SizedDomain archetype does not model the SizedDomain concept!");

struct ShapedDomain final : SizedDomain
{
  static constexpr std::size_t  ndim = 1;
  value_type                    shape() const;
  std::array<std::size_t, ndim> extents() const;
};

static_assert(mln::concepts::ShapedDomain<ShapedDomain>,
                          "ShapedDomain archetype does not model the ShapedDomain concept!");
\end{minted}


\clearpage

\subsection{Image}

\paragraph{Concept table}

\begin{table}[H]
  \begin{scriptsize}
    \begin{tabular}{llll}
      \cline{1-4}
      \thead{Concept} & \thead{Modeling type} & \thead{Inherit behavior from} & \thead{Instance of type} \\
      \cline{1-4}
      \makecell[l]{Image (InputImage,                                                                    \\ ForwardImage)}    & \texttt{Img}          & $\emptyset$                                                                  & \texttt{img}             \\
      WritableImage   & \texttt{WImg}         & Image                         & \texttt{wimg}            \\
      \cline{1-4}
    \end{tabular}
    \smallskip

    \begin{tabular}{llll}
      \cline{1-4}
      \thead{Concept}                               & \thead{Definition}          & \thead{Description}                            & \thead{Requirement}                 \\
      \cline{1-4}
      \multicolumn{1}{c|}{\multirow{117}{*}{Image}} & \texttt{pixel\_type}        & Type of the image's pixel.                     & Models the concept \texttt{Pixel}.  \\
      \multicolumn{1}{c|}{}                         & \texttt{point\_type}        & Type of the image's point.                     & Models the concept \texttt{Point}.  \\
      \multicolumn{1}{c|}{}                         & \texttt{value\_type}        & \makecell[l]{Type of the image's value.                                              \\ Cannot be constant or reference} & Models the concept \texttt{Value}. \\
      \multicolumn{1}{c|}{}                         & \texttt{domain\_type}       & Type of the image's domain.                    & Models the concept \texttt{Domain}. \\
      \multicolumn{1}{c|}{}                         & \texttt{reference}          & \makecell[l]{Type used to mutate an image                                            \\ pixel's value if non-const} & \makecell[l]{Models the concept \\ \texttt{std::indirectly\_writable} \\ if non-const.}                                   \\
      \multicolumn{1}{c|}{}                         & \texttt{concrete\_type}     & Image concrete type (that holds data).         & Models the concept \texttt{Image}.  \\
      \multicolumn{1}{c|}{}                         & \texttt{ch\_value\_type<V>} & \makecell{Facility to return a new image type.                                       \\ that casts the underlying \texttt{value\_type} \\into \texttt{V}} & Models the concept \texttt{Image}. \\
      \cline{1-4}
    \end{tabular}
    \smallskip

    \caption{Concepts Image: definitions (1)}
    \label{table:concept.image.definitions.1}
  \end{scriptsize}
\end{table}

\begin{table}[H]
  \begin{scriptsize}
    \begin{tabular}{llll}
      \cline{1-4}
      \thead{Concept}                                     & \thead{Expression}                          & \thead{Return Type}                         &
      \thead{Description}                                                                                                                                                                           \\
      \cline{1-4}
      \multicolumn{1}{c|}{\multirow{7}{*}{Image}}         & \texttt{img.concretize()}                   & \makecell[l]{\texttt{std::convertible\_to<}                                               \\\texttt{concrete\_type>}} & \makecell[l]{Return a concrete image \\ that holds data.} \\
      \multicolumn{1}{c|}{}                               & \texttt{img.ch\_value<V>()}                 & \makecell[l]{\texttt{std::convertible\_to<}                                               \\\texttt{ch\_value\_type<V> >}} & \makecell[l]{Return an image whose values \\ are casted to \texttt{V}.}  \\
      \multicolumn{1}{c|}{}                               & \texttt{img.domain()}                       & \makecell[l]{\texttt{std::convertible\_to<}                                               \\\texttt{domain\_type>}} & Return the image's domain.  \\
      \multicolumn{1}{c|}{}                               & \texttt{img.pixels()}                       & \multirow{2}{*}{\texttt{MDRange}}           & \makecell[l]{Return a range that yields     \\ all the image pixels.} \\
      \multicolumn{1}{c|}{}                               & \texttt{img.values()}                       &                                             & \makecell[l]{Return a range that yields     \\ all the image values.} \\
      \multicolumn{1}{c|}{}                               & \makecell[l]{\texttt{std::convertible\_to<}                                                                                             \\\texttt{std::ranges::ranges\_value\_t<} \\\texttt{decltype(img.pixels())>,} \\\texttt{pixel\_type>}} & \multirow{2}{*}{\texttt{std::true\_type}}     & \multirow{2}{*}{\makecell[l]{Ranges converts to compatible\\ element types.}} \\
      \multicolumn{1}{c|}{}                               & \makecell[l]{\texttt{std::convertible\_to<}                                                                                             \\\texttt{std::ranges::ranges\_value\_t<} \\\texttt{decltype(img.values())>,} \\\texttt{value\_type>}} &      &  \\
      \cline{1-4}
      \multicolumn{1}{c|}{\multirow{2}{*}{WritableImage}} & \texttt{wimg.values()}                      & \texttt{OutputMDRange}                      & \makecell[l]{Return a range that yields all \\ the image values (mutable).} \\
      \multicolumn{1}{c|}{}                               & \makecell[l]{\texttt{OutputPixel<}                                                                                                      \\\texttt{std::ranges::ranges\_value\_t<} \\\texttt{decltype(wimg.pixels())> >}} & \texttt{std::true\_type}     & \multirow{2}{*}{Ranges whose elements are mutable.} \\
      \cline{1-4}
    \end{tabular}
    \smallskip

    \caption{Concepts Image: expressions (1)}
  \end{scriptsize}
  \label{table:concept.image.expressions.1}
\end{table}

\paragraph{Concept code}

\begin{minted}{c++}
template <typename I>
concept Image =
  // Minimum constraint on image object
  // Do not requires DefaultConstructible
  std::is_base_of_v<mln::details::Image<I>, I> &&
  std::copy_constructible<I> &&
  std::move_constructible<I> &&
  std::derived_from<image_category_t<I>, forward_image_tag> &&
  requires {
    typename image_pixel_t<I>;
    typename image_point_t<I>;
    typename image_value_t<I>;
    typename image_domain_t<I>;
    typename image_reference_t<I>;
    typename image_concrete_t<I>;
    typename image_ch_value_t<I, mln::archetypes::Value>;
    // traits
    typename image_indexable_t<I>;
    typename image_accessible_t<I>;
    typename image_extension_category_t<I>;
    typename image_category_t<I>;
    typename image_view_t<I>;
  } &&
  Pixel<image_pixel_t<I>> &&
  Point<image_point_t<I>> &&
  Value<image_value_t<I>> &&
  Domain<image_domain_t<I>> &&
  std::convertible_to<pixel_point_t<image_pixel_t<I>>, image_point_t<I>> &&
  std::convertible_to<pixel_reference_t<image_pixel_t<I>>, image_reference_t<I>> &&
  // Here we don't want a convertible constraint as value_type is the decayed type and should really be the same
  std::same_as<pixel_value_t<image_pixel_t<I>>, image_value_t<I>> &&
  std::common_reference_with<image_reference_t<I>&&, image_value_t<I>&> &&
  std::common_reference_with<image_reference_t<I>&&, image_value_t<I>&&> &&
  std::common_reference_with<image_value_t<I>&&, const image_value_t<I>&> &&
  requires(I ima, const I cima, image_domain_t<I> d, image_point_t<I> p) {
    { cima.template ch_value<mln::archetypes::Value>() }
        -> std::convertible_to<image_ch_value_t<I, mln::archetypes::Value>>;
    { cima.concretize() } -> std::convertible_to<image_concrete_t<I>>;
    { cima.domain() }     -> std::convertible_to<image_domain_t<I>>;
    { ima.pixels() }  -> mln::ranges::mdrange;
    { ima.values() }  -> mln::ranges::mdrange;
    requires std::convertible_to<mln::ranges::mdrange_value_t<decltype(ima.pixels())>, image_pixel_t<I>>;
    requires std::convertible_to<mln::ranges::mdrange_value_t<decltype(ima.values())>, image_value_t<I>>;
  };

namespace detail
{
  // WritableImage
  template <typename I>
  concept WritableImage =
    Image<I> &&
    OutputPixel<image_pixel_t<I>> &&
    requires(I ima) {
    { ima.values() }  -> mln::ranges::output_mdrange<image_value_t<I>>;
      // Check Writability of each pixel of the range
      requires OutputPixel<
                  std::common_type_t<
                    mln::ranges::mdrange_value_t<decltype(ima.pixels())>,
                    image_pixel_t<I>>>;
    };
} // namespace detail


// InputImage
template <typename I>
concept InputImage = Image<I>;


// ForwardImage
template <typename I>
concept ForwardImage = InputImage<I>;
\end{minted}

\paragraph{Archetype code}

\begin{minted}{c++}
namespace details
{
  template <class I>
  struct AsImage : I, mln::details::Image<AsImage<I>>
  {
    using I::I;

    using concrete_type = AsImage<typename I::concrete_type>;
    concrete_type concretize() const;


    template <typename V>
    using ch_value_type = AsImage<typename I::template ch_value_type<V>>;

    template <typename V>
    ch_value_type<V> ch_value() const;
  };


  struct Image
  {
    using pixel_type = archetypes::Pixel;
    using value_type     = pixel_value_t<mln::archetypes::Pixel>;
    using reference      = pixel_reference_t<mln::archetypes::Pixel>;
    using point_type     =  std::ranges::range_value_t<Domain>;
    using domain_type    = Domain;
    using category_type  = forward_image_tag;
    using concrete_type  = Image;

    template <class V>
    using ch_value_type = Image;

    // additional traits
    using extension_category = mln::extension::none_extension_tag;
    using indexable          = std::false_type;
    using accessible         = std::false_type;
    using view               = std::false_type;

    Image()                     = default;
    Image(const Image&) = default;
    Image(Image&&)      = default;
    Image& operator=(const Image&) = default;
    Image& operator=(Image&&) = default;

    domain_type domain() const;


    struct pixel_range
    {
      const pixel_type* begin();
      const pixel_type* end();
    };
    pixel_range pixels();


    struct value_range
    {
      const value_type* begin();
      const value_type* end();
    };

    value_range values();
  };

  struct OutputImage : Image
  {
    using pixel_type = archetypes::OutputPixel;
    using reference  = pixel_reference_t<mln::archetypes::OutputPixel>;

    struct pixel_range
    {
      const pixel_type* begin();
      const pixel_type* end();
    };

    pixel_range pixels();

    struct value_range
    {
      value_type* begin();
      value_type* end();
    };

    value_range values();
  };
} // namespace details


using Image         = details::AsImage<details::Image>;
using ForwardImage  = Image;
using WritableImage = details::AsImage<details::OutputImage>;
\end{minted}


\clearpage

\section{Advanced way to access image data}


\subsection{Index}

\paragraph{Concept table}

\begin{table}[H]
  \begin{scriptsize}
    \begin{tabular}{llll}
      \cline{1-4}
      \thead{Concept} & \thead{Modeling type} & \thead{Inherit behavior from} & \thead{Instance of type} \\
      \cline{1-4}
      Index           & \texttt{Idx}          & $\emptyset$                   & \texttt{idx, idy}        \\
      \cline{1-4}
    \end{tabular}
    \smallskip

    \begin{tabular}{llll}
      \cline{1-4}
      \thead{Concept}                             & \thead{Expression}                   & \thead{Return Type}      &
      \thead{Description}                                                                                             \\
      \cline{1-4}
      \multicolumn{1}{c|}{\multirow{2}{*}{Index}} & \texttt{std::signed\_integral<Idx>}  & \texttt{std::true\_type} &
      Idx is a signed integral arithmetic type                                                                        \\
      \multicolumn{1}{c|}{}                       & \texttt{idx + idy, idx - idy, \dots} & \texttt{Idx}             &
      Supports all trivial arithmetical operations                                                                    \\
      \cline{1-4}
    \end{tabular}
    \smallskip

    \caption{Concepts Index: expressions}
  \end{scriptsize}
  \label{table:concept.index.expressions}
\end{table}

\paragraph{Concept code}

\begin{minted}{c++}
// Index
template <typename Idx>
concept Index = std::signed_integral<Idx>;
\end{minted}

\paragraph{Archetype code}

\begin{minted}{c++}
using Index = int;

static_assert(mln::concepts::Index<Index>, "Index archetype does not model the Index concept!");
\end{minted}


\clearpage

\subsection{Indexable image}

\paragraph{Concept table}

\begin{table}[H]
  \begin{scriptsize}
    \begin{tabular}{llll}
      \cline{1-4}
      \thead{Concept}        & \thead{Modeling type} & \thead{Inherit behavior from} & \thead{Instance of type} \\
      \cline{1-4}
      IndexableImage         & \texttt{IdxImg}       & Image                         & \texttt{idximg}          \\
      WritableIndexableImage & \texttt{WIdxImg}      & IndexableImage, WritableImage & \texttt{widximg}         \\
    \end{tabular}
    \smallskip

    \begin{tabular}{llll}
      \cline{1-4}
      \thead{Concept}                     & \thead{Definition}   & \thead{Description}               & \thead{Requirement}                \\
      \cline{1-4}
      \multicolumn{1}{c|}{IndexableImage} & \texttt{index\_type} & Type of the image's buffer index. & Models the concept \texttt{Index}. \\
      \cline{1-4}
    \end{tabular}
    \smallskip

    \caption{Concepts Image: definitions (2)}
    \label{table:concept.image.definitions.2}
  \end{scriptsize}
\end{table}

\begin{table}[H]
  \begin{scriptsize}
    \begin{tabular}{lll}
      \cline{1-2}
      \thead{Type}                 & \thead{Instance of type} \\
      \cline{1-2}
      \texttt{Img::value\_type}    & \texttt{val}             \\
      \texttt{IdxImg::index\_type} & \texttt{k}               \\
      \cline{1-2}
    \end{tabular}
    \smallskip

    \begin{tabular}{llll}
      \cline{1-4}
      \thead{Concept}                     & \thead{Expression} & \thead{Return Type}               &
      \thead{Description}                                                                                                                           \\
      \cline{1-4}
      \multicolumn{1}{c|}{IndexableImage} & \texttt{idximg[k]} & \texttt{std::same\_as<reference>} & \makecell[l]{Access a value at a given index.} \\
      \cline{1-4}
      \multicolumn{1}{c|}{\makecell[l]{Writable                                                                                                     \\IndexableImage}} & \texttt{widximg[k] = val}                            & \texttt{void}                      & \makecell[l]{Mutate a value at a given index.} \\
      \cline{1-4}
    \end{tabular}
    \smallskip

    \caption{Concepts Image: expressions (2)}
  \end{scriptsize}
  \label{table:concept.image.expressions.2}
\end{table}

\paragraph{Concept code}

\begin{minted}{c++}
// IndexableImage
template <typename I>
concept IndexableImage =
  Image<I> &&
  requires {
    typename image_index_t<I>;
  } &&
  image_indexable_v<I> &&
  requires (I ima, image_index_t<I> k) {
    { ima[k] }  -> std::same_as<image_reference_t<I>>; // For concrete image it returns a const_reference
  };


namespace detail
{
  // WritableIndexableImage
  template <typename I>
  concept WritableIndexableImage =
    WritableImage<I> &&
    IndexableImage<I> &&
    requires(I ima, image_index_t<I> k, image_value_t<I> v) {
      { ima[k] = v }
    };
} // namespace detail
\end{minted}

\paragraph{Archetype code}

\begin{minted}{c++}
namespace details
{
    
  struct IndexableImage : Image
  {
    using index_type = int;
    using indexable  = std::true_type;

    using concrete_type = IndexableImage;

    template <class V>
    using ch_value_type = IndexableImage;


    reference operator[](index_type);
  };

  struct OutputIndexableImage : OutputImage
  {
    using index_type = int;
    using indexable  = std::true_type;

    using concrete_type = OutputIndexableImage;

    template <class V>
    using ch_value_type = OutputIndexableImage;

    reference operator[](index_type);
  };

} // namespace details

using IndexableImage              = details::AsImage<details::IndexableImage>;
using WritableIndexableImage      = details::AsImage<details::OutputIndexableImage>;
\end{minted}


\clearpage

\subsection{Accessible image}

\paragraph{Concept table}

\begin{table}[H]
  \begin{scriptsize}
    \begin{tabular}{llll}
      \cline{1-4}
      \thead{Concept}         & \thead{Modeling type} & \thead{Inherit behavior from}  & \thead{Instance of type} \\
      \cline{1-4}
      AccessibleImage         & \texttt{AccImg}       & Image                          & \texttt{accimg}          \\
      WritableAccessibleImage & \texttt{WAccImg}      & AccessibleImage, WritableImage & \texttt{waccimg}         \\
      \cline{1-4}
    \end{tabular}
    \smallskip

    \caption{Concepts Image: definitions (3)}
    \label{table:concept.image.definitions.3}
  \end{scriptsize}
\end{table}

\begin{table}[H]
  \begin{scriptsize}
    \begin{tabular}{lll}
      \cline{1-2}
      \thead{Type}              & \thead{Instance of type} \\
      \cline{1-2}
      \texttt{Img::point\_type} & \texttt{pnt}             \\
      \cline{1-2}
    \end{tabular}
    \smallskip

    \begin{tabular}{llll}
      \cline{1-4}
      \thead{Concept}                                       & \thead{Expression}                          & \thead{Return Type}                 &
      \thead{Description}                                                                                                                                                                         \\
      \cline{1-4}
      \multicolumn{1}{c|}{\multirow{4}{*}{AccessibleImage}} & \texttt{accimg(pnt)}                        & \texttt{std::same\_as<reference>}   & \makecell[l]{Access a value for a given point.} \\
      \multicolumn{1}{c|}{}                                 & \texttt{accimg.at(pnt)}                     & \texttt{std::same\_as<reference>}   & \makecell[l]{Access a value for a given point.  \\ No bound checking.} \\
      \multicolumn{1}{c|}{}                                 & \texttt{accimg.pixel(pnt)}                  & \texttt{std::same\_as<pixel\_type>} & \makecell[l]{Access a pixel for a given point.} \\
      \multicolumn{1}{c|}{}                                 & \texttt{accimg.pixel\_at(pnt)}              & \texttt{std::same\_as<pixel\_type>} & \makecell[l]{Access a pixel for a given point.  \\ No bound checking.} \\
      \cline{1-4}
      \multicolumn{1}{c|}{\multirow{4}{*}{\makecell[l]{Writable                                                                                                                                   \\AccessibleImage}}} & \texttt{img(pnt) = val}                            & \texttt{void}                      & \makecell[l]{Mutate a value at a given point.} \\
      \multicolumn{1}{c|}{}                                 & \texttt{waccimg.at(pnt) = val}              & \texttt{void}                       & \makecell[l]{Mutate a value at a given point.   \\ No bound checking.} \\
      \multicolumn{1}{c|}{}                                 & \makecell[l]{\texttt{OutputPixel<decltype(}                                                                                         \\\texttt{waccimg.pixel(pnt))>}}                            & \texttt{std::true\_type}                      & \makecell[l]{The returned pixel \\ models \texttt{OutputPixel}.} \\
      \multicolumn{1}{c|}{}                                 & \makecell[l]{\texttt{OutputPixel<decltype(}                                                                                         \\\texttt{waccimg.pixel\_at(pnt))>}}                            & \texttt{std::true\_type}                      & \makecell[l]{The returned pixel \\ models \texttt{OutputPixel}. \\ No bound checking.} \\
      \cline{1-4}
    \end{tabular}
    \smallskip

    \caption{Concepts Image: expressions (3)}
  \end{scriptsize}
  \label{table:concept.image.expressions.3}
\end{table}

\paragraph{Concept code}

\begin{minted}{C++}
// AccessibleImage
template <typename I>
concept AccessibleImage =
  Image<I> &&
  image_accessible_v<I> &&
  requires (I ima, image_point_t<I> p) {
    { ima(p) }              -> std::same_as<image_reference_t<I>>; // For concrete image it returns a const_reference
    { ima.at(p) }           -> std::same_as<image_reference_t<I>>; // idem
    { ima.pixel(p) }    -> std::same_as<image_pixel_t<I>>; // For concrete image pixel may propagate constness
    { ima.pixel_at(p) } -> std::same_as<image_pixel_t<I>>; // idem
  };


namespace detail
{
  // WritableAccessibleImage
  template <typename I>
  concept WritableAccessibleImage =
    detail::WritableImage<I> &&
    AccessibleImage<I> &&
    requires(I ima, image_point_t<I> p, image_value_t<I> v) {
      { ima(p) = v };
      { ima.at(p) = v };

      requires OutputPixel<decltype(ima.pixel(p))>;
      requires OutputPixel<decltype(ima.pixel_at(p))>;
    };
} // namespace detail
\end{minted}

\paragraph{Archetype code}
\begin{minted}{C++}
namespace details {  
  struct OutputAccessibleImage : OutputImage
  {
    using accessible    = std::true_type;
    using concrete_type = OutputAccessibleImage;

    template <class V>
    using ch_value_type = OutputAccessibleImage;


    reference      operator()(point_type);
    reference      at(point_type);
    pixel_type pixel(point_type);
    pixel_type pixel_at(point_type);
  };


  struct AccessibleImage : Image
  {
    using accessible    = std::true_type;
    using concrete_type = AccessibleImage;

    template <class V>
    using ch_value_type = AccessibleImage;

    reference      operator()(point_type);
    reference      at(point_type);
    pixel_type pixel(point_type);
    pixel_type pixel_at(point_type);
  };
} // namespace details

using AccessibleImage             = details::AsImage<details::AccessibleImage>;
using WritableAccessibleImage     = details::AsImage<details::OutputAccessibleImage>;
\end{minted}


\clearpage

\subsection{Indexable and accessible image}

\paragraph{Concept table}

\begin{table}[H]
  \begin{scriptsize}
    \begin{tabular}{llll}
      \cline{1-4}
      \thead{Concept} & \thead{Modeling type} & \thead{Inherit behavior from} & \thead{Instance of type} \\
      \makecell[l]{IndexableAnd                                                                          \\ AccessibleImage}         & \texttt{IdxAccImg}    & IndexableImage, AccessibleImage                                              & \texttt{idxaccimg}       \\
      \makecell[l]{ WritableIndexable                                                                    \\ AndAccessibleImage} & \texttt{WIdxAccImg}   & \makecell[l]{IndexableAndAccessibleImage, \\ WritableIndexableImage, \\WritableAccessibleImage} & \texttt{widxaccimg}      \\
      \cline{1-4}
    \end{tabular}
    \smallskip

    \caption{Concepts Image: definitions (4)}
    \label{table:concept.image.definitions.4}
  \end{scriptsize}
\end{table}

\begin{table}[H]
  \begin{scriptsize}
    \begin{tabular}{llll}
      \cline{1-4}
      \thead{Concept}       & \thead{Expression}                       & \thead{Return Type}  &
      \thead{Description}                                                                                                                    \\
      \cline{1-4}
      \multicolumn{1}{c|}{\multirow{3}{*}{\makecell[l]{IndexableAnd                                                                          \\AccessibleImage}}} & \texttt{img.point\_at\_index(k)}                            & \texttt{point\_type}                      & \makecell[l]{Get the point corresponding\\ to the given index.} \\
      \multicolumn{1}{c|}{} & \texttt{idxaccimg.index\_of\_point(pnt)} & \texttt{index\_type} & \makecell[l]{Get the linear index (offset in \\ the buffer) of multi-dimensional point.} \\
      \multicolumn{1}{c|}{} & \texttt{idxaccimg.delta\_index(pnt)}     & \texttt{index\_type} & \makecell[l]{Get the linear index offset     \\ for the given point.} \\
      \cline{1-4}
    \end{tabular}
    \smallskip

    \caption{Concepts Image: expressions (4)}
  \end{scriptsize}
  \label{table:concept.image.expressions.4}
\end{table}

\paragraph{Concept code}

\begin{minted}{C++}
// IndexableAndAccessibleImage
template <typename I>
concept IndexableAndAccessibleImage =
  IndexableImage<I> &&
  AccessibleImage<I> &&
  requires (const I cima, image_index_t<I> k, image_point_t<I> p) {
    { cima.point_at_index(k) }  -> std::same_as<image_point_t<I>>;
    { cima.index_of_point(p) }  -> std::same_as<image_index_t<I>>;
    { cima.delta_index(p) }     -> std::same_as<image_index_t<I>>;
  };


namespace detail
{
  // WritableIndexableAndAccessibleImage
  template <typename I>
  concept WritableIndexableAndAccessibleImage =
    IndexableAndAccessibleImage<I> &&
    detail::WritableImage<I> &&
    detail::WritableIndexableImage<I>;
} // namespace detail
\end{minted}

\paragraph{Archetype code}

\begin{minted}{C++}
namespace details {
  struct OutputIndexableAndAccessibleImage : OutputAccessibleImage
  {
    using index_type = int;
    using indexable  = std::true_type;

    using concrete_type = OutputIndexableAndAccessibleImage;

    template <class V>
    using ch_value_type = OutputIndexableAndAccessibleImage;


    reference  operator[](index_type);
    point_type point_at_index(index_type) const;
    index_type index_of_point(point_type) const;
    index_type delta_index(point_type) const;
  };

  struct IndexableAndAccessibleImage : AccessibleImage
  {
    using index_type = int;
    using indexable  = std::true_type;

    using concrete_type = IndexableAndAccessibleImage;

    template <class V>
    using ch_value_type = IndexableAndAccessibleImage;

    reference  operator[](index_type);
    point_type point_at_index(index_type) const;
    index_type index_of_point(point_type) const;
    index_type delta_index(point_type) const;
  };
} // namespace details

using IndexableAndAccessibleImage       = details::AsImage<details::IndexableAndAccessibleImage>;
using WritableIndexableAndAccessibleImage = details::AsImage<details::OutputIndexableAndAccessibleImage>;
\end{minted}


\clearpage

\subsection{Bidirectional image}

\paragraph{Concept table}

\begin{table}[H]
  \begin{scriptsize}
    \begin{tabular}{llll}
      \cline{1-4}
      \thead{Concept}            & \thead{Modeling type} & \thead{Inherit behavior from}     & \thead{Instance of type} \\
      \cline{1-4}
      BidirectionalImage         & \texttt{BidirImg}     & Image                             & \texttt{bidirimg}        \\
      WritableBidirectionalImage & \texttt{WBidirImg}    & BidirectionalImage, WritableImage & \texttt{wbidirimg}       \\
      \cline{1-4}
    \end{tabular}
    \smallskip

    \caption{Concepts Image: definitions (5)}
    \label{table:concept.image.definitions.5}
  \end{scriptsize}
\end{table}

\begin{table}[H]
  \begin{scriptsize}
    \begin{tabular}{lll}
      \cline{1-2}
      \thead{Type}                 & \thead{Instance of type} \\
      \cline{1-2}
      \texttt{Img::point\_type}    & \texttt{pnt}             \\
      \texttt{Img::value\_type}    & \texttt{val}             \\
      \texttt{IdxImg::index\_type} & \texttt{k}               \\
      \texttt{int}                 & \texttt{dim}             \\
      \cline{1-2}
    \end{tabular}
    \smallskip

    \begin{tabular}{llll}
      \cline{1-4}
      \thead{Concept}                                          & \thead{Expression}         & \thead{Return Type}                         &
      \thead{Description}                                                                                                                                                                      \\
      \cline{1-4}
      \multicolumn{1}{c|}{\multirow{2}{*}{BidirectionalImage}} & \texttt{bidirimg.pixels()} & \multirow{2}{*}{\texttt{ReversibleMDRange}} & \makecell[l]{Return a reversible range that yields \\ all the image pixels.} \\
      \multicolumn{1}{c|}{}                                    & \texttt{bidirimg.values()} &                                             & \makecell[l]{Return a reversible range that yields \\ all the image values.} \\
      \cline{1-4}
      \cline{1-4}
    \end{tabular}
    \smallskip

    \caption{Concepts Image: expressions (5)}
  \end{scriptsize}
  \label{table:concept.image.expressions.5}
\end{table}

\paragraph{Concept code}
\begin{minted}{C++}
  namespace detail
  {
    // WritableBidirectionalImage
    template <typename I>
    concept WritableBidirectionalImage =
      WritableImage<I> &&
      BidirectionalImage<I>;
  } // namespace detail
  
  // RawImage (not contiguous, stride = padding)
  template <typename I>
  concept RawImage =
    IndexableAndAccessibleImage<I> &&
    BidirectionalImage<I> &&
    std::derived_from<image_category_t<I>, raw_image_tag> &&
    requires (I ima, const I cima, int dim) {
      { ima.data() }        -> std::convertible_to<const image_value_t<I>*>; // data() may be proxied by a view
      { cima.stride(dim) } -> std::same_as<std::ptrdiff_t>;
    };
\end{minted}

\paragraph{Archetype code}

\begin{minted}{C++}
namespace details {
  struct BidirectionalImage : Image
  {
    using category_type = bidirectional_image_tag;

    struct pixel_range
    {
      const pixel_type* begin();
      const pixel_type* end();
      pixel_range           reversed();
    };

    pixel_range pixels();

    struct value_range
    {
      const value_type* begin();
      const value_type* end();
      value_range       reversed();
    };

    value_range values();
  };

  struct OutputBidirectionalImage : BidirectionalImage
  {
    using pixel_type = archetypes::OutputPixel;
    using reference      = pixel_reference_t<mln::archetypes::OutputPixel>;

    struct value_range
    {
      value_type* begin();
      value_type* end();
      value_range reversed();
    };
    value_range values();


    struct pixel_range
    {
      const pixel_type* begin();
      const pixel_type* end();
      pixel_range           reversed();
    };

    pixel_range pixels();
  };
} // namespace details

using BidirectionalImage = details::AsImage<details::BidirectionalImage>;
using OutputBidirectionalImage = details::AsImage<details::OutputBidirectionalImage>;
\end{minted}


\clearpage

\subsection{Raw image}

\paragraph{Concept table}

\begin{table}[H]
  \begin{scriptsize}
    \begin{tabular}{llll}
      \cline{1-4}
      \thead{Concept}  & \thead{Modeling type} & \thead{Inherit behavior from}                  & \thead{Instance of type} \\
      \cline{1-4}
      RawImage         & \texttt{RawImg}       & \makecell[l]{IndexableAndAccessibleImage,                                 \\BidirectionalImage}                              & \texttt{rawimg}          \\
      WritableRawImage & \texttt{WRawImg}      & \makecell[l]{RawImage, WritableIndexableImage,                            \\WritableBidirectionalImage}                 & \texttt{wrawimg}         \\
      \cline{1-4}
    \end{tabular}
    \smallskip

    \caption{Concepts Image: definitions (6)}
    \label{table:concept.image.definitions.6}
  \end{scriptsize}
\end{table}

\begin{table}[H]
  \begin{scriptsize}
    \begin{tabular}{llll}
      \cline{1-4}
      \thead{Concept}                                & \thead{Expression}                   & \thead{Return Type}                         &
      \thead{Description}                                                                                                                                                                       \\
      \cline{1-4}
      \multicolumn{1}{c|}{\multirow{2}{*}{RawImage}} & \texttt{rawimg.data()}               & \makecell[l]{\texttt{std::convertible\_to<}                                                       \\\texttt{const value\_type*>}} & \makecell[l]{Get a constant pointer to \\ the first element of the domain.} \\
      \multicolumn{1}{c|}{}                          & \texttt{rawimg.stride(dim)}          & \texttt{std::ptrdiff\_t}                    & \makecell[l]{Get the stride (in number of elements) \\ between two consecutive elements \\in the given \texttt{dim}.} \\
      \cline{1-4}
      \multicolumn{1}{c|}{\multirow{2}{*}{\makecell[l]{Writable                                                                                                                                 \\RawImage}}}           & \texttt{img.data()}                         & \makecell[l]{\texttt{std::convertible\_to<}\\\texttt{value\_type*>}} & \makecell[l]{Get a pointer to the first\\ element of the domain.} \\
      \multicolumn{1}{c|}{}                          & \texttt{*(wrawimg.data() + k) = val} & \texttt{void}                               & \makecell[l]{Access an element from the data        \\buffer at index \texttt{k} and mutate it \\ to \texttt{val}.} \\
      \cline{1-4}
      \multicolumn{1}{c|}{\makecell[l]{WithExtension                                                                                                                                            \\Image}}           & \texttt{wextimg.extension()}                         & \makecell[l]{\texttt{std::convertible\_to<}\\\texttt{extension\_type>}} & \makecell[l]{Get the extension of the image.} \\
      \cline{1-4}
    \end{tabular}
    \smallskip

    \caption{Concepts Image: expressions (6)}
  \end{scriptsize}
  \label{table:concept.image.expressions.6}
\end{table}

\paragraph{Concept code}

\begin{minted}{C++}
// RawImage (not contiguous, stride = padding)
template <typename I>
concept RawImage =
  IndexableAndAccessibleImage<I> &&
  BidirectionalImage<I> &&
  std::derived_from<image_category_t<I>, raw_image_tag> &&
  requires (I ima, const I cima, int dim) {
    { ima.data() }        -> std::convertible_to<const image_value_t<I>*>; // data() may be proxied by a view
    { cima.stride(dim) } -> std::same_as<std::ptrdiff_t>;
  };


namespace detail
{
  // WritableRawImage
  template <typename I>
  concept WritableRawImage =
    WritableIndexableAndAccessibleImage<I> &&
    WritableBidirectionalImage<I> &&
    RawImage<I> &&
    requires(I ima, image_value_t<I> v, image_index_t<I> k) {
      { ima.data() }        -> ::concepts::convertible_to<image_value_t<I>*>;
      { *(ima.data() + k) = v };
    };
} // namespace detail
\end{minted}


\paragraph{Archetype code}

\begin{minted}{C++}
namespace details {
  struct RawImage : IndexableAndAccessibleImage
  {
    using category_type = raw_image_tag;
    using pixel_range   = BidirectionalImage::pixel_range;
    using value_range   = BidirectionalImage::value_range;

    pixel_range pixels();
    value_range values();


    const value_type* data() const;
    std::ptrdiff_t    strides(int) const;
  };

  struct OutputRawImage : OutputIndexableAndAccessibleImage
  {
    using category_type = raw_image_tag;
    using pixel_range   = OutputBidirectionalImage::pixel_range;
    using value_range   = OutputBidirectionalImage::value_range;

    pixel_range pixels();
    value_range values();

    value_type*    data() const;
    std::ptrdiff_t strides(int) const;
  };
} // namespace details

using RawImage           = details::AsImage<details::RawImage>;
using OutputRawImage     = details::AsImage<details::OutputRawImage>;
\end{minted}


\clearpage

\section{Local algorithm concepts: structuring elements and extensions}


\subsection{Structuring element}

\paragraph{Concept table}

\begin{table}[H]
  \begin{scriptsize}
    \begin{tabular}{llll}
      \cline{1-4}
      \thead{Concept}                & \thead{Modeling type} & \thead{Inherit behavior from} & \thead{Instance of type} \\
      \cline{1-4}
      StructuringElement             & \texttt{SE, Pnt}      & $\emptyset$, Point            & \texttt{se, pnt}         \\
      DecomposableStructuringElement & \texttt{DSE, Pnt}     & StructuringElement, Point     & \texttt{dse}             \\
      SeparableStructuringElement    & \texttt{SSE, Pnt}     & StructuringElement, Point     & \texttt{sse}             \\
      IncrementalStructuringElement  & \texttt{ISE, Pnt}     & StructuringElement, Point     & \texttt{ise}             \\
      \cline{1-4}
    \end{tabular}
    \smallskip

    \begin{tabular}{llll}
      \cline{1-4}
      \thead{Concept}                                          & \thead{Definition}    & \thead{Description}                           &
      \thead{Requirement}                                                                                                                                                                                   \\
      \cline{1-4}
      \multicolumn{1}{c|}{\multirow{3}{*}{StructuringElement}} & \texttt{incremental}  & \multirow{3}{*}{\texttt{std::bool\_constant}} & \multirow{3}{*}{\makecell[l]{\texttt{std::true\_type} if supported \\ \texttt{std::false\_false} otherwise.}} \\
      \multicolumn{1}{c|}{}                                    & \texttt{decomposable} &                                               &                                                                    \\
      \multicolumn{1}{c|}{}                                    & \texttt{separable}    &                                               &                                                                    \\
      \cline{1-4}
    \end{tabular}
    \smallskip

    \caption{Concepts Structuring Elements: definitions}
    \label{table:concept.se.definitions}
  \end{scriptsize}
\end{table}

\begin{table}[H]
  \begin{scriptsize}
    \begin{tabular}{lll}
      \cline{1-3}
      \thead{Type} & \thead{Instance of type} & \thead{Requirement}                           \\
      \cline{1-3}
      \texttt{Pix} & \texttt{pix}             & \texttt{std::same\_as<Pix::point\_type, Pnt>} \\
      \texttt{Pnt} & \texttt{pnt}             &                                               \\
      \cline{1-3}
    \end{tabular}
    \smallskip

    \begin{tabular}{llll}
      \cline{1-4}
      \thead{Concept}                                           & \thead{Expression}                                    & \thead{Return Type}                       &
      \thead{Description}                                                                                                                                                                                     \\
      \cline{1-4}
      \multicolumn{1}{c|}{\multirow{11}{*}{StructuringElement}} & \multirow{2}{*}{\texttt{std::regular\_invocable<SE,
      Pnt>}}                                                    & \multirow{4}{*}{\texttt{std::true\_type}}             &
      \multirow{4}{*}{\makecell[l]{\texttt{se} can called using \text{std::invoke},                                                                                                                           \\ is equality preserving and \\
      does not modify function object                                                                                                                                                                         \\ nor arguments.}} \\
      \multicolumn{1}{c|}{}                                     &                                                       &
                                                                &                                                                                                                                             \\
      \multicolumn{1}{c|}{}                                     & \multirow{2}{*}{\texttt{std::regular\_invocable<SE,
      Pix>}}                                                    &                                                       &
      \\
      \multicolumn{1}{c|}{}                                     &                                                       &                                           &                                         \\
      \multicolumn{1}{c|}{}                                     & \texttt{se(pnt)}                                      & \texttt{std::forward\_range}              &
      \makecell[l]{Return a range that yeilds the                                                                                                                                                             \\
      neighboring points of \texttt{pnt}.}                                                                                                                                                                    \\
      \multicolumn{1}{c|}{}                                     & \texttt{se.offsets()}                                 & \texttt{std::forward\_range}              &
      \makecell[l]{Return a range that yeilds the                                                                                                                                                             \\
      neighboring points,                                                                                                                                                                                     \\in relative coordinates.}                                                                                                                                                          \\
      \multicolumn{1}{c|}{}                                     & \texttt{se(pix)}                                      & \texttt{std::forward\_range}              &
      \makecell[l]{Return a range that yeilds the                                                                                                                                                             \\
      neighboring pixels of \texttt{pix}.}                                                                                                                                                                    \\
      \multicolumn{1}{c|}{}                                     & \texttt{se.radial\_extent()}                          & \texttt{int}                              &
      \makecell[l]{Returns the radial extent                                                                                                                                                                  \\of the SE,
      the radius                                                                                                                                                                                              \\of the disc (square).}                                                                                                                                                                      \\
      \multicolumn{1}{c|}{}                                     & \makecell[l]{\texttt{std::convertible\_to<}                                                                                                 \\
      \texttt{std::ranges::range\_value\_t<}                                                                                                                                                                  \\ \texttt{decltype(se(pnt))>, Pnt>}}      &
      \multirow{3}{*}{\texttt{std::true\_type}}                 & \multirow{3}{*}{\makecell[l]{Converts to a compatible                                                                                       \\ point type}}
      \\
      \multicolumn{1}{c|}{}                                     & \makecell[l]{\texttt{std::convertible\_to<}                                                                                                 \\
      \texttt{std::ranges::range\_value\_t<}                                                                                                                                                                  \\ \texttt{decltype(se.offsets())>, Pnt>}} & & \\
      \multicolumn{1}{c|}{}                                     &
      \makecell[l]{\texttt{Pixel<std::ranges::range\_value\_t<}
      \\ \texttt{decltype(se(pix))> >}}                          &                                           & \makecell[l]{Is a range of compatible \\pixel type}                                                                                                                                                                \\
      \cline{1-4}
      \multicolumn{1}{c|}{\multirow{4}{*}{\makecell[l]{Decomposable                                                                                                                                           \\StructuringElement}}} & \texttt{dse.is\_decomposable()} & \texttt{std::bool\_constant} & \makecell[l]{\texttt{std::true\_type} if supported\\ \texttt{std::false\_false} otherwise.}\\
      \multicolumn{1}{c|}{}                                     & \texttt{dse.decompose()}                              & \texttt{std::forward\_range}              & \makecell[l]{Return a range that yeilds \\ simpler structuring elements.}\\
      \multicolumn{1}{c|}{}                                     & \makecell[l]{\texttt{StructuringElement<}                                                                                                   \\ \texttt{std::ranges::range\_value\_t<} \\ \texttt{decltype(dse.decompose())> >}}                            & \texttt{std::true\_type} & \makecell[l]{Is a range of compatible \\structuring elements types.}\\
      \multicolumn{1}{c|}{}                                     & \texttt{dse.is\_decomposable()}                       & \texttt{bool}                             & \makecell[l]{Wether the decompose       \\ facility is supported}\\
      \cline{1-4}
      \multicolumn{1}{c|}{\multirow{4}{*}{\makecell[l]{Separable                                                                                                                                              \\StructuringElement}}} & \texttt{dse.is\_decomposable()} & \texttt{std::bool\_constant} & \makecell[l]{\texttt{std::true\_type} if supported\\ \texttt{std::false\_false} otherwise.}\\
      \multicolumn{1}{c|}{}                                     & \texttt{sse.separate()}                               & \texttt{std::forward\_range}              & \makecell[l]{Return a range that yeilds \\ simpler structuring elements.}\\
      \multicolumn{1}{c|}{}                                     & \makecell[l]{\texttt{StructuringElement<}                                                                                                   \\ \texttt{std::ranges::range\_value\_t<} \\ \texttt{decltype(sse.separate())> >}}                            & \texttt{std::true\_type} & \makecell[l]{Is a range of compatible \\structuring elements types.}\\
      \multicolumn{1}{c|}{}                                     & \texttt{sse.is\_separable()}                          & \texttt{bool}                             & \makecell[l]{Wether the separate        \\ facility is supported}\\
      \cline{1-4}
      \multicolumn{1}{c|}{\multirow{4}{*}{\makecell[l]{Incremental                                                                                                                                            \\StructuringElement}}} & \texttt{dse.is\_decomposable()} & \texttt{std::bool\_constant} & \makecell[l]{\texttt{std::true\_type} if supported\\ \texttt{std::false\_false} otherwise.}\\
      \multicolumn{1}{c|}{}                                     & \texttt{ise.inc()}                                    & \makecell[l]{\texttt{StructuringElement<}                                           \\\texttt{SE, Pnt>}} & \makecell[l]{Return the next            \\ simpler structuring elements.}\\
      \multicolumn{1}{c|}{}                                     & \texttt{ise.dec()}                                    & \makecell[l]{\texttt{StructuringElement<}                                           \\\texttt{SE, Pnt>}} & \makecell[l]{Return the previous        \\ simpler structuring elements.}\\
      \multicolumn{1}{c|}{}                                     & \texttt{ise.is\_incremental()}                        & \texttt{bool}                             & \makecell[l]{Wether the incremental     \\ facility is supported}\\
      \cline{1-4}
    \end{tabular}
    \smallskip

    \caption{Concepts Structuring Elements: expressions}
  \end{scriptsize}
  \label{table:concept.se.expressions}
\end{table}

\paragraph{Concept code}

\begin{minted}{c++}
namespace details
{
  template <typename SE>
  concept DynamicStructuringElement =
    requires (SE se) {
    { se.radial_extent() }  -> std::same_as<int>;
    };

  constexpr bool implies(bool a, bool b) { return !a || b; }
}

template <typename SE, typename P>
concept StructuringElement =
  std::convertible_to<SE, mln::details::StructuringElement<SE>> &&
  std::ranges::regular_invocable<SE, P> &&
  std::ranges::regular_invocable<SE, mln::archetypes::PixelT<P>> &&
  requires {
    typename SE::category;
    typename SE::incremental;
    typename SE::decomposable;
    typename SE::separable;
  } &&
  std::convertible_to<typename SE::category, mln::adaptative_neighborhood_tag> &&
  details::implies(std::convertible_to<typename SE::category, mln::dynamic_neighborhood_tag>,
                    details::DynamicStructuringElement<SE>) &&
  requires (SE se, const SE cse, P p, mln::archetypes::PixelT<P> px) {
    { se(p) }         -> std::ranges::forward_range;
    { se(px) }        -> std::ranges::forward_range;
    { cse.offsets() } -> std::ranges::forward_range;

    requires std::convertible_to<std::ranges::range_value_t<decltype(se(p))>, P>;
    requires std::Pixel<std::ranges::range_value_t<decltype(se(px))>>;
    requires std::convertible_to<std::ranges::range_value_t<decltype(cse.offsets())>, P>;
  };

namespace details
{
  template <typename R, typename P>
  concept RangeOfStructuringElement =
    StructuringElement<std::ranges::range_value_t<R>, P>;
}

template <typename SE, typename P>
concept DecomposableStructuringElement =
  StructuringElement<SE, P> &&
  std::convertible_to<typename SE::decomposable, std::true_type> &&
  requires(const SE se) {
    { se.is_decomposable() }  -> std::same_as<bool>;
    { se.decompose() }        -> std::ranges::forward_range;
    requires details::RangeOfStructuringElement<decltype(se.decompose()), P>;
  };

template <typename SE, typename P>
concept SeparableStructuringElement =
  StructuringElement<SE, P> &&
  std::convertible_to<typename SE::separable, std::true_type> &&
  requires(const SE se) {
    { se.is_separable() } -> std::same_as<bool>;
    { se.separate() }     -> std::ranges::forward_range;
    requires details::RangeOfStructuringElement<decltype(se.separate()), P>;
  };

template <typename SE, typename P>
concept IncrementalStructuringElement =
  StructuringElement<SE, P> &&
  std::convertible_to<typename SE::incremental, std::true_type> &&
  requires(const SE se) {
    { se.is_incremental() } -> std::same_as<bool>;
    { se.inc() }  -> StructuringElement<SE, P>;
    { se.dec() }  -> StructuringElement<SE, P>;
  };
\end{minted}

\paragraph{Achetype code}

\begin{minted}{c++}
namespace details
{
  template <class P, class Pix>
  requires mln::concepts::Point<P>&& mln::concepts::Pixel<Pix>
  struct StructuringElement
  {
    using category     = adaptative_neighborhood_tag;
    using incremental  = std::false_type;
    using decomposable = std::false_type;
    using separable    = std::false_type;

      std::ranges::subrange<P*> operator()(P p);

      std::ranges::subrange<Pix*> operator()(Pix px);
      std::ranges::subrange<P*>   offsets() const;
  };


  template <class SE>
  struct AsSE : SE, mln::details::Neighborhood
  helper<AsSE<SE>>
  {
  };
} // namespace details

template <class P = Point, class Pix = PixelT<P>>
using StructuringElement = details::AsSE<details::StructuringElement<P, Pix>>;


namespace details
{
  template <class P, class Pix>
  struct DecomposableStructuringElement : StructuringElement<P, Pix>
  {
    using decomposable = std::true_type;

    bool                                                                is_decomposable() const;
    std::ranges::subrange<mln::archetypes::StructuringElement<P, Pix>*> decompose() const;
  };

  template <class P, class Pix>
  struct SeparableStructuringElement : StructuringElement<P, Pix>
  {
    using separable = std::true_type;

    bool                                                                is_separable() const;
    std::ranges::subrange<mln::archetypes::StructuringElement<P, Pix>*> separate() const;
  };

  template <class P, class Pix>
  struct IncrementalStructuringElement : StructuringElement<P, Pix>
  {
    using incremental = std::true_type;

    bool                                   is_incremental() const;
    archetypes::StructuringElement<P, Pix> inc()          const;
    archetypes::StructuringElement<P, Pix> dec()          const;
  };
} // namespace details


template <class P = Point, class Pix = PixelT<P>>
using DecomposableStructuringElement = details::AsSE<details::DecomposableStructuringElement<P, Pix>>;

template <class P = Point, class Pix = PixelT<P>>
using SeparableStructuringElement = details::AsSE<details::SeparableStructuringElement<P, Pix>>;

template <class P = Point, class Pix = PixelT<P>>
using IncrementalStructuringElement = details::AsSE<details::IncrementalStructuringElement<P, Pix>>;
\end{minted}


\clearpage

\subsection{Neighborhood}

\paragraph{Concept table}

\begin{table}[H]
  \begin{scriptsize}
    \begin{tabular}{llll}
      \cline{1-4}
      \thead{Concept} & \thead{Modeling type} & \thead{Inherit behavior from} & \thead{Instance of type} \\
      \cline{1-4}
      Neighborhood    & \texttt{SE, Pnt}      & StructuringElement, Point     & \texttt{se, pnt}         \\
      \cline{1-4}
    \end{tabular}
    \smallskip

    \caption{Concepts Neighborhood: definitions}
    \label{table:concept.nbh.definitions}
  \end{scriptsize}
\end{table}

\begin{table}[H]
  \begin{scriptsize}
    \begin{tabular}{lll}
      \cline{1-3}
      \thead{Type} & \thead{Instance of type} & \thead{Requirement}                           \\
      \cline{1-3}
      \texttt{Pix} & \texttt{pix}             & \texttt{std::same\_as<Pix::point\_type, Pnt>} \\
      \texttt{Pnt} & \texttt{pnt}             &                                               \\
      \cline{1-3}
    \end{tabular}
    \smallskip

    \begin{tabular}{llll}
      \cline{1-4}
      \thead{Concept}                                    & \thead{Expression}                          & \thead{Return Type}                           &
      \thead{Description}                                                                                                                                                                                           \\
      \cline{1-4}
      \multicolumn{1}{c|}{\multirow{8}{*}{Neighborhood}} & \texttt{se.before(pnt)}                     & \multirow{4}{*}{\texttt{std::forward\_range}} & Return a range that yields the points before \texttt{pnt}. \\
      \multicolumn{1}{c|}{}                              & \texttt{se.after(pnt)}                      &                                               & Return a range that yields the points after \texttt{pnt}.  \\
      \multicolumn{1}{c|}{}                              & \texttt{se.before(pix)}                     &                                               & Return a range that yields the pixels before \texttt{pix}. \\
      \multicolumn{1}{c|}{}                              & \texttt{se.after(pix)}                      &                                               & Return a range that yields the pixels after \texttt{pnt}.  \\
      \multicolumn{1}{c|}{}                              & \makecell[l]{\texttt{std::convertible\_to<}                                                                                                              \\\texttt{std::ranges::ranges\_value\_t<}\\\texttt{decltype(se.before(pnt))>, Pnt>}} & \multirow{4}{*}{\texttt{std::true\_type}}     & \multirow{2}{*}{Ranges converts to compatible element types.} \\
      \multicolumn{1}{c|}{}                              & \makecell[l]{\texttt{std::convertible\_to<}                                                                                                              \\\texttt{std::ranges::ranges\_value\_t<}\\\texttt{decltype(se.after(pnt))>, Pnt>}}  &                                               &                                                                    \\
      \multicolumn{1}{c|}{}                              & \makecell[l]{\texttt{Pixel<}                                                                                                                             \\\texttt{std::ranges::ranges\_value\_t<}\\\texttt{decltype(se.before(pix))>, Pix>}}                 &                                               & \multirow{2}{*}{Ranges that are of compatible element types.}      \\
      \multicolumn{1}{c|}{}                              & \makecell[l]{\texttt{Pixel<}                                                                                                                             \\\texttt{std::ranges::ranges\_value\_t<}\\\texttt{decltype(se.after(pix))>, Pix>}}                  &                                               &                                                                    \\
      \cline{1-4}
    \end{tabular}
    \smallskip

    \caption{Concepts Neighborhood: expressions}
  \end{scriptsize}
  \label{table:concept.nbh.expressions}
\end{table}

\paragraph{Concept code}

\begin{minted}{c++}
template <typename SE, typename P>
concept Neighborhood =
  StructuringElement<SE, P> &&
  requires (SE se, P p, mln::archetypes::PixelT<P> px) {
    { se.before(p) }  -> std::ranges::forward_range;
    { se.after(p) }   -> std::ranges::forward_range;
    { se.before(px) } -> std::ranges::forward_range;
    { se.after(px) }  -> std::ranges::forward_range;

    requires std::convertible_to<std::ranges::range_value_t<decltype(se.before(p))>, P>;
    requires std::convertible_to<std::ranges::range_value_t<decltype(se.after(p))>, P>;
    requires std::Pixel<std::ranges::range_value_t<decltype(se.before(px))>>;
    requires std::Pixel<std::ranges::range_value_t<decltype(se.after(px))>>;
  };
\end{minted}

\paragraph{Archetype code}

\begin{minted}{c++}
  namespace details
  {
    template <class P, class Pix>
    requires mln::concepts::Point<P>&& mln::concepts::Pixel<Pix>
    struct Neighborhood : StructuringElement<P, Pix>
    {
       std::ranges::iterator_range<P*>   before(P p);
       std::ranges::iterator_range<P*>   after(P p);
       std::ranges::iterator_range<Pix*> before(Pix px);
       std::ranges::iterator_range<Pix*> after(Pix px);
    };
  
    template <class N>
    struct AsNeighborhood : N, mln::details::Neighborhood<AsNeighborhood<N>>
    {
    };
  
  } // namespace details
  
  template <class P = Point, class Pix = PixelT<P>>
  using Neighborhood = details::AsSE<details::Neighborhood<P, Pix>>;
\end{minted}


\clearpage

\subsection{Extensions}

\paragraph{Concept table}


\begin{table}[H]
  \begin{scriptsize}
    \begin{tabular}{llll}
      \cline{1-4}
      \thead{Concept}       & \thead{Modeling type}  & \thead{Inherit behavior from} & \thead{Instance of type} \\
      \cline{1-4}
      Extension             & \texttt{Ext, Pnt}      & $\emptyset$, Point            & \texttt{ext, pnt}        \\
      FillableExtension     & \texttt{FExt, Pnt}     & Extension, Point              & \texttt{fext}            \\
      MirrorableExtension   & \texttt{MExt, Pnt}     & Extension, Point              & \texttt{mext}            \\
      PeriodizableExtension & \texttt{PExt, Pnt}     & Extension, Point              & \texttt{pext}            \\
      ClampableExtension    & \texttt{CExt, Pnt}     & Extension, Point              & \texttt{cext}            \\
      ExtendWithExtension   & \texttt{EwExt, Pnt, U} & Extension, Point, Image       & \texttt{ewext, u}        \\
      \cline{1-4}
    \end{tabular}
    \smallskip

    \begin{tabular}{llll}
      \cline{1-4}
      \thead{Concept}                                 & \thead{Definition}                     & \thead{Description}                           &
      \thead{Requirement}                                                                                                                                                                                           \\
      \cline{1-4}
      \multicolumn{1}{c|}{\multirow{6}{*}{Extension}} & \texttt{value\_type}                   & \makecell[l]{Type of value
      contained                                                                                                                                                                                                     \\ in the range.                                                                                                                                                                             Cannot be \\ constant
      or reference.}                                  & Models the concept \texttt{Value}.                                                                                                                          \\
      \multicolumn{1}{c|}{}                           & \texttt{support\_fill}                 & \multirow{5}{*}{\texttt{std::bool\_constant}} & \multirow{5}{*}{\makecell[l]{\texttt{std::true\_type} if supported \\ \texttt{std::false\_false} otherwise.}} \\
      \multicolumn{1}{c|}{}                           & \texttt{support\_mirror}               &                                               &                                                                    \\
      \multicolumn{1}{c|}{}                           & \texttt{support\_periodize}            &                                               &                                                                    \\
      \multicolumn{1}{c|}{}                           & \texttt{support\_clamp}                &                                               &                                                                    \\
      \multicolumn{1}{c|}{}                           & \texttt{support\_extend\_with}         &                                               &                                                                    \\
      \cline{1-4}
      \multicolumn{1}{c|}{ExtendWithExtension}        & \texttt{point\_type}                   & \makecell[l]{Type of point in                                                                                      \\ the
      extended image.}                                & \makecell[l]{Converts to the sub-image                                                                                                                      \\ points type.}                                                                                                            \\
      \cline{1-4}
    \end{tabular}
  \end{scriptsize}
  \smallskip

  \caption{Concepts Extensions: definitions}
  \label{table:concept.extensions.definitions}
\end{table}

\begin{table}[H]
  \begin{scriptsize}
    \begin{tabular}{ll}
      \cline{1-2}
      \thead{Type}              & \thead{Instance of type} \\
      \cline{1-2}
      \texttt{SE<Pnt, Pix>}     & \texttt{se}              \\
      \texttt{Ext::value\_type} & \texttt{val}             \\
      \texttt{Ext::point\_type} & \texttt{offset}          \\
      \cline{1-2}
    \end{tabular}
    \smallskip

    \begin{tabular}{llll}
      \cline{1-4}
      \thead{Concept}                                              & \thead{Expression}                            & \thead{Return Type} &
      \thead{Description}                                                                                                                                      \\
      \cline{1-4}
      \multicolumn{1}{c|}{\multirow{2}{*}{Extension}}              & \texttt{ext.fit(se)}                          & \texttt{bool}       & \makecell[l]{Wether
      the extension fit                                                                                                                                        \\ the structuring element.}                                                                                                   \\
      \multicolumn{1}{c|}{}                                        & \texttt{ext.extent()}                         & \texttt{int}        &
      \makecell[l]{Extension's border width.                                                                                                                   \\
      std::numeric\_limits<int>::max()                                                                                                                         \\ for infinite size.}                                                                                         \\
      \cline{1-4}
      \multicolumn{1}{c|}{\multirow{2}{*}{FillableExtension}}      & \texttt{fext.fill(val)}                       & \texttt{void}       &
      \makecell[l]{Fill the extension with                                                                                                                     \\the value \texttt{val}.}                                                                                              \\
      \multicolumn{1}{c|}{}                                        & \texttt{fext.is\_fill\_supported}             & \texttt{bool}       &
      \makecell[l]{Wether the fill facility                                                                                                                    \\is supported.}                                                                                                                            \\
      \cline{1-4}
      \multicolumn{1}{c|}{\multirow{2}{*}{MirrorableExtension}}    & \texttt{mext.mirror()}                        & \texttt{void}       & \makecell[l]{Fill
      the extension with                                                                                                                                       \\mirrored image's values.}                                                                                                  \\
      \multicolumn{1}{c|}{}                                        & \texttt{mext.is\_mirror\_supported()}         &
      \texttt{bool}                                                & \makecell[l]{Wether the mirror facility                                                   \\ is supported.}                                             \\
      \cline{1-4}
      \multicolumn{1}{c|}{\multirow{2}{*}{PeriodizeableExtension}} & \texttt{mext.periodize()}                     & \texttt{void}       &
      \makecell[l]{Fill the extension with                                                                                                                     \\periodic copies of image's values.}                                                                                   \\
      \multicolumn{1}{c|}{}                                        & \texttt{mext.is\_periodize\_supported()}      &
      \texttt{bool}                                                & \makecell[l]{Wether the mirror facility                                                   \\is supported.}                                      \\
      \cline{1-4}
      \multicolumn{1}{c|}{\multirow{2}{*}{ClampableExtension}}     & \texttt{mext.clamp()}                         & \texttt{void}       & \makecell[l]{Fill
      the extension by                                                                                                                                         \\extending image's values \\at extremities.}                                                                                    \\
      \multicolumn{1}{c|}{}                                        & \texttt{mext.is\_clamp\_supported()}          &
      \texttt{bool}                                                & \makecell[l]{Wether the clamp facility                                                    \\is supported.}                                       \\
      \cline{1-4}
      \multicolumn{1}{c|}{\multirow{3}{*}{ExtendWithExtension}}    &
      \makecell[l]{\texttt{std::convertible\_to<point\_type,}                                                                                                  \\\texttt{U::point\_type>}} & \texttt{std::true\_type} &
      \makecell[l]{Converts to the sub-image                                                                                                                   \\ points type.}                                                                                                       \\
      \multicolumn{1}{c|}{}                                        & \texttt{mext.extend\_with(u, offset)}         &
      \texttt{void}                                                & \makecell[l]{Fill the extension with                                                      \\sub-image      \texttt{u}'s values  starting \\ from \texttt{offset}.} \\
      \multicolumn{1}{c|}{}                                        & \texttt{mext.is\_extent\_with\_supported()}   &
      \texttt{bool}                                                & \makecell[l]{Wether the extend-with-sub-image                                             \\ facility is
      supported.}                                                                                                                                              \\
      \cline{1-4}
    \end{tabular}
    \smallskip

    \caption{Concepts Extensions: expressions}
  \end{scriptsize}
  \label{table:concept.extensions.expressions}
\end{table}

\paragraph{Concept code}

\begin{minted}{c++}
template <typename Ext, typename Pnt>
concept Extension =
  std::is_base_of_v<mln::Extension<Ext>, Ext> &&
  requires {
    typename Ext::support_fill;
    typename Ext::support_mirror;
    typename Ext::support_periodize;
    typename Ext::support_clamp;
    typename Ext::support_extend_with;
  } &&
  Value<typename Ext::value_type> &&
  requires (const Ext cext,
      mln::archetypes::StructuringElement<
        Pnt,
        mln::archetypes::Pixel> se) {
    { cext.fit(se) }      -> std::same_as<bool>;
    { cext.extent() }     -> std::same_as<int>;
  };

template <typename Ext, typename Pnt>
concept FillableExtension =
  Extension<Ext, Pnt> &&
  std::convertible_to<typename Ext::support_fill, std::true_type> &&
  requires {
    typename Ext::value_type;
  } &&
  requires (Ext ext, const Ext cext, const typename Ext::value_type& v) {
    { ext.fill(v) };
    { cext.is_fill_supported() }  -> std::same_as<bool>;
  };

template <typename Ext, typename Pnt>
concept MirrorableExtension =
  Extension<Ext, Pnt> &&
  std::convertible_to<typename Ext::support_mirror, std::true_type> &&
  requires (Ext ext, const Ext cext) {
    { ext.mirror() };
    { cext.is_mirror_supported() }  -> std::same_as<bool>;
  };

template <typename Ext, typename Pnt>
concept PeriodizableExtension =
  Extension<Ext, Pnt> &&
  std::convertible_to<typename Ext::support_periodize, std::true_type> &&
  requires (Ext ext, const Ext cext) {
    { ext.periodize() };
    { cext.is_periodize_supported() }  -> std::same_as<bool>;
  };

template <typename Ext, typename Pnt>
concept ClampableExtension =
  Extension<Ext, Pnt> &&
  std::convertible_to<typename Ext::support_clamp, std::true_type> &&
  requires (Ext ext, const Ext cext) {
    { ext.clamp() };
    { cext.is_clamp_supported() }  -> std::same_as<bool>;
  };

template <typename Ext, typename Pnt, typename U>
concept ExtendWithExtension =
  Extension<Ext, Pnt> &&
  std::convertible_to<typename Ext::support_extend_with, std::true_type> &&
  InputImage<U> &&
  requires {
    typename Ext::point_type;
  } &&
  std::convertible_to<typename U::value_type, typename Ext::value_type> &&
  std::convertible_to<typename Ext::point_type, typename U::point_type> &&
  requires (Ext ext, const Ext cext, U u, typename Ext::point_type offset) {
    { ext.extend_with(u, offset) };
    { cext.is_extend_with_supported() }  -> std::same_as<bool>;
  };
\end{minted}

\paragraph{Archetype code}

\begin{minted}{c++}
  // TODO
\end{minted}


\clearpage

\subsection{Extended image}

\paragraph{Concept table}

\begin{table}[H]
  \begin{scriptsize}
    \begin{tabular}{llll}
      \cline{1-4}
      \thead{Concept}    & \thead{Modeling type} & \thead{Inherit behavior from} & \thead{Instance of type} \\
      \cline{1-4}
      WithExtensionImage & \texttt{WExtImg}      & Image                         & \texttt{wextimg}         \\
      ConcreteImage      & \texttt{CImg}         & Image                         & \texttt{cimg}            \\
      ViewImage          & \texttt{VImg}         & Image                         & \texttt{vimg}            \\
      \cline{1-4}
    \end{tabular}
    \smallskip

    \begin{tabular}{llll}
      \cline{1-4}
      \thead{Concept}                         & \thead{Definition}       & \thead{Description}            & \thead{Requirement}                    \\
      \cline{1-4}
      \multicolumn{1}{c|}{WithExtensionImage} & \texttt{extension\_type} & Type of the image's extension. & Models the concept \texttt{Extension}. \\
      \cline{1-4}
    \end{tabular}
    \smallskip

    \caption{Concepts Image: definitions (7)}
    \label{table:concept.image.definitions.7}
  \end{scriptsize}
\end{table}

\begin{table}[H]
  \begin{scriptsize}
    \begin{tabular}{llll}
      \cline{1-4}
      \thead{Concept} & \thead{Expression} & \thead{Return Type} &
      \thead{Description}                                          \\
      \cline{1-4}
      \multicolumn{1}{c|}{\makecell[l]{WithExtension               \\Image}}           & \texttt{wextimg.extension()}                         & \makecell[l]{\texttt{std::convertible\_to<}\\\texttt{extension\_type>}} & \makecell[l]{Get the extension of the image.} \\
      \cline{1-4}
    \end{tabular}
    \smallskip

    \caption{Concepts Image: expressions (7)}
  \end{scriptsize}
  \label{table:concept.image.expressions.7}
\end{table}

\paragraph{Concept code}

\begin{minted}{c++}
template <typename I>
concept WithExtensionImage =
  Image<I> &&
  requires {
    typename image_extension_t<I>;
  } &&
  Extension<image_extension_t<I>, image_point_t<I>> &&
  not std::same_as<mln::extension::none_extension_tag, image_extension_category_t<I>> &&
  requires (I ima, image_point_t<I> p) {
    { ima.extension() } -> std::convertible_to<image_extension_t<I>>;
  };
\end{minted}

\paragraph{Archetype code}

\begin{minted}{c++}
namespace details {
  struct WithExtensionImage : Image
  {
    struct Extension : ::mln::Extension<Extension>
    {
      using support_fill = std::false_type;
      using support_mirror = std::false_type;
      using support_periodize = std::false_type;
      using support_clamp = std::false_type;
      using support_extend_with = std::false_type;
      using value_type = image_value_t<Image>;
      bool fit(mln::archetypes::StructuringElement<image_point_t<Image>, mln::archetypes::Pixel> se) const;
      int extent() const;
    };
  
    using extension_type = Extension;

    using extension_category = mln::extension::custom_extension_tag;

    extension_type extension() const;
  };
} // namespace details

using WithExtensionImage = details::AsImage<details::WithExtensionImage>;
\end{minted}


\clearpage

\subsection{Output image}

\paragraph{Concept table}

\begin{figure}[!htbp]
  \begin{scriptsize}
    \begin{equation}
      \begin{aligned}
        OutputImage ={} & (Image \implies WritableImage) \wedge                                             \\
                        & (IndexableImage \implies WritableIndexableImage) \wedge                           \\
                        & (AccessibleImage \implies WritableAccessibleImage) \wedge                         \\
                        & (IndexableAndAccessibleImage \implies WritableIndexableAndAccessibleImage) \wedge \\
                        & (BidirectionalImage \implies WritableBidirectionalImage) \wedge                   \\
                        & (RawImage \implies WritableRawImage)
      \end{aligned}
    \end{equation}
    \smallskip

    \caption{Concepts OutputImage: definition}
  \end{scriptsize}
  \label{table:concept.image.expressions.8}
\end{figure}

\paragraph{Concept code}

\begin{minted}{c++}
// OutputImage
// Usage: RawImage<I> && OutputImage<I>
template <typename I>
concept OutputImage =
  (not Image<I> || (detail::WritableImage<I>)) &&
  (not IndexableImage<I> || (detail::WritableIndexableImage<I>)) &&
  (not AccessibleImage<I> || (detail::WritableAccessibleImage<I>)) &&
  (not IndexableAndAccessibleImage<I> ||
    (detail::WritableIndexableAndAccessibleImage<I>)) &&
  (not BidirectionalImage<I> || (detail::WritableBidirectionalImage<I>)) &&
  (not RawImage<I> || (detail::WritableRawImage<I>));
\end{minted}
