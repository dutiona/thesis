Through concrete benchmarks and examples, we have shown how to leverage genericity nowadays without slowing down the
performances. There are several types of genericity which have been presented, as well as several widely used
implementation of them in the industry. Our take offer a new approach to reach the goal of having one code for several
algorithm, one algorithm for several image types. Furthermore, we introduce meta-algorithms (canvas) that are based on
behavior patterns known once the image type is known. We also show how C++ template metaprogramming techniques allow not
to impact the performances despites the indirections induced. Finally, we show how an approach based on properties (on
image type as well as on external data such as structuring elements) can be beneficial to introduce customization points
to take advantage of opportunities to increase performances. When coupling both properties and algorithm canvas, it
becomes standard for a user to write efficient and generic algorithm by default.

It is also shown how we were able to abstract two of the three main families of algorithms. First are point-wise
algorithms that can all be expressed through the \emph{views}.\emph{Views} enables streamlining the writting process of
image processing pipelines so that it is shorter, efficient and expressive. Second are local algorithm whose
problematics (border management, structuring elements, pass number) can all be abstracted away behind a canvas hiding
the complexity and taking advantage of opportunities to increase performance for you.

The solutions presented in this paper do have some disadvantages, such as the readability when using local algorithm
canvas, code-bloat due to heavy instantiation of C++ templates especially in the views and finally the lack of
availability to the dynamic (prototyping) world by default. Indeed, C++ metaprogramming needs to be compiled at the time
of its usage preventing direct link with dynamic languages such as python. There is on-going work to introduce dynamic
dispatch at some key points to reduce code size without impairing the performance. For instance, a dynamic dispatch to
select the correct version of an algorithm has much less impact than a dynamic dispatch when accessing the value of a
pixel. Further work is required to improve the static-dynamic bridge and bring the capabilities of the techniques
presented in this paper to the dynamic world, and in particular, python which is vastly used in the image processing
world.

\clearpage



\begin{itemize}
  \item Synthèse générale
  \item Réponse à la problématique d'Introduction
  \item confrontation à d'autres travaux de recherche ayant donné naissance à une bibliothèque de TI
  \item Ouvertures, perspectives, limites
        \begin{itemize}
          \item continuité JIT
          \item ce qu'il reste à faire
        \end{itemize}
\end{itemize}