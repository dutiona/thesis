\documentclass{book}
\usepackage[utf8]{inputenc}
\usepackage[T1]{fontenc}
\usepackage[numbers]{natbib}
\usepackage[mode=buildmissing]{standalone}
\usepackage{tikz}
\usepackage{graphicx}
\usepackage{hyperref}
\usepackage{cleveref}
\usepackage{subcaption}
\usepackage{minted}
\usepackage{csquotes}
\usepackage{multirow}
\usepackage{tabularx}
\usepackage{tikz-uml}
\usepackage[]{siunitx}
\usepackage{makecell}

\usetikzlibrary{3d,calc,positioning}
\tikzumlset{font=\scriptsize\tt}
\setminted{fontsize=\scriptsize}

\renewcommand\UrlFont{\color{blue}\rmfamily}

\begin{document}
%
\title{Programming in modern C++ for image processing}


\author{Michaël Roynard}

%\institute{EPITA Research and Development Laboratory\\
%  \email{michael.roynard@lrde.epita.fr}
%}


\maketitle

\section{Acknowledgement}


\section{Abstract}
  TODO EN

  TODO FR

\section{Long abstract}



\tableofcontents
\listoffigures
\listoftables

\cleardoublepage


\chapter{Introduction}

\begin{itemize}
  \item Définition, sujet, contexte d'étude -> reprendre texte de ma thèses en 3 mn
  \item Question de départ, problématique -> généricité + traitement d'images
  \item Présentation rapide de la méthodologie -> génie logiciel bas niveau dans le but de remonter haut niveau pour utilisateur TI
  \item Objectif et plan
  \item Justification de l'étude
\end{itemize}

\cleardoublepage


\chapter{Motivations and context}

\begin{itemize}
  \item Domaines du TI et besoins par domaine
  \item Profils des utilisateurs et besoins en fonction des profils Intégrateur/Développeur vs Praticien
  \item Contexte du LRDE (Expériences de Olena \& spécialisation en Morph. Math, Topo. Discrète), définitions de nos besoins spécifiques (outils pour l'expérimentation, outils pour l'éducation, outils pour le logiciel de production)
  \item Définitions du périmètre de la bibliothèque et de ses objectifs:
    \begin{itemize}
      \item Performance
      \item Facile d'utilisation (UX client)
      \item Facile de développement (Core developer xp)
      \item Versatilité des types d'images
      \item Utilisable depuis Python, Orientée MM
    \end{itemize}
\end{itemize}

\cleardoublepage


\chapter{Genericity}

\section{Les différentes façon d'atteindre la généricité}

\begin{itemize}
  \item Approches de la généricité dans les bibliothèques
  \item Approches de la généricité dans les langages
\end{itemize}

\section{La généricité en C++ pre-11}

\begin{itemize}
  \item explications techniques bas niveau, SFINAE
  \item aboutissement à SCOOP -> synthèse et explication des travaux précédents sur SCOOP
\end{itemize}

\section{La généricité en Modern C++, C++11 et C++20 (concepts)}

\begin{itemize}
  \item post C++11 :
    \begin{itemize}
      \item simplification d'écriture du SFINAE (variadic, traits)
      \item if constexpr (c++14), lambda auto
      \item apports du C++17 pour simplifier encore : folds, template deduction guides, visit/overload for variant, any, template <auto V> struct S
      \item apports du C++20 : concepts, ranges
    \end{itemize}
  \item montrer des exemples de simplification de code
  \item faire des bench de temps de compilation (flagrant quand passage du SFINAE aux concepts)
\end{itemize}

\section{Les templates C++ dans un monde dynamique}

\begin{itemize}
   \item rapide rappel pros./cons. language compilés vs. languages interprétés
   \item rappel template C++ -> pas de binaire
   \item faire un topo sur la problématique de distribution de binaire et/ou de compilation chez le client pour un code générique
   \item passer en revue les solutions existantes :
     \begin{itemize}
       \item SWILENA
       \item VCSN
       \item VIGRA
       \item Cython
       \item cppyy
    \end{itemize}
\end{itemize}


\cleardoublepage


\chapter{Images and algorithms taxonomy}

\begin{itemize}
  \item Définitions des types/catégories de types/propriétés de types
  \item Value/Ref semantics des images
  \item Conceptualisation expliquer par l'exemple (papier rrpr) la relation Image (n) <-> Implem (m) avec n >> m
    \begin{itemize}
      \item Concepts déduis des algorithmes et non des types
      \item Plusieurs algos pour le même opérateur
      \item Plusieurs implems pour le même algo
    \end{itemize}
  \item Vision ensembliste des types d'images/algo
    \begin{itemize}
      \item versions d'algorithmes
      \item spécialisations d'algorithmes
    \end{itemize}
  \item canvas d'algorithmes
    \begin{itemize}
      \item factorisation
      \item opportunités d'optimisation (utilisation de propriétés, parallélisation)
    \end{itemize}
  \item listing et explications des concepts de la bibliothèque (images, élément structurant)
\end{itemize}

\cleardoublepage


\chapter{Images views}

\begin{itemize}
  \item origine, parallèle avec range-v3
  \item Comment préserver les propriétés
  \item Évaluation paresseuse
  \item Composabilité/chaînage
  \item ...
  \item Performances + Bench
\end{itemize}

\cleardoublepage


\chapter{Static-dynamic bridge}

\begin{itemize}
  \item rappel de la problématique (backward ref depuis Généricité/4.)
  \item expliquer l'approche hybride (son design et les techniques de dispatch n*n avec variants)
  \item bench, trade-off
  \item continuité sur JIT avec autowig, cppyy (perspective)
\end{itemize}

\cleardoublepage


\chapter{Conclusion and perspectives}

\begin{itemize}
  \item Synthèse générale
  \item Réponse à la problématique d'Introduction
  \item confrontation à d'autres travaux de recherche ayant donné naissance à une bibliothèque de TI
  \item Ouvertures, perspectives, limites
  \begin{itemize}
    \item continuité JIT
    \item ce qu'il reste à faire
  \end{itemize}
\end{itemize}

\cleardoublepage


\chapter{Appendices}

\appendix

\section{Bibliography}

\bibliography{bibliography}


\end{document}
