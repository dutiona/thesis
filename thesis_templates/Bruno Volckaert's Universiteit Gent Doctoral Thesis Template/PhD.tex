\documentclass[10pt,a4paper,twoside,dutch,english]{book}                % options used for 'Hoofdstuk' or 'Chapter'

%%%%%%%%%%%%%%%%%%%%%%%%%%
%% PACKAGE LOADING TIME %%
%%%%%%%%%%%%%%%%%%%%%%%%%%

%\usepackage{lscape}
\usepackage{pseudocode}
\usepackage{times}      % use Times New Roman Type 1 fonts  (redefines sfdefault,rmdefault,ttdefault)
%\usepackage[T1]{fontenc}
%\usepackage{pslatex}
\usepackage[times]{quotchap}   % fancy chapter beginning
\usepackage{fancyhdr}
\usepackage[sectionbib]{chapterbib2} % bibliography per chapter
%\usepackage[sectionbib]{chapterbib2} % bibliography per chapter
% (sectionbib -> bibliography is \section* instead of \chapter*), should come before babel chapterbib2 
%because local version is 1.9 and solves bug that header was 'References' instead of Chaptername
\usepackage[dutch,english]{babel}
%\usepackage[sectionbib,numbers,sort&compress]{natbib}  %for citations a la 'Vermeulen et al.' instead of [1]
%\usepackage{tocbibind} % automatically add bibliography, list of figures, ... to table of contents
\usepackage[a4paper,verbose, asymmetric, centering]{geometry}  % better control over margins
%\usepackage[hang]{caption}     %better control over captions (sideways, font, ...)  hang -> 2nd line of caption is indented (caption2 is deprecated and beta)
\usepackage[justification=centering]{caption}     %better control over captions (sideways, font, ...)  
\usepackage{subfigure}  % with scriptsize or so, one can adapt the size
\usepackage{cite}

\usepackage{enumerate}  % to make it possible to define the numbers (A,a, ...)
\usepackage{verbatim}   % extra support for verbatim environments
\usepackage{float}      % you can define 'H' so that floats are forced to be putted 'here'
\usepackage{multirow}   % multirow{nrows}[bigstruts]{width}[fixup]{text} multirow cells
\ifx\pdftexversion\undefined
\usepackage[dvips]{graphicx}
\else
\usepackage[pdftex]{graphicx}
\fi
%\usepackage{psfrag}
\usepackage{chappg}     % page numbering (chapno-pageno), for ToC
\usepackage{url}        % for better url typesetting
\usepackage{expdlist}   % Expanded description (e.g. better alignement) -> needed for acronym_expdlist package
\usepackage{acronym_expdlist}   % for list of acronyms
\usepackage{hhline}     % generates nicer table lines (without missing pixels) + more flexible
%\usepackage{colortbl}  % for coloured columns
%\usepackage{threeparttable}     % adds the possibility to add footnotes in tables
\usepackage{afterpage}  % adds \afterpage command, which makes it possible to issue \afterpage{\clearpage} which flushes all floats after this page
%\usepackage{amsmath}    % adds extra commands, ao. \text within math environment
\usepackage{amsmath,amsfonts,amsthm}
\usepackage{marvosym}   % for Euro symbol
\usepackage{ifthen}     % ifthenelse command
%\usepackage{mathenv}	% better eqnarray
\usepackage{listings}


%%%%%%%%%%%%%%%%%
%% PAGE LAYOUT %%
%%%%%%%%%%%%%%%%%

\input{page_layout_definition.tex}

%%%%%%%%%%%%%%%%%
%% HYPHENATION %%
%%%%%%%%%%%%%%%%%

\input{hyphenation.tex} % to have a separate file with hyphenations

%%%%%%%%%%%%%%%%%%%%
%%  START BOOK    %%
%%%%%%%%%%%%%%%%%%%%

\begin{document}
\graphicspath{{fig/}}
\restylefloat{figure}
\restylefloat{table}
\newfloat{algorithm}{ht}{alg}

%%   FRONT PAGE       %%
%%%%%%%%%%%%%%%%%%%%%%%%
% 
 \thispagestyle{empty}   % no headings for this page
% 
% % Header
 \noindent
 \begin{minipage}{3cm}%
   \includegraphics*[width=2cm]{UGentlogo}% logo ugent
 \end{minipage}\hfill
 \begin{minipage}{8cm}
 \raggedleft
 \textsf{Universiteit Gent\\
 Faculteit Ingenieurswetenschappen\\
 Vakgroep Informatietechnologie}
 \end{minipage}
% 
% % Title
\bigskip
   \begin{flushleft}
     \Large \textsf{Architecturen en algoritmen voor netwerk- en dienstbewust Grid-resourcebeheer}\\
     \vspace{0.1in}\large{\textsf{Architectures and Algorithms for Network and Service Aware Grid Resource Management}}%\noindent
   \end{flushleft}
 \hrule
% 
 \bigskip
   \LARGE\noindent \textsf{Bruno Volckaert} \hfill
 \bigskip
% 
% % Front Figure
% %\bigskip
% %\begin{figure}[htb]%
% %  \centering%
% %  \includegraphics*[width=\textwidth]{figFront}%
% %\end{figure}%
% %\normalsize
% 
% \bigskip
% \begin{figure}[h!]%
%   \centering%
%   \includegraphics*[height=10cm, keepaspectratio, clip]{figFront}%
% \end{figure}%
 \normalsize
% 
% % Footer
 \vfill
 \begin{minipage}{2.0cm}%
     \includegraphics*[width=2.0cm]{intec}%       % intec logo
 \end{minipage}\hfill
 \begin{minipage}{9cm}
 \raggedleft
 \textsf{Proefschrift tot het bekomen van de graad van \\
 Doctor in de Ingenieurswetenschappen: \\
 Computerwetenschappen \\
 Academiejaar 2005-2006}
 \end{minipage}


%% INFORMATION PAGE     %%
%%%%%%%%%%%%%%%%%%%%%%%%%%

\clearpage{\pagestyle{empty}\cleardoublepage}
\thispagestyle{empty}

\normalsize

% Header
\noindent
\begin{minipage}{3cm}%
  \includegraphics*[width=2cm]{UGentlogo}% logo ugent
\end{minipage}\hfill
\begin{minipage}{8cm}
\raggedleft
\textsf{Universiteit Gent\\
Faculteit Ingenieurswetenschappen\\
Vakgroep Informatietechnologie}
\end{minipage}
\\[2cm]

\noindent \begin{tabular}{ @{} l l}
Promotoren: & Prof.\ Dr.\ Ir.\ Bart Dhoedt \\
 & Prof.\ Dr.\ Ir.\ Filip De Turck \\
\end{tabular}
\\[2cm]

%\noindent Promotor: Prof.\ Dr.\ Ir.\ P.\ Demeester \\
\noindent Universiteit Gent \\
\noindent Faculteit Ingenieurswetenschappen
\\[0.3cm]
\noindent Vakgroep Informatietechnologie \\
\noindent Gaston Crommenlaan 8 bus 201,
\noindent B-9050 Gent, Belgi\"e 
\\[0.3cm]
\noindent Tel.: +32-9-331.49.00\\
\noindent Fax.: +32-9-331.48.99
\\[4cm]
\noindent Dit werk kwam tot stand in het kader van een specialisatiebeurs van het IWT-Vlaanderen (Instituut voor de aanmoediging van Innovatie door Wetenschap en Technologie in Vlaanderen).

\vfill

\begin{minipage}{2.0cm}%
    \includegraphics*[width=2.0cm]{intec}%       % intec logo
\end{minipage}\hfill
\begin{minipage}{9cm}
\raggedleft
\textsf{Proefschrift tot het behalen van de graad van \\
Doctor in de Ingenieurswetenschappen: \\
Computerwetenschappen \\
Academiejaar 2005-2006}
\end{minipage}
\clearpage{\pagestyle{empty}\cleardoublepage}


%% ACKNOWLEDGMENT   %%
%%%%%%%%%%%%%%%%%%%%%%%

\selectlanguage{dutch}
\hyphenation{bu-reau-ge-no-ten}
\frontmatter
\chapter{Dankwoord}
\vspace{0.35in}

dankwoord...

\begin{flushright}{\emph{Gent, maart 2006\\
Bruno Volckaert}}
\end{flushright}
\selectlanguage{english}

\clearpage{\pagestyle{empty}\cleardoublepage}

%%   TOC, LIST OF FIGURES, LIST OF TABLES, ACRONYMS     %%
%%%%%%%%%%%%%%%%%%%%%%%%%%%%%%%%%%%%%%%%%%%%%%%%%%%%%%%%%%

\renewcommand{\contentsname}{Table of Contents} % original name = Contents

\tableofcontents
\clearpage{\pagestyle{empty}\cleardoublepage}

\listoffigures
\clearpage{\pagestyle{empty}\cleardoublepage}

\listoftables
\clearpage{\pagestyle{empty}\cleardoublepage}

% List of Acronyms
%%%%%%%%%%%%%%%%%%%%%%%%%%%%%%%%%%%%%%%%%%%%%%%%%%%%%%%%%%%%%%%%%%%%%

\chapter*{List of Acronyms}
%     \@mkboth{\MakeUppercase List of Acronyms}%
%              {\MakeUppercase List of Acronyms}%

%\begin{center}
%\textbf{\Large List of Acronyms}
%\end{center}
\textbf{\newline\newline\Large A} \newline
\begin{acronym_expdlist}
\acro{ACE}{Adaptive Communication Environment}
\acro{ASAP}{As Soon As Possible}
\end{acronym_expdlist}

\textbf{\newline\newline\Large C} \newline
\begin{acronym_expdlist}
\acro{CDR}{Common Data Representation}
\acro{CM}{Connection Manager}
\acro{CORBA}{Common Object Request Broker Architecture}
\acro{CR}{Computational Resource}
\end{acronym_expdlist}

\textbf{\newline\newline\Large D} \newline
\begin{acronym_expdlist}
\acro{DIT}{Directory Information Tree}
\acro{DN}{Distinguished Name}
\acro{DR}{Data Resource}
\end{acronym_expdlist}

\textbf{\newline\newline\Large E} \newline
\begin{acronym_expdlist}
\acro{EGEE}{Enabling Grids for E-Science in Europe}
\end{acronym_expdlist}

\textbf{\newline\newline\Large F} \newline
\begin{acronym_expdlist}
\acro{FCFS}{First Come First Served}
\end{acronym_expdlist}

\textbf{\newline\newline\Large G} \newline
\begin{acronym_expdlist}
\acro{GA}{Genetic Algorithm}
\acro{GGF}{Global Grid Forum}
\acro{GIIS}{Grid Information Index Server}
\acro{GMA}{Grid Monitoring Architecture}
\acro{GRIS}{Grid Resource Information Service}
\acro{GS}{Grid Scheduler}
\acro{GUI}{Graphical User Interface}
\end{acronym_expdlist}

\textbf{\newline\newline\Large H} \newline
\begin{acronym_expdlist}
\acro{HTTP}{HyperText Transfer Protocol}
\end{acronym_expdlist}

\textbf{\newline\newline\Large I} \newline
\begin{acronym_expdlist}
\acro{IAT}{Inter Arrival Time}
\acro{ILP}{Integer Linear Program}
\acro{IP}{Internet Protocol}
\acro{IS}{Information Service}
\acro{ISP}{Internet Service Provider}
\end{acronym_expdlist}

\textbf{\newline\newline\Large J} \newline
\begin{acronym_expdlist}
\acro{JAMM}{Java Agents for Monitoring and Management}
\end{acronym_expdlist}

%\textbf{\newline\newline\Large K} \newline
%\begin{acronym_expdlist}
%\end{acronym_expdlist}

\textbf{\newline\newline\Large L} \newline
\begin{acronym_expdlist}
\acro{LAN}{Local Area Network}
\acro{LCG}{LHC Computing Grid}
\acro{LDAP}{Lightweight Directory Access Protocol}
\acro{LHC}{Large Hadron Collider}
\end{acronym_expdlist}

\textbf{\newline\newline\Large M} \newline
\begin{acronym_expdlist}
\acro{MAN}{Medium Area Network}
\acro{MDS}{Globus Monitoring and Discovery Service}
\acro{MPLS}{MultiProtocol Label Switching}
\end{acronym_expdlist}

\textbf{\newline\newline\Large N} \newline
\begin{acronym_expdlist}
\acro{NWS}{Network Weather Service}
\end{acronym_expdlist}

\textbf{\newline\newline\Large O} \newline
\begin{acronym_expdlist}
\acro{OGSA}{Open Grid Services Architecture}
\acro{OGSI}{Open Grid Services Infrastructure}
\end{acronym_expdlist}

\textbf{\newline\newline\Large Q} \newline
\begin{acronym_expdlist}
\acro{QoS}{Quality of Service}
\end{acronym_expdlist}

\textbf{\newline\newline\Large R} \newline
\begin{acronym_expdlist}
\acro{R-GMA}{Relational Grid Monitoring Architecture}
\acro{RM}{Replica Manager}
\acro{RTT}{Round Trip Time}
\end{acronym_expdlist}

\textbf{\newline\newline\Large S} \newline
\begin{acronym_expdlist}
\acro{SM}{Service Manager}
\acro{SMON}{Service Monitor}
\acro{SOAP}{Simple Object Access Protocol}
\acro{SQL}{Standard Query Language}
\acro{SR}{Storage Resource}
\end{acronym_expdlist}

\textbf{\newline\newline\Large T} \newline
\begin{acronym_expdlist}
\acro{TCP}{Transport Control Protocol}
\acro{TIO}{Tivoli Intelligent Orchestrator}
\acro{TPM}{Tivoli Provisioning Manager}
\end{acronym_expdlist}

\textbf{\newline\newline\Large U} \newline
\begin{acronym_expdlist}
\acro{UDP}{User Datagram Protocol}
\end{acronym_expdlist}

\textbf{\newline\newline\Large V} \newline
\begin{acronym_expdlist}
\acro{VO}{Virtual Organisation}
\acro{VPG}{Virtual Private Grid}
\acro{VPN}{Virtual Private Network}
\end{acronym_expdlist}

\textbf{\newline\newline\Large W} \newline
\begin{acronym_expdlist}
\acro{WAN}{Wide Area Network}
\acro{WS-IS}{Web Service Information Service}
\acro{WSDL}{Web Service Description Language}
\acro{WSRF}{Web Service Resource Framework}
\end{acronym_expdlist}

\textbf{\newline\newline\Large X} \newline
\begin{acronym_expdlist}
\acro{XML} {eXtensible Markup Language}
\end{acronym_expdlist}

%\textbf{\newline\newline\Large Y} \newline
%\begin{acronym_expdlist}
%\end{acronym_expdlist}

%\textbf{\newline\newline\Large Z} \newline
%\begin{acronym_expdlist}
%\end{acronym_expdlist}

% End of list of Acronyms
%%%%%%%%%%%%%%%%%%%%%%%%%%%%%%%%%%%%%%%%%%%%%%%%%%%%%%%%%%%%%%%%%%%%%%



\clearpage      % clear the current page
\thispagestyle{empty}   % prevent header on this (empty -> next line) page
\mbox{}         % to insert a not empty page, so that there is min. 1 blank page before next chapter
\clearpage{\pagestyle{empty}\cleardoublepage}   % clear the current page and start on the right side

%\renewcommand{\tablename}{Table}

\renewcommand{\bibname}{References}     % instead of Bibliography
%\renewcommand\citeform{\thechapter.}
%\addcontentsline{toc}{section}{\listfigurename}   % not really sure

	%% SUMMARY IN DUTCH       %%
%%%%%%%%%%%%%%%%%%%%%%%%%%%%
\selectlanguage{dutch}
\include{chapt_dutch/chapt_dutch}
\selectlanguage{english}
\include{chapt_english/chapt_english}

%% THE BOOK ITSELF   %%
%%%%%%%%%%%%%%%%%%%%%%%
\mainmatter     % related to chappg numbering
\selectlanguage{english}
\renewcommand*{\thesection}{\thechapter.\arabic{section}}

\newcommand\fdtsvrightmarktmp{{\scshape\small Chapter }}
\renewcommand\evenpagerightmark{{\scshape\small\chaptername\ \thechapter}}
\renewcommand\oddpageleftmark{{\scshape\small\leftmark}}

%\addtolength{\headwidth}{\marginparsep}
%\addtolength{\headwidth}{\marginparwidth}

%\newcommand{\tm}[1]{$\mbox{#1}^{\mbox{\emph{\scriptsize TM}}}$}

\baselineskip 13.0pt

\graphicspath{{chapt_dutch/}{intro/}{chapt2/}{chapt3/}{chapt4/}}

% Header
\renewcommand\evenpagerightmark{{\scshape\small Introduction}}
\renewcommand\oddpageleftmark{{\scshape\small Introduction}}

\renewcommand{\bibname}{References}

\hyphenation{}

\chapter[Introduction]%
{Introduction}

%% Introduction
%%%%%%%%%%%%%%%
\section{Introduction}

\section{Outline}

\section{Publications}
\subsection{Publications in international journals}

\subsection{Chapters in international publications}

\subsection{Publications in international conferences}

\subsection{Publications in national conferences}

\clearpage

%\renewcommand*{\thesection}{\thechapter.\arabic{section}}       % reset again to chaptnum.sectnum
\bibliographystyle{phdbib}
\bibliography{PhD}

\clearpage{\pagestyle{empty}\cleardoublepage}

\graphicspath{{chapt_dutch/}{intro/}{chapt2/}{chapt3/}{chapt4/}{chapt5/}}

% Header
\renewcommand\evenpagerightmark{{\scshape\small Grid Monitoring}}
\renewcommand\oddpageleftmark{{\scshape\small Chapter 2}}

\renewcommand{\bibname}{References}

\hyphenation{}

\chapter[Grid Monitoring]%
 {Grid Monitoring}
\label{ch2}

\section{Introduction}
insert text here...

\lipsum

\clearpage

\bibliographystyle{phdbib}
\bibliography{PhD}

\clearpage{\pagestyle{empty}\cleardoublepage}

\graphicspath{{chapt_dutch/}{intro/}{chapt2/}{chapt3/}{chapt4/}{chapt5/}}

% Header
\renewcommand\evenpagerightmark{{\scshape\small Chapter 3}}
\renewcommand\oddpageleftmark{{\scshape\small Grid Simulation}}

\renewcommand{\bibname}{References}

\hyphenation{}

\chapter[Grid Simulation]%
{Grid Simulation}
\label{ch3}

\section{Introduction} 

Write text here

\clearpage

\bibliographystyle{phdbib}
\bibliography{PhD}

\clearpage{\pagestyle{empty}\cleardoublepage}

\graphicspath{{chapt_dutch/}{intro/}{chapt2/}{chapt3/}{chapt4/}{chapt5/}}

% Header
\renewcommand\evenpagerightmark{{\scshape\small Chapter 4}}
\renewcommand\oddpageleftmark{{\scshape\small Grid Service Management}}

\renewcommand{\bibname}{References}

\hyphenation{}

\chapter[Grid Service Management]%
{Grid Service Management}
\label{ch4}

\section{Introduction}

insert text here

\clearpage
%\footnotesize
%\bibliographystyle{ieeetr}
%\bibliography{chapt8}
%\normalsize

\bibliographystyle{phdbib}
\bibliography{PhD}

\clearpage{\pagestyle{empty}\cleardoublepage}

\graphicspath{{chapt_dutch/}{intro/}{chapt2/}{chapt3/}{chapt4/}{chapt5/}}

% Header
\renewcommand\evenpagerightmark{{\scshape\small Chapter 5}}
\renewcommand\oddpageleftmark{{\scshape\small Media Grids}}

\renewcommand{\bibname}{References}

\hyphenation{}

\chapter[Media Grids]%
{Media Grids}
\label{ch5}

\section{Introduction}
insert text...

\clearpage

\bibliographystyle{phdbib}
\bibliography{PhD}

\clearpage{\pagestyle{empty}\cleardoublepage}

\graphicspath{{chapt_dutch/}{intro/}{chapt2/}{chapt3/}{chapt4/}}

% Header
\renewcommand\evenpagerightmark{{\scshape\small Conclusion}}
\renewcommand\oddpageleftmark{{\scshape\small Conclusion}}

\hyphenation{}

\chapter[Overall Conclusion]%
{Overall Conclusion}

insert conclusion here

%\renewcommand*{\thesection}{\thechapter.\arabic{section}}       % reset again to chaptnum.sectnum

\clearpage{\pagestyle{empty}\cleardoublepage}


%\addcontentsline{toc}{section}{}

\appendix
\include{app1/app1}
\include{app2/app2}
\include{app3/app3}

%\renewcommand\fdtsvrightmarktmp{{\scshape\small Appendix }}
\renewcommand\evenpagerightmark{{\scshape\small Appendix \thechapter}}
\renewcommand\oddpageleftmark{{\scshape\small\leftmark}}

\end{document}